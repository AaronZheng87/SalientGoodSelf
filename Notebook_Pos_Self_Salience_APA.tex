% Options for packages loaded elsewhere
\PassOptionsToPackage{unicode}{hyperref}
\PassOptionsToPackage{hyphens}{url}
%
\documentclass[
  man]{apa6}
\usepackage{amsmath,amssymb}
\usepackage{lmodern}
\usepackage{iftex}
\ifPDFTeX
  \usepackage[T1]{fontenc}
  \usepackage[utf8]{inputenc}
  \usepackage{textcomp} % provide euro and other symbols
\else % if luatex or xetex
  \usepackage{unicode-math}
  \defaultfontfeatures{Scale=MatchLowercase}
  \defaultfontfeatures[\rmfamily]{Ligatures=TeX,Scale=1}
\fi
% Use upquote if available, for straight quotes in verbatim environments
\IfFileExists{upquote.sty}{\usepackage{upquote}}{}
\IfFileExists{microtype.sty}{% use microtype if available
  \usepackage[]{microtype}
  \UseMicrotypeSet[protrusion]{basicmath} % disable protrusion for tt fonts
}{}
\makeatletter
\@ifundefined{KOMAClassName}{% if non-KOMA class
  \IfFileExists{parskip.sty}{%
    \usepackage{parskip}
  }{% else
    \setlength{\parindent}{0pt}
    \setlength{\parskip}{6pt plus 2pt minus 1pt}}
}{% if KOMA class
  \KOMAoptions{parskip=half}}
\makeatother
\usepackage{xcolor}
\usepackage{graphicx}
\makeatletter
\def\maxwidth{\ifdim\Gin@nat@width>\linewidth\linewidth\else\Gin@nat@width\fi}
\def\maxheight{\ifdim\Gin@nat@height>\textheight\textheight\else\Gin@nat@height\fi}
\makeatother
% Scale images if necessary, so that they will not overflow the page
% margins by default, and it is still possible to overwrite the defaults
% using explicit options in \includegraphics[width, height, ...]{}
\setkeys{Gin}{width=\maxwidth,height=\maxheight,keepaspectratio}
% Set default figure placement to htbp
\makeatletter
\def\fps@figure{htbp}
\makeatother
\setlength{\emergencystretch}{3em} % prevent overfull lines
\providecommand{\tightlist}{%
  \setlength{\itemsep}{0pt}\setlength{\parskip}{0pt}}
\setcounter{secnumdepth}{-\maxdimen} % remove section numbering
% Make \paragraph and \subparagraph free-standing
\ifx\paragraph\undefined\else
  \let\oldparagraph\paragraph
  \renewcommand{\paragraph}[1]{\oldparagraph{#1}\mbox{}}
\fi
\ifx\subparagraph\undefined\else
  \let\oldsubparagraph\subparagraph
  \renewcommand{\subparagraph}[1]{\oldsubparagraph{#1}\mbox{}}
\fi
\newlength{\cslhangindent}
\setlength{\cslhangindent}{1.5em}
\newlength{\csllabelwidth}
\setlength{\csllabelwidth}{3em}
\newlength{\cslentryspacingunit} % times entry-spacing
\setlength{\cslentryspacingunit}{\parskip}
\newenvironment{CSLReferences}[2] % #1 hanging-ident, #2 entry spacing
 {% don't indent paragraphs
  \setlength{\parindent}{0pt}
  % turn on hanging indent if param 1 is 1
  \ifodd #1
  \let\oldpar\par
  \def\par{\hangindent=\cslhangindent\oldpar}
  \fi
  % set entry spacing
  \setlength{\parskip}{#2\cslentryspacingunit}
 }%
 {}
\usepackage{calc}
\newcommand{\CSLBlock}[1]{#1\hfill\break}
\newcommand{\CSLLeftMargin}[1]{\parbox[t]{\csllabelwidth}{#1}}
\newcommand{\CSLRightInline}[1]{\parbox[t]{\linewidth - \csllabelwidth}{#1}\break}
\newcommand{\CSLIndent}[1]{\hspace{\cslhangindent}#1}
\ifLuaTeX
\usepackage[bidi=basic]{babel}
\else
\usepackage[bidi=default]{babel}
\fi
\babelprovide[main,import]{english}
% get rid of language-specific shorthands (see #6817):
\let\LanguageShortHands\languageshorthands
\def\languageshorthands#1{}
% Manuscript styling
\usepackage{upgreek}
\captionsetup{font=singlespacing,justification=justified}

% Table formatting
\usepackage{longtable}
\usepackage{lscape}
% \usepackage[counterclockwise]{rotating}   % Landscape page setup for large tables
\usepackage{multirow}		% Table styling
\usepackage{tabularx}		% Control Column width
\usepackage[flushleft]{threeparttable}	% Allows for three part tables with a specified notes section
\usepackage{threeparttablex}            % Lets threeparttable work with longtable

% Create new environments so endfloat can handle them
% \newenvironment{ltable}
%   {\begin{landscape}\centering\begin{threeparttable}}
%   {\end{threeparttable}\end{landscape}}
\newenvironment{lltable}{\begin{landscape}\centering\begin{ThreePartTable}}{\end{ThreePartTable}\end{landscape}}

% Enables adjusting longtable caption width to table width
% Solution found at http://golatex.de/longtable-mit-caption-so-breit-wie-die-tabelle-t15767.html
\makeatletter
\newcommand\LastLTentrywidth{1em}
\newlength\longtablewidth
\setlength{\longtablewidth}{1in}
\newcommand{\getlongtablewidth}{\begingroup \ifcsname LT@\roman{LT@tables}\endcsname \global\longtablewidth=0pt \renewcommand{\LT@entry}[2]{\global\advance\longtablewidth by ##2\relax\gdef\LastLTentrywidth{##2}}\@nameuse{LT@\roman{LT@tables}} \fi \endgroup}

% \setlength{\parindent}{0.5in}
% \setlength{\parskip}{0pt plus 0pt minus 0pt}

% Overwrite redefinition of paragraph and subparagraph by the default LaTeX template
% See https://github.com/crsh/papaja/issues/292
\makeatletter
\renewcommand{\paragraph}{\@startsection{paragraph}{4}{\parindent}%
  {0\baselineskip \@plus 0.2ex \@minus 0.2ex}%
  {-1em}%
  {\normalfont\normalsize\bfseries\itshape\typesectitle}}

\renewcommand{\subparagraph}[1]{\@startsection{subparagraph}{5}{1em}%
  {0\baselineskip \@plus 0.2ex \@minus 0.2ex}%
  {-\z@\relax}%
  {\normalfont\normalsize\itshape\hspace{\parindent}{#1}\textit{\addperi}}{\relax}}
\makeatother

% \usepackage{etoolbox}
\makeatletter
\patchcmd{\HyOrg@maketitle}
  {\section{\normalfont\normalsize\abstractname}}
  {\section*{\normalfont\normalsize\abstractname}}
  {}{\typeout{Failed to patch abstract.}}
\patchcmd{\HyOrg@maketitle}
  {\section{\protect\normalfont{\@title}}}
  {\section*{\protect\normalfont{\@title}}}
  {}{\typeout{Failed to patch title.}}
\makeatother

\usepackage{xpatch}
\makeatletter
\xapptocmd\appendix
  {\xapptocmd\section
    {\addcontentsline{toc}{section}{\appendixname\ifoneappendix\else~\theappendix\fi\\: #1}}
    {}{\InnerPatchFailed}%
  }
{}{\PatchFailed}
\keywords{Perceptual decision-making, Self positivity bias, moral character\newline\indent Word count: X}
\DeclareDelayedFloatFlavor{ThreePartTable}{table}
\DeclareDelayedFloatFlavor{lltable}{table}
\DeclareDelayedFloatFlavor*{longtable}{table}
\makeatletter
\renewcommand{\efloat@iwrite}[1]{\immediate\expandafter\protected@write\csname efloat@post#1\endcsname{}}
\makeatother
\usepackage{lineno}

\linenumbers
\usepackage{csquotes}
\usepackage{rotating}
\DeclareDelayedFloatFlavor{sidewaysfigure}{figure}
\ifLuaTeX
  \usepackage{selnolig}  % disable illegal ligatures
\fi
\IfFileExists{bookmark.sty}{\usepackage{bookmark}}{\usepackage{hyperref}}
\IfFileExists{xurl.sty}{\usepackage{xurl}}{} % add URL line breaks if available
\urlstyle{same} % disable monospaced font for URLs
\hypersetup{
  pdftitle={The good person is me: Spontaneous binding with the self prioritizes good character},
  pdfauthor={Hu Chuan-Peng1, 2, Kaiping Peng2, \& Jie Sui3},
  pdflang={en-EN},
  pdfkeywords={Perceptual decision-making, Self positivity bias, moral character},
  hidelinks,
  pdfcreator={LaTeX via pandoc}}

\title{The good person is me: Spontaneous binding with the self prioritizes good character}
\author{Hu Chuan-Peng\textsuperscript{1, 2}, Kaiping Peng\textsuperscript{2}, \& Jie Sui\textsuperscript{3}}
\date{}


\shorttitle{Priorization of moral character}

\authornote{

Hu Chuan-Peng, School of Psychology, Nanjing Normal University, 210024 Nanjing, China.
Kaiping Peng, Department of Psychology, Tsinghua University, 100084 Beijing, China.
Jie Sui, School of Psychology, University of Aberdeen, Aberdeen, Scotland.
Authors contriubtion: HCP, JS, \& KP design the study, HCP collected the data, HCP analyzed the data and drafted the manuscript. All authors read and agreed upon the current version of the manuscripts.

Correspondence concerning this article should be addressed to Hu Chuan-Peng, School of Psychology, Nanjing Normal University, Ninghai Road 122, Gulou District, 210024 Nanjing, China. E-mail: \href{mailto:hcp4715@hotmail.com}{\nolinkurl{hcp4715@hotmail.com}}

}

\affiliation{\vspace{0.5cm}\textsuperscript{1} Nanjing Normal University, 210024 Nanjing, China\\\textsuperscript{2} Tsinghua University, 100084 Beijing, China\\\textsuperscript{3} University of Aberdeen, Aberdeen, Scotland}

\abstract{%
Moral character is central to social evaluation and moral judgment. As such, information related to moral character is prioritized in human cognition. This effect is usually explained as a valence effect. Here we report 9 experiments (N = 4XX, trials = XXX) which reveal (1) there is a robust good character prioritization effect in a perceptual matching task, i.e., when neutral geometric shapes were associated with good character, they were prioritized as compared to shapes associated with neutral or bad characters; (2) the prioritization of good character was robust only when the good character referred to the self but weak or non-exist when it referred to others, suggesting a binding effect of the self; (3) the binding between the self and good character exist even when one of them was task-irrelevant. Together, these results provided evidence for spontaneous self-referential processing, i.e., binding the good character with self, as a novel mechanism of the prioritization effect of good character.
}



\begin{document}
\maketitle

Alternative title: Self-relevance modulates the priorization of the good character in perceptual matching

\hypertarget{introduction}{%
\section{Introduction}\label{introduction}}

{[}quotes about moral character{]}

{[}Morality is central to social life, moral character is the central of morality{]} \textbf{People experience a substantial amount of moral events in everyday life (e.g., Hofmann, Wisneski, Brandt, \& Skitka, 2014) and judging the moral character of people is indispensable part of these events}. Whether we are the agent, target, or a third party of a moral event, we always judge moral behaviors as ``right'' or ``wrong'', and by doing so, we judge people as ``good'' or ``bad'' (Uhlmann, Pizarro, \& Diermeier, 2015). Moral character is so important in social life that it is a basic dimension in our social evaluation (Goodwin, 2015; Goodwin, Piazza, \& Rozin, 2014) and that a substantial part of people's conversation are gossiping others' moral character (or, reputation) (e.g., Dunbar, 2004). These moral character information may help us to evaluate our in-group members and distinguish out-group members (Ellemers, 2018).

{[}Two possibilities about moral character{]}
Given the importance of moral character and limited cognitive resources to process all the information in a social world, will people prioritize information with certain moral character? Focus on the valence of moral character, previous studies explore both negativity effect and positivity effect. The negativity effect, i.e., `bad' character are prioritized, is consistent with early studies in impression formation which found that negative traits are weighted more in overall impression (N. H. Anderson, 1965; Fiske, 1980; Skowronski \& Carlston, 1987). This idea also seemed to consistent with the more general idea that ``bad is stronger than good'' (Baumeister, Bratslavsky, Finkenauer, \& Vohs, 2001; Pratto \& John, 1991). A few studies provided evidence for this possibility. For example, E. Anderson, Siegel, Bliss-Moreau, and Barrett (2011) asked participants to associate faces with different behaviors (e.g., negative and neutral behaviors from both social and nonsocial domains) and then perform a binocular rivalry task, where a face and a building were presented to each eye. Participants were required report the content of their visual awareness by pressing buttons. The results revealed that faces associated negative social behaviors dominated participants' visual awareness longer than faces associated with other types of behaviors (but see Stein, Grubb, Bertrand, Suh, \& Verosky, 2017). Similarly, Eiserbeck and Abdel Rahman (2020) combined associative learning with attention blink paradigm, where neutral faces were associated with sentences about neutral or negative trust behaviors. They also found that neutral faces associated with negative behavior were processed preferentially.

The positivity effect, i.e., good moral characters are prioritized, is also plausible (see recent reviews, Pool, Brosch, Delplanque, \& Sander, 2016; Unkelbach, Alves, \& Koch, 2020). Unkelbach et al. (2020) pointed out that bad is not necessarily stronger than good in all aspects of information processing. Sometimes, good is stronger than bad. For example, when participants are asked to classify words as good or bad, positive trait words are classified faster than negative words (Bargh, Chaiken, Govender, \& Pratto, 1992). Similarly, in a lexical decision task, participants judge positive words faster than negative words (Unkelbach et al., 2010). Also, Anisfeld and Lambert (1966) found that positive words are easier to associate with nonsense word-like strings, and this advantage in associative potential also appeared in implicit association test (IAT) (Anselmi, Vianello, \& Robusto, 2011). Direct evidence for positivity effect of moral character also exist: Shore and Heerey (2013) found that faces with positive interaction in a trust game were prioritized in pre-attentive process.

These two possibilities, however, ignore the agency of participants who is perceiving the information and making perceptual decisions. The external stimuli only contain subjective value if they are relevant to the self of the decision-maker {[}{]}. When it comes to moral character, there are long-history of studies showing that moral character is central for people's self-concept and identity. A positive moral character is viewed as the core feature of identity (e.g., Strohminger, Knobe, \& Newman, 2017). A lot of studies revealed that people distort their perception, memory, and change their actions to maintain a positive view of their moral self-view. Given this strong motivation, it is possible that participant has spontaneous self-referential for the perception tasks where no self-referential process were not explicitly excluded {[}citation related to spontaneous self-referential{]}.

Here, we report nine experiments where we found (1) there is a robust good character prioritization effect in social associative learning task, i.e., when neutral geometric shapes were associated with good character, they were prioritized as compared to shapes associated with neutral or bad characters; (2) prioritization of good character was robust only when it is relevant to the self but weak or non-exist when it referred to a non-self label; (3) the binding between good character and self exist even when one of the label became task-irrelevant. Together, these results provided evidence for spontaneous self-referential processing as a novel mechanism of the prioritization effect of good character. In all experiments, a social associative learning task in which th effect of physical features are minimized --- participants performed a perceptual matching task after associated different moral characters (good, neutral, and bad) with different geometric shapes.

\hypertarget{disclosures}{%
\section{Disclosures}\label{disclosures}}

We reported all the measurements, analyses, and results in all the experiments in the current study. Participants whose overall accuracy lower than 60\% were excluded from analysis. Also, the accurate responses with less than 200ms reaction times were excluded from the analysis. These excluded data can be found in the shared raw data files.

All the experiments reported were not pre-registered. Most experiments (1a \textasciitilde{} 4b, except experiment 3b) reported in the current study were first finished between 2013 to 2016 in Tsinghua University, Beijing, China. Participants in these experiments were recruited in the local community. To increase the sample size of experiments to 50 or more (Simmons, Nelson, \& Simonsohn, 2013), we recruited additional participants in Wenzhou University, Wenzhou, China in 2017 for experiment 1a, 1b, 4a, and 4b. Experiment 3b was finished in Wenzhou University in 2017 (See Table S1 for overview of these experiments).

All participant received informed consent and compensated for their time. These experiments were approved by the ethic board in the Department of Psychology, Tsinghua University.

\hypertarget{general-methods}{%
\section{General methods}\label{general-methods}}

\hypertarget{design-and-procedure}{%
\subsection{Design and Procedure}\label{design-and-procedure}}

This series of experiments studied the perceptual process of moral character, using the social associative learning paradigm (or tagging paradigm, see (Sui, He, \& Humphreys, 2012), in which participants first learned the associations between geometric shapes and labels of person with different moral character (e.g., in first three studies, the triangle, square, and circle and good person, neutral person, and bad person, respectively). The associations of the shapes and label were counterbalanced across participants. After remembered the associations, participants finished a practice phase to familiar with the task, in which they viewed one of the shapes upon the fixation while one of the labels below the fixation and judged whether the shape and the label matched the association they learned. When participants reached 60\% or higher accuracy at the end of the practicing session, they started the experimental task which was the same as in the practice phase.

The experiment 1a, 1b, 1c, 2, 5, and 6a shared a 2 (matching: match vs.~nonmatch) by 3 (moral character: good vs.~neutral vs.~bad person) within-subject design. Experiment 1a was the first one of the whole series studies and found the prioritization of stimuli associated with good-person. To confirm that it is the moral character that caused the effect, we further conducted experiment 1b, 1c, and 2. More specifically, experiment 1b used different Chinese words as labels to test whether the effect only occurred with certain words. Experiment 1c manipulated the moral valence indirectly: participants first learned to associate different moral behaviors with different Chinese names, after remembered the association, they then performed the perceptual matching task by associating names with different shapes. Experiment 2 further tested whether the way we presented the stimuli influence the effect of valence, by sequentially presenting labels and shapes. Note that part of participants of experiment 2 were from experiment 1a because we originally planned a cross task comparison. Experiment 5 was designed to compare the effect size of moral character and other importance social evaluative dimensions (aesthetics and emotion). Different social evaluative dimensions were implemented in different blocks, the moral character blocks shared the design of experiment 1a. Experiment 6a, which shared the same design as experiment 2, was an EEG experiment which aimed at exploring the neural correlates of the effect. But we will focus on the behavioral results of experiment 6a in the current manuscript.

For experiment 3a, 3b, and 6b, we included self-reference as another within-subject variable in the experimental design. For example, the experiment 3a directly extend the design of experiment 1a into a 2 (matching: match vs.~nonmatch) by 2 (reference: self vs.~other) by 3 (moral character: good vs.~neutral vs.~bad) within-subject design. Thus in experiment 3a, there were six conditions (good-self, neutral-self, bad-self, good-other, neutral-other, and bad-other) and six shapes (triangle, square, circle, diamond, pentagon, and trapezoids). The experiment 6b was an EEG experiment based on experiment 3a but presented the label and shape sequentially. Because of the relatively high working memory load (six label-shape pairs), experiment 6b were conducted in two days: the first day participants finished perceptual matching task as a practice, and the second day, they finished the task again while the EEG signals were recorded. We only focus on the first day's data here. Experiment 3b was designed to separate the self-referential trials and other-referential trials. That is, participants finished two different types of block: in the self-referential blocks, they only responded to good-self, neutral-self, and bad-self, with half match trials and half nonmatch trials; in the other-reference blocks, they only responded to good-other, neutral-other, and bad-other.

Experiment 4a and 4b were design to explore the mechanism underlying the prioritization of good-self. In 4a, we only used two labels (self vs.~other) and two shapes (circle, square). To manipulate the moral character, we added the moral-related words within the shape and instructed participants to ignore the words in the shape during the task. In 4b, we reversed the role of self-reference and moral character in the task: participant learned three labels (good-person, neutral-person, and bad-person) and three shapes (circle, square, and triangle), and the words related to identity, ``self'' or ``other'', were presented in the shapes. As in 4a, participants were told to ignore the words inside the shape during the task.

E-prime 2.0 was used for presenting stimuli and collecting behavioral responses. For participants recruited in Tsinghua University, they finished the experiment individually in a dim-lighted chamber, stimuli were presented on 22-inch CRT monitors and their head were fixed by a chin-rest brace. The distance between participants' eyes and the screen was about 60 cm. The visual angle of geometric shapes was about \(3.7^\circ × 3.7^\circ\), the fixation cross is of \(0.8^\circ × 0.8^\circ\) visual angle at the center of the screen. The words were of \(3.6^\circ\) × \(1.6^\circ\) visual angle. The distance between the center of the shape or the word and the fixation cross was \(3.5^\circ\) of visual angle. For participants recruited in Wenzhou University, they finished the experiment in a group consisted of 3 \textasciitilde{} 12 participants in a dim-lighted testing room. Participants were required to finished the whole experiment independently. Also, they were instructed to start the experiment at the same time, so that the distraction between participants were minimized. The stimuli were presented on 19-inch CRT monitor. The visual angles are could not be exactly controlled because participants' chin were not fixed.

In most of these experiments, participant were also asked to fill a battery of questionnaire after they finish the behavioral tasks. All the questionnaire data are open (see, dataset 4 in Liu et al., 2020). See Table S1 for a summary information about all the experiments.

\hypertarget{data-analysis}{%
\subsection{Data analysis}\label{data-analysis}}

We used the \texttt{tidyverse} of r (see script \texttt{Load\_save\_data.r}) to preprocess the data. Results of each experiment were then analyzed using Bayesian hierarchical models.

We used the Bayesian hierarchical model (BHM, or Bayesian generalized linear mixed models, Bayesian multilevel models) to model the reaction time and accuracy data, because BHM provided three advantages over the classic NHST approach (repeated measure ANOVA or \emph{t}-tests): first, BHM estimate the posterior distributions of parameters for statistical inference, therefore provided uncertainty in estimation (Rouder \& Lu, 2005). Second, BHM, where generalized linear mixed models could be easily implemented, can use distributions that fit the distribution of real data instead of using normal distribution for all data. Using appropriate distributions for the data will avoid misleading results and provide better fitting of the data. For example, Reaction times are not normally distributed but right skewed, and the linear assumption in ANOVAs is not satisfied (Rousselet \& Wilcox, 2019). Third, BHM provided an unified framework to analyze data from different levels and different sources, avoid the information loss when we need to combine data from different levels.

We used the \texttt{r} package \texttt{BRMs} (Bürkner, 2017), which used Stan (Carpenter et al., 2017) for the BHM analyses. We estimated the over-all effect across experiments with similar experimental design, instead of using a two-step approach where we first estimate parameters, e.g., \(d'\) for each participant, and then use a random effect model meta-analysis to synthesize the effect (Goh, Hall, \& Rosenthal, 2016).

\hypertarget{accuracy}{%
\subsubsection{Accuracy}\label{accuracy}}

We followed practice of previous studies (Hu, Lan, Macrae, \& Sui, 2020; Sui et al., 2012) and used signal detection theory approach to analyze the accuracy data. More specifically, the match trials are treated as signal and the non-match trials are noise. As we mentioned above, we estimated the sensitivity and criterion of SDT by BHM (Rouder \& Lu, 2005). Because the BHM can model different level's data using a single unified model, we used a three-level HBM to model the moral character effect, which include five experiments: 1a, 1b, 1c, 2, 5, and 6a. Similarly, we modeled experiments with both self-referential and moral character with a three-level HBM model, which includes 3a, 3b, and 6b. For experiment 4a and 4b, we used two-level models for each separately. However, we could compare the posterior of parameters directly because we have full posterior distribution of parameters.

We used the Bernoulli distribution to model the accuracy data. For a single participant, we assume that the accuracy of \(i\)th trial is Bernoulli distributed (binomial with 1 trial), with probability \(p_{i}\) that \(y_{i} = 1\).

\[ y_{i} \sim Bernoulli(p_{i})\]
and the probability of choosing ``match'' \(p_{i}\) at the \(i\)th trial is a function of the trial type:

\[ \Phi(p_{i}) =  \beta_{0} + \beta_{1}IsMatch_{i}\]
therefore, the outcomes \(y_{i}\) are 0 if the participant responded ``nonmatch'' on the \(i\)th trial, 1 if they responded ``match''. We then write the generalized linear model on the probits (z-scores; \(\Phi\), ``Phi'') of \(p\)s. \(\Phi\) is the cumulative normal density function and maps \(z\) scores to probabilities. In this way, the intercept of the model (\(\beta_0\)) is the standardized false alarm rate (probability of saying 1 when predictor is 0), which we take as our criterion \(c\). The slope of the model (\(\beta_1\)) is the increased probability of responding ``match'' when the trial type is ``match'', in \(z\)-scores, which is another expression of \(d'\). Therefore, \(c\) = -\(z\)HR = \(-\beta_0\), and \(d' = \beta_1\).

In our experimental design, there are three conditions for both match and non-match trials, we can estimate the \(d'\) and \(c\) separately for each condition. In this case, the criterion \(c\) is modeled as the main effect of valence, and the \(d'\) can be modeled as the interaction between valence and match:

\[ \Phi(p_{i}) = 0 + \beta_{0}Valence_{i} + \beta_{1}IsMatch_{i}  * Valence_{i} \]

In each experiment, we had multiple participants. We can estimate the group-level parameters by extending the above model into a two-level model, where we can estimate parameters on individual level (varying effect) and the group level parameter simultaneously (fixed effect). The probability that the \(j\)th subject responded ``match'' (\(y_{ij} = 1\)) at the \(i\)th trial \(p_{ij}\). In the same vein, we have

\[ y_{ij} \sim Bernoulli(p_{ij})\]
The the generalized linear model can be re-written to include two levels:
\[ \Phi(p_{ij}) = 0 + \beta_{0j}Valence_{ij} + \beta_{1j}IsMatch_{ij} * Valence_{ij}\]
We again can write the generalized linear model on the probits (z-scores; \(\Phi\), ``Phi'') of \(p\)s.

The subjective-specific intercepts (\(\beta_{0} = -zFAR\)) and slopes (\(\beta_{1} = d'\)) are describe by multivariate normal with means and a covariance matrix for the parameters.
\[ \begin{bmatrix}\beta_{0j}\\
\beta_{1j}\\
\end{bmatrix} \sim N(\begin{bmatrix}\theta_{0}\\
\theta_{1}\\
\end{bmatrix}, \sum) \]

For experiments that had 2 (matching: match vs.~non-match) by 3 (moral character: good vs.~neutral vs.~bad), i.e., experiment 1a, 1b, 1c, 2, 5, and 6a, the formula for accuracy in \texttt{BRMs} is as follow:

\texttt{saymatch\ \textasciitilde{}\ 0\ +\ Valence\ +\ Valence:ismatch\ +\ (0\ +\ Valence\ +\ Valence:ismatch\ \textbar{}\ Subject),\ family\ =\ bernoulli(link="probit")}

For experiments that had two by two by three design, we used the follow formula for the BGLM:

\texttt{saymatch\ \textasciitilde{}\ 0\ +\ ID:Valence\ +\ ID:Valence:ismatch\ +\ (0\ +\ ID:Valence\ +\ ID:Valence:ismatch\ \textbar{}\ Subject),\ family\ =\ bernoulli(link="probit")}

In the same vein, we can estimate the posterior of parameters across different experiments. We can use a nested hierarchical model to model all the experiment with similar design:
\[y_{ijk} \sim Bernoulli(p_{ijk})\]
the generalized linear model is then
\[ \Phi(p_{ijk}) =  0 + \beta_{0jk}Valence_{ijk} + \beta_{1j}IsMatch_{ijk} * Valence_{ijk}\]
The outcomes \(y_{ijk}\) are 0 if participant \(j\) in experiment k responded ``nonmatch'' on trial \(i\), 1 if they responded ``match''.

\[\begin{bmatrix}\beta_{0jk}\\
\beta_{1jk}\\
\end{bmatrix} \sim N(\begin{bmatrix}\theta_{0k}\\
\theta_{1k}\\
\end{bmatrix}, \sum)\]

and the experiment level parameter \(mu_{0k}\) and \(mu_{1k}\) is from a higher order distribution:

\[\begin{bmatrix}\theta_{0k}\\
\theta_{1k}\\
\end{bmatrix} \sim N(\begin{bmatrix}\mu_{0}\\
\mu_{1}\\
\end{bmatrix}, \sum)\]
in which \(mu_{0}\) and \(mu_{1}\) means the population level parameter.

\hypertarget{reaction-times}{%
\subparagraph{Reaction times}\label{reaction-times}}

For the reaction time, we used the log normal distribution (\url{https://lindeloev.github.io/shiny-rt/\#34_(shifted)_log-normal}) to model the data. This means that we need to estimate the posterior of two parameters: \(\mu\), \(\sigma\). \(\mu\) is the mean of the \texttt{logNormal} distribution, and \(\sigma\) is the disperse of the distribution. Although the log normal distribution can be extended to shifted log normal distribution, with one more parameter: shift, which is the earliest possible response, we found that the additional parameter didnt' improved the model fitting and therefore used the logNormal in our final analysis.

The reaction time of the \(j\)th subject on \(i\)th trial is a linear function of trial type: \[y_{ij} = \beta_{0j} + \beta_{1j}*IsMatch_{ij} * Valence_{ij}\]

while the log of the reaction time is log-normal distributed:
\[ log(y_{ij}) \sim N(\mu_{j}, \sigma_{j})\]
\(y_{ij}\) is the RT of the \(i\)th trial of the \(j\)th participants.

\[\mu_{j} \sim N(\mu, \sigma)\]

\[\sigma_{j} \sim Cauchy()\]
Formula used for modeling the data as follow:

\texttt{RT\_sec\ \textasciitilde{}\ Valence*ismatch\ +\ (Valence*ismatch\ \textbar{}\ Subject),\ family\ =\ lognormal()}

or

\texttt{RT\_sec\ \textasciitilde{}\ ID*Valence*ismatch\ +\ (ID*Valence*ismatch\ \textbar{}\ Subject),\ family\ =\ lognormal()}

we expanded the RT model three-level model in which participants and experiments are two group level variable and participants were nested in the experiments.

\[ log(y_{ijk}) \sim N(\mu_{jk}, \sigma_{jk})\]

\(y_{ijk}\) is the RT of the \(i\)th trial of the \(j\)th participants in the \(k\)th experiment.

\[\mu_{jk} \sim N(\mu_{k}, \sigma_{k})\]
\[\sigma_{jk} \sim Cauchy()\]
\[\mu_{k} \sim N(\mu, \sigma)\]
\[\theta_{k} \sim Cauchy()\]

\hypertarget{prioritization-of-good-character}{%
\subsubsection{Prioritization of good character}\label{prioritization-of-good-character}}

We estimated the effect size of \(d'\) and RT from experiment 1a, 1b, 1c, 2, 5, and 6a for the effect of moral character. We reported fixed effect of three-level BHM that included all experiments that tested the valence effect.

\hypertarget{prioritization-of-good-character-is-modulated-by-self-referential}{%
\subsubsection{Prioritization of good character is modulated by self-referential}\label{prioritization-of-good-character-is-modulated-by-self-referential}}

We also estimated the interaction between moral character and self-referential process, which included results from experiment 3a, 3b, and 6b. Using three-level models, we tested two possible explanations for the prioritization of good character: value-based or social categorization based prioritization.

\hypertarget{spontaenous-binding-between-self-and-good-character}{%
\subsubsection{Spontaenous binding between self and good character}\label{spontaenous-binding-between-self-and-good-character}}

In the third part, we focused on experiment 4a and 4b, which were designed to examine two more nuanced explanation concerning the good-self. The design of experiment 4a and 4b are complementary. Together, they can test whether participants are more sensitive to the moral character of the Self (4a), or the identity of the good character (4b).

We did not analyze the questionnaire data, which were described in (Liu et al., 2020).

\hypertarget{results}{%
\section{Results}\label{results}}

\hypertarget{prioritization-of-good-character-related-information}{%
\subsection{Prioritization of good character related information}\label{prioritization-of-good-character-related-information}}

In this part, we report results from five experiments that tested whether an associative learning task, including 192 participants. Note that for both experiment 1a and 1b, there were two independent samples with different equipment, trials numbers and testing situations. Therefore, we modeled them as independent samples. These five experiments revealed a robust effect of moral character on perceptual matching task.

\begin{figure}
\centering
\includegraphics{Notebook_Pos_Self_Salience_APA_files/figure-latex/plot-bayes-meta-1-1.pdf}
\caption{\label{fig:plot-bayes-meta-1}Effect of moral valence on RT and d'}
\end{figure}

For the \emph{d} prime, we found robust effect of moral character. Shapes associated with good character (``good person'', ``kind person'' or a name associated with morally good behavioral history) has higher sensitivity (median = 2.49, 95\% HDI = {[}2.19 2.75{]}) than shapes associated with neutral character (median = 2.18, 95\% HDI = {[}1.90 2.48{]}), \(median_{diff}\) = 0.31, 95\% HDI {[}0.02 0.63{]} , but we did not find differences between shapes associated with bad character (median = 2.23, 95\% HDI = {[}1.94 2.53{]}) and neutral character, \(median_{diff}\) = 0.05, 95\% HDI {[}-0.29 0.37{]}.

For the reaction times, we also found robust effect of moral character for both match trials (see figure \ref{fig:plot-bayes-meta-1} C) and nonmatch trials (\textbf{see supplementary materials}). For match trials, shapes associated with good character has faster responses (median = 578.64 ms, 95\% HDI = {[}508.15 661.14{]}) than shapes associated with neutral character (median = 623.45 ms, 95\% HDI = {[}547.98 708.24{]}), \(median_{diff}\) = -44.05, 95\% HDI {[}-59.96 -30.43{]}. We also found that the responses to shapes associated with bad character (median = 640.41 ms, 95\% HDI = {[}559.94 719.63{]}) were slower as compared to the neutral character, \(median_{diff}\) = 17.04, 95\% HDI {[}4.02 29.92{]}. See Figure \ref{fig:plot-bayes-meta-1}.

For the nonmatch trials, we also found the advantage of good character: Shapes associated with good character (median = 653.21 ms, 95\% HDI = {[}574.65 739.57{]}) are faster than shapes associated with neutral (median = 671.14 ms, 95\% HDI = {[}591.71 760.09{]}), \(median_{diff}\) = -17.65 ms, 95\% HDI {[}-23.85 -10.36{]}. Similarly, the shapes associated with bad character (median = 676.35 ms, 95\% HDI = {[}599.13 767.76{]}) was responded slower than shapes associated with neutral character, \(median_{diff}\) = 17.04 ms, 95\% HDI {[}4.02 29.92{]}, but the effect size was smaller, (\textbf{see supplementary materials}).

\hypertarget{self-referential-process-modulates-prioritization-of-good-character}{%
\subsection{Self-referential process modulates prioritization of good character}\label{self-referential-process-modulates-prioritization-of-good-character}}

In this part, we report results from three experiments (3a, 3b, and 6b) that aimed at testing whether the moral valence effect found in the previous experiments is modulated by self-referential processes. These three experiments included data from 108 participants.

Because we have found that a facilitation effect of good character and slow-down effect of bad character in the first part, in this part, we will focus on the whether such effect interact with self-referential factor. In others words, we not only reported differences between good/bad character with neutral character for self-referential and other-referential separately, but also compare the differences between the difference. For details of individual studies, please see supplementary materials.

\begin{figure}
\centering
\includegraphics{Notebook_Pos_Self_Salience_APA_files/figure-latex/plot-bayes2-1.pdf}
\caption{\label{fig:plot-bayes2}Interaction between moral valence and self-referential}
\end{figure}

For the \emph{d} prime, we found that an interaction between moral character effect and self-referential, the self- and other-referential difference was greater than zero for good vs.~neutral character differences (\(median_{diff}\) = 0.51; 95\% HDI = {[}-1.48 2.61{]}) but not for bad vs.~neutral differences (\(median_{diff}\) = -0.02; 95\% HDI = {[}-1.85 2.17{]}). Further analyses revealed that the good vs.~neutral character effect only appeared for self-referential conditions but not other-referential conditions. The estimated \emph{d} prime for good-self was greater than neutral-self (\(median_{diff}\) = 0.56; 95\% HDI = {[}-1.05 2.15{]}), \emph{d} prime for good-self was also greater than good-other condition (\(median_{diff}\) = ; 95\% HDI = {[} {]}). The differences between bad-self and neutral-self, good-other and neutral-other, bad-other and neutral-other are all centered around zero (see Figure \ref{fig:plot-bayes2}, B, D).

For the RTs part, we also found the interaction between moral character and self-referential, the self- and other-referential differences was below zero for the good vs.~neutral differences (\(median_{diff}\) = -105.39; 95\% HDI = {[}-533.16 281.69{]}) but not for the bad vs.~neutral differences (\(median_{diff}\) = -9.46; 95\% HDI = {[}-290.72 251.38{]}). Further analyses revealed a robust good-self prioritization effect as compared to neutral-self (\(median_{diff}\) = -47.58; 95\% HDI = {[}-202.88 16.83{]}) and good-other (\(median_{diff}\) = -57.14; 95\% HDI = {[}-991.89 621.29{]}) conditions. Also, we found that both good character and bad character were responded slower than neutral character when it was other-referential. See Figure \ref{fig:plot-bayes2}.

These results suggested that the prioritization of good character is modulated by the self-referential processing: when the good character was prioritized when it was self-referential, but it was slowed down when it was other-referential.

\hypertarget{spontaneous-binding-between-the-good-character-and-the-self}{%
\subsection{Spontaneous binding between the good character and the self}\label{spontaneous-binding-between-the-good-character-and-the-self}}

Two studies further tested whether the binding between self and good character happen even when two aspect of information are separated and only one of them is task-relevant. We are interested in testing whether the task-relevance modulated the effect observed in previous experiment.

In experiment 4a, where self- and other-referential were task-relevant and moral character are task-irrelevant. We found self-related conditions were performed better than other-related conditions, on both \emph{d} prime and reaction times. This pattern is consistent with previous studies (e.g., Sui et al. (2012)).

More importantly, we found evidence, albeit weak, that task-irrelevant moral character also played an role. For shapes associated with self, \emph{d'} was greater when shapes had a good character inside the shape (median = 2.83, 95\% HDI {[}2.63 3.01{]}) than shapes that have neutral character (median = 2.74, 95\% HDI {[}2.58 2.95{]}, BF = 4.4) or bad character (median = 2.76, 95\% HDI {[}2.56 2.95{]}, 3.1), but we did not found difference between shapes with bad character and neutral character inside for the self-referential shapes. For shapes associated with other, the results of \emph{d'} revealed a reversed pattern to the self-referential condition: \emph{d} prime was smaller when shapes had a good character inside (median = 1.87, 95\% HDI {[}1.71 2.04{]}) than had neutral (median = 1.96, 95\% HDI {[}1.80 2.14{]}) or bad character (median = 1.98, 95\% HDI {[}1.79 2.17{]}) inside. See Figure \ref{fig:plot-exp4-all}.

The same pattern was found for RTs. For self-referential condition, when good character was presented as a task-irrelevant stimuli, the responds (median = 641, 95\% HDI {[}623 662{]}) were faster than when neutral character (median = 649, 95\% HDI {[}631 668{]}) or bad character (median = 648, 95\% HDI {[}628 667{]}) were inside. This effect was reversed for other-referential condition: shapes associated with other with good character inside (median = 733, 95\% HDI {[}711 754{]}) were slower than with neutral character (median = 721, 95\% HDI {[}702 741{]}) or bad character (median = 718, 95\% HDI {[}696 740{]}) inside.

\begin{figure}
\centering
\includegraphics{Notebook_Pos_Self_Salience_APA_files/figure-latex/plot-exp4-all-1.pdf}
\caption{\label{fig:plot-exp4-all}exp4: Results of Bayesian GLM analysis.}
\end{figure}

In experiment 4b, moral character was the task-relevant factor, and we found that there were main effect of moral character: shapes associated with good character were performed better than other-related conditions, on both \emph{d'} and reaction times.

Most importantly, we found evidence that task-irrelevant self-referential process also played an role. For shapes associated with good person, the \emph{d} prime was greater when shapes had an ``self'' inside than with ``other'' inside (\(mean_{diff}\) = 0.14, 95\% credible intervals {[}-0.02, 0.31{]}, BF = 12.07), but this effect did not happen when the target shape where associated with ``neutral'' (\(mean_{diff}\) = 0.04, 95\% HDI {[}-.11, .18{]}) or ``bad'' person (\(mean_{diff}\) = -.05, 95\% HDI{[}-.18, .09{]}).

The same trend appeared for the RT data. For shapes associated with good person, with a ``self'' inside the shape reduced the reaction times as compared with when a ``other'' inside the shape (\(mean_{diff}\) = -55 ms, 95\% HDI {[}-75, -35{]}), but this effect did not occur when the shapes were associated neutral (\(mean_{diff}\) = 10, 95\% HDI {[}1, 20{]}) or bad (\(mean_{diff}\) = 5, 95\% HDI {[}-16, 27{]}) person. See Figure \ref{fig:plot-exp4-all}.

\hypertarget{discussion}{%
\section{Discussion}\label{discussion}}

{[}Summary of results{]} Across nine experiments, we explored the prioritization effect of moral character and the underlying mechanism by a combination of social associative learning and perceptual matching task. We found robust effect that good character was prioritized in the shape-label matching task, regardless how good character was represented (single word or behavioral description). Further more, we found this prioritization of good character was modulated by a self-referential processing: prioritization only occur when moral characters are self-referential but not other-referential. This modulation effect of the prioritization of good character occurred even when self- or character- related information are irrelevant to the task. In contrast, when good character became other-referential, even implicitly, the information process was slowed down. Together, these findings highlight the importance of the self-referential in perceiving information related to good character, contributed to a growing literature on the social nature of perception (Freeman, Stolier, \& Brooks, 2020; Xiao, Coppin, \& Bavel, 2016) by supporting the idea that people can prioritized not just physically salient, or affective stimuli, but also socially salient stimuli, i.e., instantly acquired moral information.

{[}Effect of good character{]} The robust effect of the prioritization of good character provide solid evidence for the notion that morality do have an impact in perceptual decision making {[}van Bavel{]}. Previous research reported the effect of moral perception {[}{]} but the mechanism is disputed. For example, {[}Anderson et al 2011{]} reported that faces associated with bad social behavior capture attention more rapidly, however, independent team failed to replicate the effect {[}Stein et al{]}. Another studies by {[}{]} found that moral words are more likely to be judged as words when it was presented subliminally {[}{]}, however, this effect may caused by semantic priming instead of morality {[}Jussim et al., 2016{]}. In the current study, we used associative learning task where neutral stimuli, geometric shapes, acquired a moral meaning in the lab and then perform the perceptual matching task, which eliminated the possibility of semantics priming {[}Firestone; Jussim et al., 2016{]} or other confounding factors. Also, the effect was replicated from all five different sample, suggestion a robust prioritization effect of good character.

This positivity effect, though surprising at first, is largely consistent with previous studies. Positivity effect had been found in associative learning {[}{]}, lexical decision-making {[}{]}, and IAT {[}{]}. A common feature of these paradigms is that decision-making occurs at relative later stage of the information processing in perception, instead of early sensory processing stage. In the current paradigm, participants made a matching judgment, which was only possible after participants formed a perception of both the shape and the label and retrieve the association between them. ample evidence supported the idea that positive stimuli have advantage at the later stage of perception (Pool et al., 2016). The task used in the current study may explain why the result are different from previous studies such as E. Anderson et al. (2011) and Eiserbeck and Abdel Rahman (2020), where the early processing stage were targeted by attention blink paradigm.

{[}Self-binding as a novel explanation and consistent with broader theory about morality{]} However, the theory that the good things are similar in the mental representation {[}{]} is not sufficient to explain the current results because we only have three pair of stimuli and semantic meaning may not play a role. Also, compared to studies that using more ecological stimuli such as faces, we used geometric shape, which eliminated differences in low-level features. Because it is the actor (participant) who making the perceptual decisions and execute the action, the prioritization of good-character can be explained by either a moral categorization theory, which suggest that moral character is a categorization factor and good person are categorized as in-group, or self as binding factor view, which suggest that there is a spontaneous self-referential processing. Our results suggest that the binding between good character and self may account for the results, especially the results where either identity or moral character information were task-irrelevant. These results echo prior research on moral-self view {[}{]}, suggesting that moral-self as true self is not only at self-report level but also at perceptual level. When good character presented together with non-other information, the responses was slowed down in perceptual processing.

{[}Perspective{]}

{[}limitations{]} One would argue that the effect here may represent an effect in memory or in decision-making process instead of perpetual effect \emph{per se}. We agree that perceptual decision-making process include more processes than just perception. If we narrowly define perception as encoding of the sensory information, than perception is only one of many processes involved in a perceptual decision-making. This suggest that we need to further model the behavioral data and/or combined with other modules to deepen our understanding of the mechanism of the perceptual decision-making. For example, sequential sampling model suggested that a perceptual decision-making is a continuous evidence accumulation process and a decision is made when the amount of evidence passed a threshold. In this model, the evidence can accumulate from both sensory information but also memory{[}{]}. Recently application of sequential sample model to the perceptual matching task also suggestion that different process may contributed to the prioritization effect of self or moral self.

\hypertarget{references}{%
\section{References}\label{references}}

\begingroup
\setlength{\parindent}{-0.5in}
\setlength{\leftskip}{0.5in}

\hypertarget{refs}{}
\begin{CSLReferences}{1}{0}
\leavevmode\vadjust pre{\hypertarget{ref-anderson_visual_2011}{}}%
Anderson, E., Siegel, E. H., Bliss-Moreau, E., \& Barrett, L. F. (2011). The visual impact of gossip. \emph{Science}, \emph{332}(6036), 1446--1448. \url{https://doi.org/10.1126/science.1201574}

\leavevmode\vadjust pre{\hypertarget{ref-anderson_averaging_1965}{}}%
Anderson, N. H. (1965). Averaging versus adding as a stimulus-combination rule in impression formation. \emph{Journal of Experimental Psychology}, \emph{70}(4), 394--400. \url{https://doi.org/10.1037/h0022280}

\leavevmode\vadjust pre{\hypertarget{ref-anisfeld_when_1966}{}}%
Anisfeld, M., \& Lambert, W. E. (1966). When are pleasant words learned faster than unpleasant words? \emph{Journal of Verbal Learning and Verbal Behavior}, \emph{5}(2), 132--141. \url{https://doi.org/10.1016/S0022-5371(66)80006-3}

\leavevmode\vadjust pre{\hypertarget{ref-anselmi_positive_2011}{}}%
Anselmi, P., Vianello, M., \& Robusto, E. (2011). Positive associations primacy in the {IAT}. \emph{Experimental Psychology}. Retrieved from \url{https://econtent.hogrefe.com/doi/abs/10.1027/1618-3169/a000106}

\leavevmode\vadjust pre{\hypertarget{ref-bargh_generality_1992}{}}%
Bargh, J. A., Chaiken, S., Govender, R., \& Pratto, F. (1992). The generality of the automatic attitude activation effect. \emph{Journal of Personality and Social Psychology}, \emph{62}(6), 893--912. \url{https://doi.org/10.1037/0022-3514.62.6.893}

\leavevmode\vadjust pre{\hypertarget{ref-baumeister_bad_2001}{}}%
Baumeister, R. F., Bratslavsky, E., Finkenauer, C., \& Vohs, K. D. (2001). Bad is stronger than good. \emph{Review of General Psychology}, \emph{5}(4), 323--370. \url{https://doi.org/10.1037/1089-2680.5.4.323}

\leavevmode\vadjust pre{\hypertarget{ref-Buxfcrkner_2017}{}}%
Bürkner, P.-C. (2017). Brms: An r package for bayesian multilevel models using stan {[}Journal Article{]}. \emph{Journal of Statistical Software; Vol 1, Issue 1 (2017)}. Retrieved from \href{https://www.jstatsoft.org/v080/i01\%0Ahttp://dx.doi.org/10.18637/jss.v080.i01}{https://www.jstatsoft.org/v080/i01
http://dx.doi.org/10.18637/jss.v080.i01}

\leavevmode\vadjust pre{\hypertarget{ref-Carpenter_2017_stan}{}}%
Carpenter, B., Gelman, A., Hoffman, M. D., Lee, D., Goodrich, B., Betancourt, M., \ldots{} Riddell, A. (2017). Stan: A probabilistic programming language {[}Journal Article{]}. \emph{Journal of Statistical Software}, \emph{76}(1). \url{https://doi.org/10.18637/jss.v076.i01}

\leavevmode\vadjust pre{\hypertarget{ref-dunbar_gossip_2004}{}}%
Dunbar, R. I. M. (2004). Gossip in evolutionary perspective. \emph{Review of General Psychology}, \emph{8}(2), 100--110. \url{https://doi.org/10.1037/1089-2680.8.2.100}

\leavevmode\vadjust pre{\hypertarget{ref-eiserbeck_visual_2020}{}}%
Eiserbeck, A., \& Abdel Rahman, R. (2020). Visual consciousness of faces in the attentional blink: Knowledge-based effects of trustworthiness dominate over appearance-based impressions. \emph{Consciousness and Cognition}, \emph{83}, 102977. \url{https://doi.org/10.1016/j.concog.2020.102977}

\leavevmode\vadjust pre{\hypertarget{ref-ellemers_morality_2018}{}}%
Ellemers, N. (2018). Morality and social identity. In M. van Zomeren \& J. F. Dovidio (Eds.), \emph{The oxford handbook of the human essence} (pp. 147--158). New York, {NY}, {US}: Oxford University Press.

\leavevmode\vadjust pre{\hypertarget{ref-fiske_attention_1980}{}}%
Fiske, S. T. (1980). Attention and weight in person perception: The impact of negative and extreme behavior. \emph{Journal of Personality and Social Psychology}, \emph{38}(6), 889--906. \url{https://doi.org/10.1037/0022-3514.38.6.889}

\leavevmode\vadjust pre{\hypertarget{ref-freeman_chapter_2020}{}}%
Freeman, J. B., Stolier, R. M., \& Brooks, J. A. (2020). Chapter five - dynamic interactive theory as a domain-general account of social perception. In B. Gawronski (Ed.), \emph{Advances in experimental social psychology} (Vol. 61, pp. 237--287). Academic Press. \url{https://doi.org/10.1016/bs.aesp.2019.09.005}

\leavevmode\vadjust pre{\hypertarget{ref-Goh_2016_mini}{}}%
Goh, J. X., Hall, J. A., \& Rosenthal, R. (2016). Mini meta-analysis of your own studies: Some arguments on why and a primer on how {[}Journal Article{]}. \emph{Social and Personality Psychology Compass}, \emph{10}(10), 535--549. \url{https://doi.org/10.1111/spc3.12267}

\leavevmode\vadjust pre{\hypertarget{ref-goodwin_moral_2015}{}}%
Goodwin, G. P. (2015). Moral character in person perception. \emph{Current Directions in Psychological Science}, \emph{24}(1), 38--44. \url{https://doi.org/10.1177/0963721414550709}

\leavevmode\vadjust pre{\hypertarget{ref-goodwin_moral_2014}{}}%
Goodwin, G. P., Piazza, J., \& Rozin, P. (2014). Moral character predominates in person perception and evaluation. \emph{Journal of Personality and Social Psychology}, \emph{106}(1), 148--168. \url{https://doi.org/10.1037/a0034726}

\leavevmode\vadjust pre{\hypertarget{ref-hofmann_morality_2014}{}}%
Hofmann, W., Wisneski, D. C., Brandt, M. J., \& Skitka, L. J. (2014). Morality in everyday life. \emph{Science}, \emph{345}(6202), 1340--1343. \url{https://doi.org/10.1126/science.1251560}

\leavevmode\vadjust pre{\hypertarget{ref-Hu_2020_GoodSelf}{}}%
Hu, C.-P., Lan, Y., Macrae, C. N., \& Sui, J. (2020). Good me bad me: Does valence influence self-prioritization during perceptual decision-making? {[}Journal Article{]}. \emph{Collabra: Psychology}, \emph{6}(1), 20. \url{https://doi.org/10.1525/collabra.301}

\leavevmode\vadjust pre{\hypertarget{ref-Liu_2020_JOPD}{}}%
Liu, Q., Wang, F., Yan, W., Peng, K., Sui, J., \& Hu, C.-P. (2020). Questionnaire data from the revision of a chinese version of free will and determinism plus scale {[}Journal Article{]}. \emph{Journal of Open Psychology Data}, \emph{8}(1), 1. \url{https://doi.org/10.5334/jopd.49/}

\leavevmode\vadjust pre{\hypertarget{ref-pool_attentional_2016}{}}%
Pool, E., Brosch, T., Delplanque, S., \& Sander, D. (2016). Attentional bias for positive emotional stimuli: A meta-analytic investigation. \emph{Psychological Bulletin}, \emph{142}(1), 79--106. \url{https://doi.org/10.1037/bul0000026}

\leavevmode\vadjust pre{\hypertarget{ref-pratto_automatic_1991}{}}%
Pratto, F., \& John, O. P. (1991). Automatic vigilance: The attention-grabbing power of negative social information. \emph{Journal of Personality and Social Psychology}, \emph{61}(3), 380--391. \url{https://doi.org/10.1037//0022-3514.61.3.380}

\leavevmode\vadjust pre{\hypertarget{ref-Rouder_2005_BHM_SDT}{}}%
Rouder, J. N., \& Lu, J. (2005). An introduction to bayesian hierarchical models with an application in the theory of signal detection {[}Journal Article{]}. \emph{Psychonomic Bulletin \& Review}, \emph{12}(4), 573--604. \url{https://doi.org/10.3758/bf03196750}

\leavevmode\vadjust pre{\hypertarget{ref-Rousselet_2019}{}}%
Rousselet, G. A., \& Wilcox, R. R. (2019). Reaction times and other skewed distributions: Problems with the mean and the median {[}Preprint{]}. \emph{Meta-Psychology}. \url{https://doi.org/10.1101/383935}

\leavevmode\vadjust pre{\hypertarget{ref-shore_social_2013}{}}%
Shore, D. M., \& Heerey, E. A. (2013). Do social utility judgments influence attentional processing? \emph{Cognition}, \emph{129}(1), 114--122. \url{https://doi.org/10.1016/j.cognition.2013.06.011}

\leavevmode\vadjust pre{\hypertarget{ref-Simmons_2013_life}{}}%
Simmons, J. P., Nelson, L. D., \& Simonsohn, U. (2013). \emph{Life after p-hacking} {[}Conference Proceedings{]}. \url{https://doi.org/10.2139/ssrn.2205186}

\leavevmode\vadjust pre{\hypertarget{ref-skowronski_social_1987}{}}%
Skowronski, J. J., \& Carlston, D. E. (1987). Social judgment and social memory: The role of cue diagnosticity in negativity, positivity, and extremity biases. \emph{Journal of Personality and Social Psychology}, \emph{52}(4), 689--699. \url{https://doi.org/10.1037/0022-3514.52.4.689}

\leavevmode\vadjust pre{\hypertarget{ref-stein_no_2017}{}}%
Stein, T., Grubb, C., Bertrand, M., Suh, S. M., \& Verosky, S. C. (2017). No impact of affective person knowledge on visual awareness: Evidence from binocular rivalry and continuous flash suppression. \emph{Emotion}, \emph{17}(8), 1199--1207. \url{https://doi.org/10.1037/emo0000305}

\leavevmode\vadjust pre{\hypertarget{ref-strohminger_true_2017}{}}%
Strohminger, N., Knobe, J., \& Newman, G. (2017). The true self: A psychological concept distinct from the self: \emph{Perspectives on Psychological Science}. \url{https://doi.org/10.1177/1745691616689495}

\leavevmode\vadjust pre{\hypertarget{ref-Sui_2012_JEPHPP}{}}%
Sui, J., He, X., \& Humphreys, G. W. (2012). Perceptual effects of social salience: Evidence from self-prioritization effects on perceptual matching {[}Journal Article{]}. \emph{Journal of Experimental Psychology: Human Perception and Performance}, \emph{38}(5), 1105--1117. \url{https://doi.org/10.1037/a0029792}

\leavevmode\vadjust pre{\hypertarget{ref-uhlmann_person-centered_2015}{}}%
Uhlmann, E. L., Pizarro, D. A., \& Diermeier, D. (2015). A person-centered approach to moral judgment: \emph{Perspectives on Psychological Science}. \url{https://doi.org/10.1177/1745691614556679}

\leavevmode\vadjust pre{\hypertarget{ref-unkelbach_chapter_2020}{}}%
Unkelbach, C., Alves, H., \& Koch, A. (2020). Chapter three - negativity bias, positivity bias, and valence asymmetries: Explaining the differential processing of positive and negative information. In B. Gawronski (Ed.), \emph{Advances in experimental social psychology} (Vol. 62, pp. 115--187). Academic Press. \url{https://doi.org/10.1016/bs.aesp.2020.04.005}

\leavevmode\vadjust pre{\hypertarget{ref-unkelbach_good_2010}{}}%
Unkelbach, C., Hippel, W. von, Forgas, J. P., Robinson, M. D., Shakarchi, R. J., \& Hawkins, C. (2010). Good things come easy: Subjective exposure frequency and the faster processing of positive information. \emph{Social Cognition}, \emph{28}(4), 538--555. \url{https://doi.org/10.1521/soco.2010.28.4.538}

\leavevmode\vadjust pre{\hypertarget{ref-xiao_perceiving_2016}{}}%
Xiao, Y. J., Coppin, G., \& Bavel, J. J. V. (2016). Perceiving the world through group-colored glasses: A perceptual model of intergroup relations. \emph{Psychological Inquiry}, \emph{27}(4), 255--274. \url{https://doi.org/10.1080/1047840X.2016.1199221}

\end{CSLReferences}

\endgroup


\end{document}
