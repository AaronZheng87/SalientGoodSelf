% Options for packages loaded elsewhere
\PassOptionsToPackage{unicode}{hyperref}
\PassOptionsToPackage{hyphens}{url}
%
\documentclass[
  man]{apa6}
\usepackage{amsmath,amssymb}
\usepackage{lmodern}
\usepackage{iftex}
\ifPDFTeX
  \usepackage[T1]{fontenc}
  \usepackage[utf8]{inputenc}
  \usepackage{textcomp} % provide euro and other symbols
\else % if luatex or xetex
  \usepackage{unicode-math}
  \defaultfontfeatures{Scale=MatchLowercase}
  \defaultfontfeatures[\rmfamily]{Ligatures=TeX,Scale=1}
\fi
% Use upquote if available, for straight quotes in verbatim environments
\IfFileExists{upquote.sty}{\usepackage{upquote}}{}
\IfFileExists{microtype.sty}{% use microtype if available
  \usepackage[]{microtype}
  \UseMicrotypeSet[protrusion]{basicmath} % disable protrusion for tt fonts
}{}
\makeatletter
\@ifundefined{KOMAClassName}{% if non-KOMA class
  \IfFileExists{parskip.sty}{%
    \usepackage{parskip}
  }{% else
    \setlength{\parindent}{0pt}
    \setlength{\parskip}{6pt plus 2pt minus 1pt}}
}{% if KOMA class
  \KOMAoptions{parskip=half}}
\makeatother
\usepackage{xcolor}
\usepackage{graphicx}
\makeatletter
\def\maxwidth{\ifdim\Gin@nat@width>\linewidth\linewidth\else\Gin@nat@width\fi}
\def\maxheight{\ifdim\Gin@nat@height>\textheight\textheight\else\Gin@nat@height\fi}
\makeatother
% Scale images if necessary, so that they will not overflow the page
% margins by default, and it is still possible to overwrite the defaults
% using explicit options in \includegraphics[width, height, ...]{}
\setkeys{Gin}{width=\maxwidth,height=\maxheight,keepaspectratio}
% Set default figure placement to htbp
\makeatletter
\def\fps@figure{htbp}
\makeatother
\setlength{\emergencystretch}{3em} % prevent overfull lines
\providecommand{\tightlist}{%
  \setlength{\itemsep}{0pt}\setlength{\parskip}{0pt}}
\setcounter{secnumdepth}{-\maxdimen} % remove section numbering
% Make \paragraph and \subparagraph free-standing
\ifx\paragraph\undefined\else
  \let\oldparagraph\paragraph
  \renewcommand{\paragraph}[1]{\oldparagraph{#1}\mbox{}}
\fi
\ifx\subparagraph\undefined\else
  \let\oldsubparagraph\subparagraph
  \renewcommand{\subparagraph}[1]{\oldsubparagraph{#1}\mbox{}}
\fi
\newlength{\cslhangindent}
\setlength{\cslhangindent}{1.5em}
\newlength{\csllabelwidth}
\setlength{\csllabelwidth}{3em}
\newlength{\cslentryspacingunit} % times entry-spacing
\setlength{\cslentryspacingunit}{\parskip}
\newenvironment{CSLReferences}[2] % #1 hanging-ident, #2 entry spacing
 {% don't indent paragraphs
  \setlength{\parindent}{0pt}
  % turn on hanging indent if param 1 is 1
  \ifodd #1
  \let\oldpar\par
  \def\par{\hangindent=\cslhangindent\oldpar}
  \fi
  % set entry spacing
  \setlength{\parskip}{#2\cslentryspacingunit}
 }%
 {}
\usepackage{calc}
\newcommand{\CSLBlock}[1]{#1\hfill\break}
\newcommand{\CSLLeftMargin}[1]{\parbox[t]{\csllabelwidth}{#1}}
\newcommand{\CSLRightInline}[1]{\parbox[t]{\linewidth - \csllabelwidth}{#1}\break}
\newcommand{\CSLIndent}[1]{\hspace{\cslhangindent}#1}
\ifLuaTeX
\usepackage[bidi=basic]{babel}
\else
\usepackage[bidi=default]{babel}
\fi
\babelprovide[main,import]{english}
% get rid of language-specific shorthands (see #6817):
\let\LanguageShortHands\languageshorthands
\def\languageshorthands#1{}
% Manuscript styling
\usepackage{upgreek}
\captionsetup{font=singlespacing,justification=justified}

% Table formatting
\usepackage{longtable}
\usepackage{lscape}
% \usepackage[counterclockwise]{rotating}   % Landscape page setup for large tables
\usepackage{multirow}		% Table styling
\usepackage{tabularx}		% Control Column width
\usepackage[flushleft]{threeparttable}	% Allows for three part tables with a specified notes section
\usepackage{threeparttablex}            % Lets threeparttable work with longtable

% Create new environments so endfloat can handle them
% \newenvironment{ltable}
%   {\begin{landscape}\centering\begin{threeparttable}}
%   {\end{threeparttable}\end{landscape}}
\newenvironment{lltable}{\begin{landscape}\centering\begin{ThreePartTable}}{\end{ThreePartTable}\end{landscape}}

% Enables adjusting longtable caption width to table width
% Solution found at http://golatex.de/longtable-mit-caption-so-breit-wie-die-tabelle-t15767.html
\makeatletter
\newcommand\LastLTentrywidth{1em}
\newlength\longtablewidth
\setlength{\longtablewidth}{1in}
\newcommand{\getlongtablewidth}{\begingroup \ifcsname LT@\roman{LT@tables}\endcsname \global\longtablewidth=0pt \renewcommand{\LT@entry}[2]{\global\advance\longtablewidth by ##2\relax\gdef\LastLTentrywidth{##2}}\@nameuse{LT@\roman{LT@tables}} \fi \endgroup}

% \setlength{\parindent}{0.5in}
% \setlength{\parskip}{0pt plus 0pt minus 0pt}

% Overwrite redefinition of paragraph and subparagraph by the default LaTeX template
% See https://github.com/crsh/papaja/issues/292
\makeatletter
\renewcommand{\paragraph}{\@startsection{paragraph}{4}{\parindent}%
  {0\baselineskip \@plus 0.2ex \@minus 0.2ex}%
  {-1em}%
  {\normalfont\normalsize\bfseries\itshape\typesectitle}}

\renewcommand{\subparagraph}[1]{\@startsection{subparagraph}{5}{1em}%
  {0\baselineskip \@plus 0.2ex \@minus 0.2ex}%
  {-\z@\relax}%
  {\normalfont\normalsize\itshape\hspace{\parindent}{#1}\textit{\addperi}}{\relax}}
\makeatother

% \usepackage{etoolbox}
\makeatletter
\patchcmd{\HyOrg@maketitle}
  {\section{\normalfont\normalsize\abstractname}}
  {\section*{\normalfont\normalsize\abstractname}}
  {}{\typeout{Failed to patch abstract.}}
\patchcmd{\HyOrg@maketitle}
  {\section{\protect\normalfont{\@title}}}
  {\section*{\protect\normalfont{\@title}}}
  {}{\typeout{Failed to patch title.}}
\makeatother

\usepackage{xpatch}
\makeatletter
\xapptocmd\appendix
  {\xapptocmd\section
    {\addcontentsline{toc}{section}{\appendixname\ifoneappendix\else~\theappendix\fi\\: #1}}
    {}{\InnerPatchFailed}%
  }
{}{\PatchFailed}
\keywords{Perceptual matching, Self positivity bias, moral character\newline\indent Word count: X}
\DeclareDelayedFloatFlavor{ThreePartTable}{table}
\DeclareDelayedFloatFlavor{lltable}{table}
\DeclareDelayedFloatFlavor*{longtable}{table}
\makeatletter
\renewcommand{\efloat@iwrite}[1]{\immediate\expandafter\protected@write\csname efloat@post#1\endcsname{}}
\makeatother
\usepackage{lineno}

\linenumbers
\usepackage{csquotes}
\usepackage{rotating}
\DeclareDelayedFloatFlavor{sidewaysfigure}{figure}
\ifLuaTeX
  \usepackage{selnolig}  % disable illegal ligatures
\fi
\IfFileExists{bookmark.sty}{\usepackage{bookmark}}{\usepackage{hyperref}}
\IfFileExists{xurl.sty}{\usepackage{xurl}}{} % add URL line breaks if available
\urlstyle{same} % disable monospaced font for URLs
\hypersetup{
  pdftitle={The good person is me: Spontaneous self-referential process prioritizes moral character in perceptual matching},
  pdfauthor={Hu Chuan-Peng1, 2, Kaiping Peng2, \& Jie Sui3},
  pdflang={en-EN},
  pdfkeywords={Perceptual matching, Self positivity bias, moral character},
  hidelinks,
  pdfcreator={LaTeX via pandoc}}

\title{The good person is me: Spontaneous self-referential process prioritizes moral character in perceptual matching}
\author{Hu Chuan-Peng\textsuperscript{1, 2}, Kaiping Peng\textsuperscript{2}, \& Jie Sui\textsuperscript{3}}
\date{}


\shorttitle{The good person is self-related}

\authornote{

Hu Chuan-Peng, School of Psychology, Nanjing Normal University, 210024 Nanjing, China.
Kaiping Peng, Department of Psychology, Tsinghua University, 100084 Beijing, China.
Jie Sui, School of Psychology, University of Aberdeen, Aberdeen, Scotland.
Authors contriubtion: HCP, JS, \& KP design the study, HCP collected the data, HCP analyzed the data and drafted the manuscript. All authors read and agreed upon the current version of the manuscripts.

Correspondence concerning this article should be addressed to Hu Chuan-Peng, School of Psychology, Nanjing Normal University, Ninghai Road 122, Gulou District, 210024 Nanjing, China. E-mail: \href{mailto:hcp4715@hotmail.com}{\nolinkurl{hcp4715@hotmail.com}}

}

\affiliation{\vspace{0.5cm}\textsuperscript{1} Nanjing Normal University, 210024 Nanjing, China\\\textsuperscript{2} Tsinghua University, 100084 Beijing, China\\\textsuperscript{3} University of Aberdeen, Aberdeen, Scotland}

\abstract{%
Moral character is central to social evaluation and moral judgment. People can infer moral character from a human face in less than one second. However, whether moral character-related information is prioritized, as compared to neutral characters, in perceptual decision-making was debated. Here we investigated the effect of moral character on perceptual decision-making through an associative learning task. Participants first learned associations between different geometric shapes and moral characters and then performed a simple perceptual matching task. Across five experiments (N = 192), we found a robust prioritization effect of good character-related shapes, i.e., participants responded faster and more accurately to shapes that were associated with good characters than shapes associated with neutral or bad characters. We then examine whether the prioritization of good character was due to valence alone or an interaction between valence and self-referential processing. Results of three experiments (N = 108) demonstrated that the prioritization effect of good character was robust when referred to the self but weak or non-exist when referred to others. Additional two experiments (N = 104) further revealed that the mutual facilitation between good character and the self occurred even when one of them was task-irrelevant. Together, these results not only provide evidence for a robust prioritization effect of good character but also the crucial role of spontaneous self-referential process in the prioritization of good character.
}



\begin{document}
\maketitle

\hypertarget{introduction}{%
\section{Introduction}\label{introduction}}

Morality is one of the basic dimensions in social evaluation (Dunbar, 2004; Ellemers, 2018; Goodwin, Piazza, \& Rozin, 2014). People can form an impression of others' moral character in less than one second (Willis \& Todorov, 2006). It is intriguing whether moral character-related information is prioritized in perception. This question evoked much heat a few years ago but remains unsolved. Previous studies have found that stimuli that are also important to humans, e.g., threatening stimuli (e.g., Ohman, Lundqvist, \& Esteves, 2001), rewards (B. A. Anderson, Laurent, \& Yantis, 2011), or self-related stimuli (Sui \& Rotshtein, 2019), are prioritized when attentional resources are limited. Given the importance of morality in social life, moral information should be more salient than morally neutral information and thus be prioritized. Indeed, a few studies reported that bad characters are prioritized in visual processing (E. Anderson, Siegel, Bliss-Moreau, \& Barrett, 2011; Eiserbeck \& Abdel Rahman, 2020), suggesting that bad people are detected faster than neutral or good people. However, results that morally bad information is prioritized are debated. First, the opposite effect, positive bias, was also reported. For example, Shore and Heerey (2013) found that faces with positive interaction in a trust game were prioritized in the pre-attentive process. Second, the robustness of previous results is questioned (eg., Stein, Grubb, Bertrand, Suh, \& Verosky, 2017). Third, the prioritization effect of morality might be confounded with other factors (Firestone \& Scholl, 2015, 2016b; Jussim, Crawford, Anglin, Stevens, \& Duarte, 2016). In short, while the importance of morality is widely recognized, whether moral information is prioritized in perceptual decision-making is still an open question.

Here, we conducted a series of experiments to examine the prioritization effect of moral character. We investigated how immediately acquired moral character information modulates the processing of neutral geometric shapes in a perceptual matching task. The associative learning task is based on the well-established fact that humans can quickly learn the associations between symbols and change subsequent behaviors accordingly. This associative learning task is widely used in aversive learning and value-based learning (Atlas et al., 2022; Deltomme, Mertens, Tibboel, \& Braem, 2018). Unlike previous studies relies on faces or words as materials, stimuli used in the social associative task are geometric shapes, which acquire moral meaning before the perceptual matching task. Moreover, associations between shapes and different labels of moral characters are counter-balanced between participants, thus eliminating confounding effects by stimuli. Also, because we repeatedly presented a few pairs of shapes and labels to participants during the task, our results can not be explained by semantic priming (Unkelbach, Alves, \& Koch, 2020), which is the center of the debate on previous results (Firestone \& Scholl, 2015, 2016a; Gantman \& Bavel, 2015, 2016; Jussim et al., 2016). We found a robust effect that shapes associated with good character are prioritized in the perceptual matching task. In a series of control experiments, we further confirmed that it is the moral content that drove the prioritization effect, instead of other factors such as familiarity.

If moral character information is prioritized, the next question is how? Previous studies explain the effect based on valence. For example, the negative bias toward moral information is explained by aligning moral information with affective stimuli and threat detection was supposed to be the potential mechanism (B. A. Anderson et al., 2011). The positive bias toward moral information, on the other hand, is explained by value-based attention (Shore \& Heerey, 2013). However, these explanations often ignore the fact the value is subjective \emph{per se} (Juechems \& Summerfield, 2019). Merely associating with the self can prioritize the stimuli in perception, attention, working memory, and long-term memory (Sui \& Humphreys, 2015; Sui \& Rotshtein, 2019). Here, we introduced self-relevance in the social associative learning task. In the experiment, we explicitly instructed participants on which moral character is self-referencing and which is not. In this way, we tested whether the prioritization of moral character is by valence \emph{per se} or by the self-referential of moral valence. In the subsequent experiments, we further tested the interaction between valence and self-referential processing more implicitly. The results revealed a mutual facilitation effect of good character and the self, suggesting spontaneous moral self-referential as a novel mechanism underlying the prioritization of good character in perceptual decision-making.

\hypertarget{disclosures}{%
\section{Disclosures}\label{disclosures}}

We reported all the measurements, analyses, and results in all the experiments in the current study. Participants whose overall accuracy was lower than 60\% were excluded from analyses. Also, accurate responses with less than 200ms reaction times were excluded from the analysis. These excluded data can be found in the shared raw data files.

All the experiments reported were not pre-registered. Most experiments (1a \textasciitilde{} 4b, except experiment 3b) reported in the current study were first finished between 2013 to 2016 at Tsinghua University, Beijing, China. Participants in these experiments were recruited from the local community. To increase the sample size of experiments to 50 or more (Simmons, Nelson, \& Simonsohn, 2013), we recruited additional participants from Wenzhou University, Wenzhou, China, in 2017 for experiments 1a, 1b, 4a, and 4b. Experiment 3b was finished at Wenzhou University in 2017 (See Table 1 for an overview of these experiments).

All participants received informed consent and were compensated for their time. These experiments were approved by the ethics board in the Department of Psychology, Tsinghua University.

\hypertarget{general-methods}{%
\section{General methods}\label{general-methods}}

\hypertarget{design-and-procedure}{%
\subsection{Design and Procedure}\label{design-and-procedure}}

This series of experiments used the social associative learning paradigm (or self-tagging paradigm, see Sui, He, and Humphreys (2012)), in which participants first learned the associations between geometric shapes and labels of different moral characters (e.g., in the first three studies, the triangle, square, and circle and Chinese words for ``good person'', ``neutral person'', and ``bad person'', respectively). The associations of shapes and labels were counterbalanced across participants. The paradigm consists of a brief learning stage and a test stage. During the learning stage, participants were instructed about the association between shapes and labels. Participants started the test stage with a practice phase to familiarize themselves with the task, in which they viewed one of the shapes above the fixation while one of the labels below the fixation and judged whether the shape and the label matched the association they learned. If the overall accuracy reached 60\% or higher at the end of the practicing session, participants proceeded to the experimental task of the test stage. Otherwise, they finished another practices sessions until the overall accuracy was equal to or greater than 60\%. The experimental task shared the same trial structure as in the practice.

Experiments 1a, 1b, 1c, 2, 5, and 6a were designed to explore and confirm the effect of moral character on perceptual matching. All these experiments shared a 2 (matching: match vs.~nonmatch) by 3 (moral character: good vs.~neutral vs.~bad person) within-subject design. Experiment 1a was the first one of the whole series of studies, which aimed to examine the prioritization of moral character and found that shapes associated with good character were prioritized. Experiments 1b, 1c, and 2 were to confirm that it is the moral character that caused the effect. More specifically, experiment 1b used different Chinese words as labels to test whether the effect was contaminated by familiarity. Experiment 1c manipulated the moral character indirectly: participants first learned to associate different moral behaviors with different Chinese names, after remembering the association, they then associate the names with different shapes and finished the perceptual matching task. Experiment 2 further tested whether the way we presented the stimuli influence the prioritization of moral character, by sequentially presenting labels and shapes instead of simultaneous presentation. Note that a few participants in experiment 2 also participated in experiment 1a because we originally planned a cross-task comparison. Experiment 5 was designed to compare the prioritization of good character with other important social values (aesthetics and emotion). All social values had three levels, positive, neutral, and negative, and were associated with different shapes. Participants finished the associative learning task for different social values in different blocks, and the order of the social values was counterbalanced. Only the data from moral character blocks, which shared the design of experiment 1a, were reported here. Experiment 6a, which shared the same design as experiment 2, was an EEG experiment aimed at exploring the neural mechanism of the prioritization of good character. Only behavioral results of experiment 6a were reported here.

Experiments 3a, 3b, and 6b were designed to test whether the prioritization of good character can be explained by the valence effect alone or by an interaction between the valence effect and self-referential processing. To do so, we included self-reference as another within-subject variable. For example, experiment 3a extended experiment 1a into a 2 (matching: match vs.~nonmatch) by 2 (reference: self vs.~other) by 3 (moral character: good vs.~neutral vs.~bad) within-subject design. Thus, in experiment 3a, there were six conditions (good-self, neutral-self, bad-self, good-other, neutral-other, and bad-other) and six shapes (triangle, square, circle, diamond, pentagon, and trapezoids). Experiment 6b was an EEG experiment based on experiment 3a but presented the label and shape sequentially. Because of the relatively high working memory load (six label-shape pairs), participants finished experiment 6b in two days. On the first day, participants completed the perceptual matching task as a practice, and on the second day, they finished the task again while the EEG signals were recorded. We only focus on the first day's data here. Experiment 3b was designed to test whether the effect found in experiments 3a and 6b is robust if we separately present the self-referential trials and other-referential trials. That is, participants finished two different types of blocks: in the self-referential blocks, they only made matching judgments to shape-label pairs that related to the self (i.e., shapes and labels of good-self, neutral-self, and bad-self), in the other-referential blocks, they only responded to shape-label pairs that related to the other (i.e., shapes and labels of good-other, neutral-other, and bad-other).

Experiments 4a and 4b were designed to further test the interaction between valence and self-referential process in prioritization of good character. In experiment 4a, participants were instructed to learn the association between two shapes (circle and square) with two labels (self vs.~other) in the learning stage. In the test stage, they were instructed only respond to the shape and label during the test stage. To test the effect of moral character, we presented the labels of moral character in the shapes and instructed participants to ignore the words in shapes when making matching judgments. In the experiment 4b, we reversed the role of self and moral character in the task: Participants learned associations between three labels (good-person, neutral-person, and bad-person) and three shapes (circle, square, and triangle) and made matching judgments about the shape and label of moral character, while words related to identity, ``self'' or ``other'', were presented within the shapes. As in 4a, participants were told to ignore the words inside the shape during the perceptual matching task.

\hypertarget{stimuli-and-materials}{%
\subsection{Stimuli and Materials}\label{stimuli-and-materials}}

We used E-prime 2.0 for presenting stimuli and collecting behavioral responses. Data were collected from two universities located in two different cities in China. Participants recruited from Tsinghua University, Beijing, finished the experiment individually in a dim-lighted chamber. Stimuli were presented on 22-inch CRT monitors and participants rested their chins on a brace to fix the distance between their eyes and the screen around 60 cm. The visual angle of geometric shapes was about \(3.7^\circ × 3.7^\circ\), the fixation cross is of \(0.8^\circ × 0.8^\circ\) visual angle at the center of the screen. The words were of \(3.6^\circ\) × \(1.6^\circ\) visual angle. The distance between the center of shapes or images of labels and the fixation cross was of \(3.5^\circ\) visual angle. Participants from Wenzhou University, Wenzhou, finished the experiment in a group consisting of 3 \textasciitilde{} 12 participants in a dim-lighted testing room. They were instructed to finish the whole experiment independently. Also, they were told to start the experiment at the same time so that the distraction between participants was minimized. The stimuli were presented on 19-inch CRT monitors with the same set of parameters in E-prime 2.0 as in Tsinghua University, however, the visual angles could not be controlled because participants' chins were not fixed.

In most of these experiments, participants were also asked to fill out questionnaires after finishing the behavioral tasks. All the questionnaire data were open (see, dataset 4 in Liu et al., 2020). See Table 1 for a summary of information about all the experiments.

\begin{table}[tbp]

\begin{center}
\begin{threeparttable}

\caption{\label{tab:Table_1_exp_info}Information about all experiments.}

\begin{tabular}{llllllll}
\toprule
ExpID & \multicolumn{1}{c}{Time} & \multicolumn{1}{c}{Location} & \multicolumn{1}{c}{N} & \multicolumn{1}{c}{n.of.trials} & \multicolumn{1}{c}{Self.ref} & \multicolumn{1}{c}{Stim.for.Morality} & \multicolumn{1}{c}{Presenting.order}\\
\midrule
Exp\_1a\_1 & 2014-04 & Beijing & 38 (35) & 60 & NA & words & Simultaneously\\
Exp\_1a\_2 & 2017-04 & Wenzhou & 18 (16) & 120 & NA & words & Simultaneously\\
Exp\_1b\_1 & 2014-10 & Beijing & 39 (27) & 60 & NA & words & Simultaneously\\
Exp\_1b\_2 & 2017-04 & Wenzhou & 33 (25) & 120 & NA & words & Simultaneously\\
Exp\_1c & 2014-10 & Beijing & 23 (23) & 60 & NA & descriptions & Simultaneously\\
Exp\_2 & 2014-05 & Beijing & 35 (34) & 60 & NA & words & Sequentially\\
Exp\_3a & 2014-11 & Beijing & 38 (35) & 60 & explicit & words & Simultaneously\\
Exp\_3b & 2017-04 & Wenzhou & 61 (56) & 60 & explicit & words & Simultaneously\\
Exp\_4a\_1 & 2015-06 & Beijing & 32 (29) & 30 & implicit & words & Simultaneously\\
Exp\_4a\_2 & 2017-04 & Wenzhou & 32 (30) & 60 & implicit & words & Simultaneously\\
Exp\_4b\_1 & 2015-10 & Beijing & 34 (32) & 60 & implicit & words & Simultaneously\\
Exp\_4b\_2 & 2017-04 & Wenzhou & 19 (13) & 60 & implicit & words & Simultaneously\\
Exp\_5 & 2016-01 & Beijing & 43 (38) & 60 & NA & words & Simultaneously\\
Exp\_6a & 2014-12 & Beijing & 24 (24) & 180 & NA & words & Sequentially\\
Exp\_6b & 2016-01 & Beijing & 23 (22) & 90 & explicit & words & Sequentially\\
\bottomrule
\addlinespace
\end{tabular}

\begin{tablenotes}[para]
\normalsize{\textit{Note.} Stim of Morality = How moral character was manipulated; Presenting order = how shapes \& labels were presented. The data from experiments 7a \& 7b, which were reported in Hu et al (2020), are not analyzed here.}
\end{tablenotes}

\end{threeparttable}
\end{center}

\end{table}

\hypertarget{data-analysis}{%
\subsection{Data analysis}\label{data-analysis}}

We used the \texttt{tidyverse} of r (see script \texttt{Load\_save\_data.r}) to preprocess the data. The data from all experiments were then analyzed using Bayesian hierarchical models.

We used the Bayesian hierarchical model (BHM, or Bayesian generalized linear mixed models, Bayesian multilevel models) to model the reaction time and accuracy data because BHM provided three advantages over the classic NHST approach (repeated measure ANOVA or \emph{t}-tests). First, BHM estimates the posterior distributions of parameters for statistical inference, therefore providing uncertainty in estimation (Rouder \& Lu, 2005). Second, BHM, where generalized linear mixed models could be easily implemented, can use distributions that fit the distribution of real data instead of using the normal distribution for all data. Using appropriate distributions for the data will avoid misleading results and provide a better fitting of the data. For example, Reaction times are not normally distributed but are right skewed, and the linear assumption in ANOVAs is not satisfied (Rousselet \& Wilcox, 2020). Third, BHM provides a unified framework to analyze data from different levels and different sources, avoiding information loss when we need to combine data from different experiments.

We used the \texttt{r} package \texttt{BRMs} (Bürkner, 2017), which used Stan (Carpenter et al., 2017) as the back-end, for the BHM analyses. We estimated the overall effect across experiments that shared the same experimental design using one model, instead of a two-step approach that was adopted in mini-meta-analysis (e.g., Goh, Hall, \& Rosenthal, 2016). More specifically, a three-level model was used to estimate the overall effect of prioritization of good character, which included data from five experiments: 1a, 1b, 1c, 2, 5, and 6a. Similarly, a three-level HBM model is used for experiments 3a, 3b, and 6b. Results of individual experiments can be found in the supplementary results. For experiments 4a and 4b, which tested the implicit interaction between the self and good character, we used HBM for each experiment separately.

For questionnaire data, we only reported the subjective distance between different persons or moral characters in the supplementary results and did not analyze other questionnaire data, which are described in (Liu et al., 2020).

\hypertarget{response-data}{%
\subsubsection{Response data}\label{response-data}}

We followed previous studies (Hu, Lan, Macrae, \& Sui, 2020; Sui et al., 2012) and used the signal detection theory approach to analyze the response data. More specifically, the match trials are treated as signals and non-match trials are noise. The sensitivity and criterion of signal detection theory are modeled through BHM (Rouder \& Lu, 2005).

We used the Bernoulli distribution for the signal detection theory. The probability that the \(j\)th subject responded ``match'' (\(y_{ij} = 1\)) at the \(i\)th trial \(p_{ij}\) is distributed as a Bernoulli distribution with parameter \(p_{ij}\):

\[ y_{ij} \sim Bernoulli(p_{ij})\]
The reparameterized value of \(p_{ij}\) is a linear regression of the independent variables:
\[ \Phi(p_{ij}) = 0 + \beta_{0j}Valence_{ij} + \beta_{1j}IsMatch_{ij} * Valence_{ij}\]
where the probits (z-scores; \(\Phi\), ``Phi'') of \(p\)s is used for the regression.

The subjective-specific intercepts (\(\beta_{0} = -zFAR\)) and slopes (\(\beta_{1} = d'\)) are described by multivariate normal with means and a covariance matrix for the parameters.
\[ \begin{bmatrix}\beta_{0j}\\
\beta_{1j}\\
\end{bmatrix} \sim N(\begin{bmatrix}\theta_{0}\\
\theta_{1}\\
\end{bmatrix}, \sum) \]

We used the following formula for experiments 1a, 1b, 1c, 2, 5, and 6a, which have a 2 (matching: match vs.~non-match) by 3 (moral character: good vs.~neutral vs.~bad) within-subject design:

\texttt{saymatch\ \textasciitilde{}\ 0\ +\ Valence\ +\ Valence:ismatch\ +\ (0\ +\ Valence\ +\ Valence:ismatch\ \textbar{}\ Subject)\ +\ (0\ +\ Valence\ +\ Valence:ismatch\ \textbar{}\ ExpID\_new:Subject)\ ,\ family\ =\ bernoulli(link="probit")}

in which the \texttt{saymatch} is the response data whether participants pressed the key corresponding to ``match'', \texttt{ismatch} is the independent variable of matching, \texttt{Valence} is the independent variable of moral character, \texttt{Subject} is the index of participants, and \texttt{Exp\_ID\_new} is the index of different experiments. Not that we distinguished data collected from two universities.

For experiments 3a, 3b, and 6b, an additional variable, i.e., reference (self vs.~other), was included in the formula:

\texttt{saymatch\ \textasciitilde{}\ 0\ +\ ID:Valence\ +\ ID:Valence:ismatch\ +\ (0\ +\ ID:Valence\ +\ ID:Valence:ismatch\ \textbar{}\ Subject)\ +\ (0\ +\ ID:Valence\ +\ ID:Valence:ismatch\ \textbar{}\ ExpID\_new:Subject),\ family\ =\ bernoulli(link="probit")}
in which the \texttt{ID} is the independent variable ``reference'', which means whether the stimulus was self-referential or other-referential.

\hypertarget{reaction-times}{%
\subsubsection{Reaction times}\label{reaction-times}}

We used log-normal distribution (\url{https://lindeloev.github.io/shiny-rt/\#34_(shifted)_log-normal}) to model the RT data. This means that we need to estimate the posterior of two parameters: \(\mu\), and \(\sigma\). \(\mu\) is the mean of the \texttt{logNormal} distribution, and \(\sigma\) is the disperse of the distribution.

The reaction time of the \(j\)th subject on \(i\)th trial, \(y_{ij}\), is log-normal distributed:
\[ log(y_{ij}) \sim N(\mu_{j}, \sigma_{j})\]

The parameter \(\mu_{j}\) is a linear regression of the independent variables:
\[\mu_{j} = \beta_{0j} + \beta_{1j}*IsMatch_{ij} * Valence_{ij}\]

and the parameter \(\sigma_{j}\) does not vary with independent variables:
\[\sigma_{j} \sim HalfNormal()\]

The subjective-specific intercepts (\(\beta_{0j}\)) and slopes (\(\beta_{1j}\)) are described by multivariate normal with means and a covariance matrix for the parameters.
\[ \begin{bmatrix}\beta_{0j}\\
\beta_{1j}\\
\end{bmatrix} \sim N(\begin{bmatrix}\theta_{0}\\
\theta_{1}\\
\end{bmatrix}, \sum) \]

The formula used for experiments 1a, 1b, 1c, 2, 5, and 6a, which have a 2 (matching: match vs.~non-match) by 3 (moral character: good vs.~neutral vs.~bad) within-subject design, is as follows:

\texttt{RT\_sec\ \textasciitilde{}\ 1\ +\ Valence*ismatch\ +\ (Valence*ismatch\ \textbar{}\ Subject)\ +\ (Valence*ismatch\ \textbar{}\ ExpID\_new:Subject),\ family\ =\ lognormal()}
in which \texttt{RT\_sec} is the reaction times data with the second as a unit. The other variables in this formula have the same meaning as the response data.

For experiments 3a, 3b, and 6b, which have a 2 by 2 by 3 within-subject design, the formula is as follows:
\texttt{RT\_sec\ \textasciitilde{}\ 1\ +\ ID*Valence\ +\ (ID*Valence\ \textbar{}\ Subject)\ +\ (ID*Valence\ \textbar{}\ ExpID\_new:Subject),\ family\ =\ lognormal()}

Note that for experiments 3a, 3b, and 6b, the three-level model for reaction times only included the matched trials to avoid divergence when estimating the posterior of the parameters.

\hypertarget{testing-hypotheses}{%
\subsubsection{Testing hypotheses}\label{testing-hypotheses}}

To test hypotheses, we used the Sequential Effect eXistence and sIgnificance Testing (SEXIT) framework suggested by Makowski, Ben-Shachar, Chen, and Lüdecke (2019). In this approach, we directly use the posterior distributions of model parameters or other effects that can be derived from posterior distributions. The SEXIT approach reports centrality, uncertainty, existence, significance, and size of the input posterior, which is intuitive for making statistical inferences. We used \texttt{bayestestR} for implementing this approach (Makowski, Ben-Shachar, \& Lüdecke, 2019). Following the SEXIT framework, we reported the median of the posterior distribution and its 95\% HDI (Highest Density Interval), along the probability of direction (pd), the probability of significance. The thresholds beyond which the effect is considered as significant (i.e., non-negligible).

\hypertarget{prioritization-of-moral-character}{%
\paragraph{Prioritization of moral character}\label{prioritization-of-moral-character}}

We tested whether moral characters are prioritized by examining the population-level effects (also called fixed effect) of the three-level Bayesian hierarchical model of experiments 1a, 1b, 1c, 2, 5, and 6a. More specifically, we calculated the differences between the posterior distributions of the good/bad character and the neutral character and then tested these posterior distributions with the SEXIT approach.

\hypertarget{modulation-of-self-referential-processing}{%
\paragraph{Modulation of self-referential processing}\label{modulation-of-self-referential-processing}}

We tested the modulation effect of self-referential processing by examining the interaction between moral character and self-referential process for the three-level Bayesian hierarchical model of experiments 3a, 3b, and 6b. More specifically, we tested two possible explanations for the prioritization of good character: the valence effect alone or an interaction between the valence effect and the self-referential process. If the former is correct, then there will be no interaction between moral character and self-referential processing, i.e., the prioritization effect exhibits a similar pattern for both self- and other-referential conditions. On the other hand, if the spontaneous self-referential processing account is true, then there will be an interaction between the two factors, i.e., the prioritization effect exhibits different patterns for self- and other-referential conditions. To test the interaction, we calculated the posterior distribution of the difference of difference: \((good - neutral)_{self}\) vs.~\((good - neutral)_{other}\). We then tested the difference of difference with SEXIT framework.

\hypertarget{spontaneous-binding-between-the-self-and-good-character}{%
\paragraph{Spontaneous binding between the self and good character}\label{spontaneous-binding-between-the-self-and-good-character}}

For data from experiments 4a and 4b, we further examined whether the self-referential processing for moral characters is spontaneous (i.e., whether the good character is spontaneously bound with the self). For experiment 4a, if there exists a spontaneous binding between self and good character, there should be an interaction between moral character and self-referential processing. More specifically, we tested the posterior distributions of \(good_{self} - neutral_{self}\) and \(good_{other} - neutral_{other}\), as well as the difference between these differences with the SEXIT framework. For experiment 4b, if there exists a spontaneous binding between self and good character, then, there will be a self-other difference for some moral character conditions but not for other moral character conditions. More specifically, we tested the posteriors of \(good_{self} - good_{other}\), \(neutral_{self} - neutral_{other}\), and \(bad_{self} - bad_{other}\) as well as the difference between them with SEXIT framework.

\hypertarget{results}{%
\section{Results}\label{results}}

\hypertarget{prioritization-of-good-character}{%
\subsection{Prioritization of good character}\label{prioritization-of-good-character}}

To test whether moral characters are prioritized, we modeled data from experiments 1a, 1b, 1c, 2, 5, and 6a with three-level Bayesian hierarchical models. All these experiments shared similar designs and can be used for testing the prioritization effect of moral character. The valid and unique sample size is 192. Note that for both experiments 1a and 1b, two datasets were collected at different time points and locations, thus we treated them as independent samples. Here we only reported the population-level results of three-level Bayesian models, the detailed results of each experiment can be found in supplementary materials.

For the \emph{d} prime, results from the Bayesian model revealed a robust effect of moral character. Shapes associated with good characters (``good person'', ``kind person'' or a name associated with good behaviors) have higher sensitivity (median = 2.51, 95\% HDI = {[}2.23 2.78{]}) than shapes associated with neutral characters (median = 2.19, 95\% HDI = {[}1.88 2.50{]}), the difference (\(median_{diff}\) = 0.31, 95\% HDI {[}0, 0.62{]}) has a 97.31\% probability of being positive (\textgreater{} 0), 94.91\% of being significant (\textgreater{} 0.05). But we did not find a difference between shapes associated with bad characters (median = 2.25, 95\% HDI = {[}1.94 2.55{]}) and neutral character, the difference (\(median_{diff}\) = 0.05, 95\% HDI {[}-0.27, 0.38{]}) only has a 60.56\% probability of being positive (\textgreater{} 0), 49.34\% of being significant (\textgreater{} 0.05).

\begin{figure}
\centering
\includegraphics{Notebook_Pos_Self_Salience_APA_files/figure-latex/plot-bayes-meta-1-1.pdf}
\caption{\label{fig:plot-bayes-meta-1}Effect of moral character on perceptual matching}
\end{figure}

The results from reaction times data also found a robust effect of moral character for both match trials (see figure \ref{fig:plot-bayes-meta-1} C) and nonmatch trials (\textbf{see supplementary materials}). For match trials, shapes associated with good characters were faster (median = 579 ms, 95\% HDI = {[}500 661{]}) than shapes associated with neutral characters (median = 624 ms, 95\% HDI = {[}543 711{]}), the effect (\(median_{diff}\) = -44, 95\% HDI {[}-67, -24{]}) has a 99.94\% probability of being negative (\textless{} 0), 99.94\% of being significant (\textless{} -0.05). We also found that RTs to shapes associated with bad characters (median = 641 ms, 95\% HDI = {[}561 730{]}) were slower as compared to the neutral character, the effect (\(median_{diff}\) = 17, 95\% HDI {[}-6, 36{]}) has a 93.58\% probability of being positive (\textgreater{} 0), 93.55\% of being significant (\textgreater{} 0.05).

For the nonmatch trials, we found a similar pattern but a much smaller effect size. Shapes associated with good characters (median = 654 ms, 95\% HDI = {[}573 743{]}) were faster than shapes associated with neutral characters (median = 672 ms, 95\% HDI = {[}588 763{]}), the difference (\(median_{diff}\) = -18, 95\% HDI {[}-27, -8{]}) has a 99.91\% probability of being negative (\textless{} 0), 99.91\% of being significant (\textless{} -0.05). In contrast, the shapes associated with bad characters (median = 677 ms, 95\% HDI = {[}590 766{]}) were slower than shapes associated with neutral characters, the effect (\(median_{diff}\) = 5, 95\% HDI {[}-3, 13{]}) has a 92.43\% probability of being positive (\textgreater{} 0), 92.31\% of being significant (\textgreater{} 0.05).

\hypertarget{modulation-effect-self-referential-processing}{%
\subsection{Modulation effect self-referential processing}\label{modulation-effect-self-referential-processing}}

To test the modulation effect of self-referential processing, we also modeled data from three experiments (3a, 3b, and 6b) with three-level Bayesian models. These three experiments included 108 unique participants. We focused on the population-level effect of the interaction between self-referential processing and moral valence. Also, we examined the differences of differences, i.e., how the differences between good/bad characters and the neutral character under the self-referential conditions differ from that under other-referential conditions. The detailed results of each experiment can be found in supplementary materials.

For the \emph{d} prime, we found an interaction between the moral valence and self-referential processing: the good-neutral differences are larger for the self-referential condition than for the other-referential condition: The difference (\(median_{diff}\) = 0.48, 95\% HDI {[}-0.62, 1.65{]}) has a 93.04\% probability of being positive (\textgreater{} 0), 91.92\% of being significant (\textgreater{} 0.05). However, the bad-neutral differences (\(median_{diff}\) = 0.0087, 95\% HDI {[}-0.96, 1.00{]}) only have a 51.85\% probability of being positive (\textgreater{} 0), 41.29\% of being significant (\textgreater{} 0.05). Further analyses revealed that the prioritization effect of good character (as compared to neutral) only appeared for self-referential conditions but not other-referential conditions. The estimated \emph{d} prime for good-self was greater than neutral-self (\(median_{diff}\) = 0.54, 95\% HDI {[}-0.30, 1.41{]}), with a 95.99\% probability of being positive (\textgreater{} 0), 95.36\% of being significant (\textgreater{} 0.05). The differences between bad-self and neutral-self, good-other and neutral-other, and bad-other and neutral-other are all centered around zero (see Figure \ref{fig:plot-bayes2}, B, D).

\begin{figure}
\centering
\includegraphics{Notebook_Pos_Self_Salience_APA_files/figure-latex/plot-bayes2-1.pdf}
\caption{\label{fig:plot-bayes2}Interaction between moral character and self-referential}
\end{figure}

For the RTs of matched trials, we also found an interaction between moral valence and self-referential processing: the good-neutral differences were larger for the self- than the other-referential conditions (\(median_{diff}\) = -148, 95\% HDI {[}-413, 73{]}) has a 96.05\% probability of being negative (\textless{} 0), 96.05\% of being significant (\textless{} -0.05). However, this pattern was much weaker for bad-neutral differences (\(median_{diff}\) = -47, 95\% HDI {[}-280, 182{]}) has a 79.91\% probability of being negative (\textless{} 0) and 79.88\% of being significant (\textless{} -0.05). Bayes analyses revealed a robust good-self prioritization effect as compared to neutral-self (\(median_{diff}\) = -59, 95\% HDI {[}-115, -22{]}) has a 98.87\% probability of being negative (\textless{} 0) and 98.87\% of being significant (\textless{} -0.05)) and good-other (\(median_{diff}\) = -109, 95\% HDI {[}-227, -31{]}) has a 98.65\% probability of being negative (\textless{} 0) and 98.65\% of being significant (\textless{} -0.05)) conditions. Similar to the results of \emph{d'}, we found that participants responded slower for both good character than for the neutral character when they referred to others, \(median_{diff}\) = 85.01, 95\% HDI {[}-112, 328{]}) has a 92.16\% probability of being positive (\textgreater{} 0) and 92.15\% of being significant (\textgreater{} 0.05). A similar pattern was also found for the bad character when referred to others: bad-other responded slower than neutral-other, \(median_{diff}\) = 44, 95\% HDI {[}-146, 268{]}) has an 80.03\% probability of being positive (\textgreater{} 0) and 79.99\% of being significant (\textgreater{} 0.05). See Figure \ref{fig:plot-bayes2}.

These results suggested that the prioritization of good character is not solely driven by the valence of moral character. Instead, the self-referential processing modulated the prioritization of good character: good character was prioritized only when it was self-referential. When the moral character was other-referential, responses to both good and bad characters were slowed down.

\hypertarget{spontaneous-binding-between-the-good-character-and-the-self}{%
\subsection{Spontaneous binding between the good character and the self}\label{spontaneous-binding-between-the-good-character-and-the-self}}

Experiments 4a and 4b were designed to test whether the good character and self-referential processing bind together spontaneously. Because these two experiments have different experimental designs, we model their data separately.

In experiment 4a, where ``self'' vs.~``other'' were task-relevant and moral character were task-irrelevant, we found the ``self'' conditions performed better than the ``other'' conditions for both \emph{d} prime and reaction times. This pattern is consistent with previous studies (e.g., Sui et al. (2012)).

More importantly, we found evidence, albeit weak, that task-irrelevant moral character also played a role. For shapes associated with ``self'', \emph{d'} was greater when shapes had a good character inside (median = 2.83, 95\% HDI {[}2.63 3.01{]}) than shapes that have neutral character (median = 2.74, 95\% HDI {[}2.58 2.95{]}), the difference (median = 0.08, 95\% HDI {[}-0.10, 0.27{]}) has an 81.60\% probability of being positive (\textgreater{} 0), 64.33\% of being significant (\textgreater{} 0.05). For shapes associated with ``other'', the pattern reversed: \emph{d} prime was smaller when shapes had a good character inside (median = 1.87, 95\% HDI {[}1.71 2.04{]}) than had neutral (median = 1.96, 95\% HDI {[}1.80 2.14{]}), the difference (median = -0.09, 95\% HDI {[}-0.25, 0.05{]}) has an 89.03\% probability of being negative (\textless{} 0), 71.38\% of being significant (\textless{} -0.05). The difference between these two effects (median = 0.18, 95\% HDI {[}-0.06, 0.43{]}) has a 92.88\% probability of being positive (\textgreater{} 0), 85.08\% being significant (\textgreater{} 0.05). See Figure \ref{fig:plot-exp4-all}.

A similar pattern was found for RTs in matched trials. For the ``self'' condition, when a good character was presented inside the shapes, the RTs (median = 641, 95\% HDI {[}623 662{]}) were faster than when a neutral character (median = 649, 95\% HDI {[}631 668{]}) was inside, the effect (median = -8, 95\% HDI {[}-17, 2{]}) has a 94.55\% probability of being negative (\textless{} 0) and 94.50\% of being significant (\textless{} -0.05). In contrast, RTs for shapes associated with good character inside (median = 733, 95\% HDI {[}711 755{]}) were slower than those with neutral character (median = 722, 95\% HDI {[}702 741{]}) inside, the effect (median = 12, 95\% HDI {[}-4, 28{]}) has a 93.00\% probability of being positive (\textgreater{} 0) and 92.83\% of being significant (\textgreater{} 0.05). The difference between the effects (median = -19, 95\% HDI {[}-43, 4{]}) has a 94.90\% probability of being negative (\textless{} 0) and 94.88\% of being significant (\textless{} -0.05).

In experiment 4b, where moral characters were task-relevant and ``self'' vs ``other'' were task-irrelevant, we found a main effect of moral character: performance for shapes associated with good characters was better than other-related conditions on both \emph{d'} and reaction times. This pattern, again, shows a robust prioritization effect of good character.

Most importantly, we found evidence that task-irrelevant labels, ``self'' or ``other'', also played a role. For shapes associated with good character, the \emph{d} prime was greater when shapes had a ``self'' inside than with ``other'' inside (\(mean_{diff}\) = 0.14, 95\% HDI {[}-0.05, 0.34{]}) has a 92.35\% probability of being positive (\textgreater{} 0) and 81.80\% of being significant (\textgreater{} 0.05). However, the difference did not occur when the target shape where associated with ``neutral'' (\(mean_{diff}\) = 0.04, 95\% HDI {[}-0.13, 0.22{]}) and has a 67.20\% probability of being positive (\textgreater{} 0) and 44.80\% of being significant (\textgreater{} 0.05). Neither for the ``bad'' person condition: \(mean_{diff}\) = 0.10, 95\% HDI {[}-0.16, 0.37{]}) has a 77.03\% probability of being positive (\textgreater{} 0) and 64.62\% of being significant (\textgreater{} 0.05).

\begin{figure}
\centering
\includegraphics{Notebook_Pos_Self_Salience_APA_files/figure-latex/plot-exp4-all-1.pdf}
\caption{\label{fig:plot-exp4-all}Experiment 4: Implicit binding between good character and the self.}
\end{figure}

The same trend appeared for the RT data. For shapes associated with good character, having a ``self'' inside shapes reduced the reaction times as compared to having an ``other'' inside the shapes (\(mean_{diff}\) = -55, 95\% HDI {[}-75, -35{]}) has a 100\% probability of being negative (\textless{} 0) and 100.00\% of being significant (\textless{} -0.05). However, when the shapes were associated with the neutral character, having a ``self'' inside shapes increased the RTs: \(mean_{diff}\) = 11, 95\% HDI {[}1, 21{]}) has a 98.20\% probability of being positive (\textgreater{} 0) and 98.15\% of being significant (\textgreater{} 0.05). While having ``self'' slightly increased the RT than having ``other'' inside the shapes for the bad character: \(mean_{diff}\) = 5, 95\% HDI {[}-17, 27{]}) has a 69.45\% probability of being positive (\textgreater{} 0) and 69.27\% of being significant (\textgreater{} 0.05), See Figure \ref{fig:plot-exp4-all}.

\hypertarget{discussion}{%
\section{Discussion}\label{discussion}}

Across nine experiments, we explored the prioritization effect of moral character and the underlying mechanism by a combination of social associative learning and perceptual matching task. First, we found a robust effect that good character was prioritized in the shape-label matching task across five experiments. Second, across three experiments, we found that the prioritization of good character was not solely driven by moral valence itself, i.e., ``good'' vs ``bad''. Instead, this effect was modulated by self-referential processing: prioritization only occurred when moral characters are self-referential. Finally, the prioritization of the combination of good character and self occurred, albeit weak, even when either the self- or character-related information was irrelevant to the experimental task (experiment 4a and 4b). In contrast, performance to the combination of good character and ``other'', explicitly or implicitly, was worse than the combination of neutral character and ``other''. Together, these results highlighted the importance of the self in perceiving information related to moral characters, suggesting a spontaneous self-referential process when making perceptual decision-making for moral characters. These results are in line with a growing literature on the social and relational nature of perception (Xiao, Coppin, and Bavel (2016); Freeman, Stolier, and Brooks (2020); @hafri\_perception\_2021) and deepened our understanding of mechanisms of perceptual decision-making of moral information.

The current study provided robust evidence for the prioritization of good character in perceptual decision-making. The existence of the effect of moral valence on perception has been disputed. For instance, (E. Anderson et al., 2011) reported that faces associated with bad social behavior capture attention more rapidly, however, an independent team failed to replicate the effect (Stein et al., 2017). Another study by Gantman and Van Bavel (2014) found that moral words are more likely to be judged as words when it was presented subliminally, however, this effect may be caused by semantic priming instead of morality (Firestone \& Scholl, 2015; Jussim et al., 2016). In the current study, we found the prioritization effect across five experiments, the sample size of individual experiments and combined provide strong evidence for the existence of the effect. Moreover, the associative learning task allowed us to eliminate the semantic priming effect for two reasons. First, associations between shapes and moral characters were acquired right before the perceptual matching task, semantic priming from pre-existed knowledge was impossible. Second, there were only a few pairs of stimuli were used and each stimulus represented different conditions, making it impossible for priming between trials. Importantly, a series of control experiments (1b, 1c, and 2) further excluded other confounding factors such as familiarity, presenting sequence, or words-based associations, suggesting that it was the moral content that drove the prioritization of good character.

The robust prioritization of good character found in the current study was incongruent with previous moral perception studies, which usually reported a negativity effect, i.e., information related to bad character is processed preferentially (E. Anderson et al., 2011; Eiserbeck \& Abdel Rahman, 2020). This discrepancy may be caused by the experimental task: while in many previous moral perception studies, the participants were asked to detect the existence of a stimulus, the current task asked participants to recognize a pattern. In other words, previous studies targeted early stages of perception while the current task focused more on decision-making at a relatively later stage of information processing. This discrepancy is consistent with the pattern found in studies with emotional stimuli (Pool, Brosch, Delplanque, \& Sander, 2016).

We expanded previous moral perception studies by focusing on the agent who made the perceptual decision-making and examined the interaction between moral valence and self-referential processing. Our results revealed that prioritization of good character is modulated by self-referential processing: the good character was prioritized when it was related to the ``self'', even when the self-relatedness was task-irrelevant. By contrast, good character information was not prioritized when it was associated with ``other''. The modulation effect of self-referential processing was large when the relationship between moral character and the self was explicit, which is consistent with previous studies that only positive aspects of the self are prioritized (Hu et al., 2020). More importantly, the effect persisted when the relationship between moral character and self-information was implicit, suggesting spontaneous self-referential processing when both pieces of information were presented. A possible explanation for this spontaneous self-referential of good character is that the positive moral self-view is central to our identity (Freitas, Cikara, Grossmann, \& Schlegel, 2017; Strohminger, Knobe, \& Newman, 2017) and the motivation to maintain a moral self-view influences how we perceive (e.g., Ma \& Han, 2010) and remember (e.g., Carlson, Maréchal, Oud, Fehr, \& Crockett, 2020; Stanley, Henne, \& De Brigard, 2019).

Although the results here revealed the prioritization of good character in perceptual decision-making, we did not claim that the motivation of a moral self-view \emph{penetrates} perception. The perceptual decision-making process involves processes more than just encoding the sensory inputs. To fully account for the nuance of behavioral data and/or related data collected from other modules (e.g., Sui, He, Golubickis, Svensson, \& Neil Macrae, 2023), we need computational models and an integrative experimental approach (Almaatouq et al., 2022). For example, sequential sampling models suggest that, when making a perceptual decision, the agent continuously accumulates evidence until the amount of evidence passed a threshold, then a decision is made (Chuan-Peng et al., 2022; Forstmann, Ratcliff, \& Wagenmakers, 2016; Ratcliff, Smith, Brown, \& McKoon, 2016). In these models, the evidence, or decision variable, can accumulate from both sensory information but also memory (Shadlen \& Shohamy, 2016). Recently, applications of sequential sample models to perceptual matching tasks also suggest that different processes may contribute to the prioritization effect of self (Golubickis et al., 2017) or good self (Hu et al., 2020). Similarly, reinforcement learning models also revealed that the key difference between self- and other-referential learning lies in the learning rate (Lockwood et al., 2018). These studies suggest that computational models are needed to disentangle the cognitive processes underlying the prioritization of good character.

\hypertarget{references}{%
\section{References}\label{references}}

\begingroup
\setlength{\parindent}{-0.5in}
\setlength{\leftskip}{0.5in}

\hypertarget{refs}{}
\begin{CSLReferences}{1}{0}
\leavevmode\vadjust pre{\hypertarget{ref-Almaatouq_BBS_2022}{}}%
Almaatouq, A., Griffiths, T. L., Suchow, J. W., Whiting, M. E., Evans, J., \& Watts, D. J. (2022). \emph{Beyond {Playing} 20 {Questions} with {Nature}: {Integrative} {Experiment} {Design} in the {Social} and {Behavioral} {Sciences}}. 1--55. \url{https://doi.org/10.1017/S0140525X22002874}

\leavevmode\vadjust pre{\hypertarget{ref-anderson_value_2011}{}}%
Anderson, B. A., Laurent, P. A., \& Yantis, S. (2011). Value-driven attentional capture. \emph{Proceedings of the National Academy of Sciences}, \emph{108}(25), 10367--10371. \url{https://doi.org/10.1073/pnas.1104047108}

\leavevmode\vadjust pre{\hypertarget{ref-anderson_visual_2011}{}}%
Anderson, E., Siegel, E. H., Bliss-Moreau, E., \& Barrett, L. F. (2011). The visual impact of gossip. \emph{Science}, \emph{332}(6036), 1446--1448. \url{https://doi.org/10.1126/science.1201574}

\leavevmode\vadjust pre{\hypertarget{ref-atlas_instructions_2022}{}}%
Atlas, L. Y., Dildine, T. C., Palacios-Barrios, E. E., Yu, Q., Reynolds, R. C., Banker, L. A., \ldots{} Pine, D. S. (2022). Instructions and experiential learning have similar impacts on pain and pain-related brain responses but produce dissociations in value-based reversal learning. \emph{eLife}, \emph{11}, e73353. \url{https://doi.org/10.7554/eLife.73353}

\leavevmode\vadjust pre{\hypertarget{ref-Buxfcrkner_2017}{}}%
Bürkner, P.-C. (2017). Brms: An r package for bayesian multilevel models using stan {[}Journal Article{]}. \emph{Journal of Statistical Software; Vol 1, Issue 1 (2017)}. Retrieved from \href{https://www.jstatsoft.org/v080/i01\%0Ahttp://dx.doi.org/10.18637/jss.v080.i01}{https://www.jstatsoft.org/v080/i01
http://dx.doi.org/10.18637/jss.v080.i01}

\leavevmode\vadjust pre{\hypertarget{ref-carlson_nat_comm_2020}{}}%
Carlson, R. W., Maréchal, M. A., Oud, B., Fehr, E., \& Crockett, M. J. (2020). Motivated misremembering of selfish decisions. \emph{Nature Communications}, \emph{11}(1), 2100. \url{https://doi.org/10.1038/s41467-020-15602-4}

\leavevmode\vadjust pre{\hypertarget{ref-Carpenter_2017_stan}{}}%
Carpenter, B., Gelman, A., Hoffman, M. D., Lee, D., Goodrich, B., Betancourt, M., \ldots{} Riddell, A. (2017). Stan: A probabilistic programming language {[}Journal Article{]}. \emph{Journal of Statistical Software}, \emph{76}(1). \url{https://doi.org/10.18637/jss.v076.i01}

\leavevmode\vadjust pre{\hypertarget{ref-Hu_hitchhikers_2022}{}}%
Chuan-Peng, H., Geng, H., Zhang, L., Fengler, A., Frank, M., \& Zhang, R.-Y. (2022). \emph{A {Hitchhiker}'s {Guide} to {Bayesian} {Hierarchical} {Drift}-{Diffusion} {Modeling} with {dockerHDDM}}. PsyArXiv. \url{https://doi.org/10.31234/osf.io/6uzga}

\leavevmode\vadjust pre{\hypertarget{ref-deltomme_instructed_2018}{}}%
Deltomme, B., Mertens, G., Tibboel, H., \& Braem, S. (2018). Instructed fear stimuli bias visual attention. \emph{Acta Psychologica}, \emph{184}, 31--38. \url{https://doi.org/10.1016/j.actpsy.2017.08.010}

\leavevmode\vadjust pre{\hypertarget{ref-dunbar_gossip_2004}{}}%
Dunbar, R. I. M. (2004). Gossip in evolutionary perspective. \emph{Review of General Psychology}, \emph{8}(2), 100--110. \url{https://doi.org/10.1037/1089-2680.8.2.100}

\leavevmode\vadjust pre{\hypertarget{ref-eiserbeck_visual_2020}{}}%
Eiserbeck, A., \& Abdel Rahman, R. (2020). Visual consciousness of faces in the attentional blink: Knowledge-based effects of trustworthiness dominate over appearance-based impressions. \emph{Consciousness and Cognition}, \emph{83}, 102977. \url{https://doi.org/10.1016/j.concog.2020.102977}

\leavevmode\vadjust pre{\hypertarget{ref-ellemers_morality_2018}{}}%
Ellemers, N. (2018). Morality and social identity. In M. van Zomeren \& J. F. Dovidio (Eds.), \emph{The oxford handbook of the human essence} (pp. 147--158). New York, {NY}, {US}: Oxford University Press.

\leavevmode\vadjust pre{\hypertarget{ref-firestone_cognition_2015}{}}%
Firestone, C., \& Scholl, B. J. (2015). Enhanced visual awareness for morality and pajamas? Perception vs. Memory in "top-down" effects. \emph{Cognition}, \emph{136}, 409--416. \url{https://doi.org/10.1016/j.cognition.2014.10.014}

\leavevmode\vadjust pre{\hypertarget{ref-Firestone_2016_BBS}{}}%
Firestone, C., \& Scholl, B. J. (2016a). Cognition does not affect perception: {Evaluating} the evidence for {``top-down''} effects. \emph{Behavioral and Brain Sciences}, \emph{39}, e229. \url{https://doi.org/10.1017/S0140525X15000965}

\leavevmode\vadjust pre{\hypertarget{ref-firestone_moral_2016}{}}%
Firestone, C., \& Scholl, B. J. (2016b). {``{Moral} {Perception}''} {Reflects} {Neither} {Morality} {Nor} {Perception}. \emph{Trends in Cognitive Sciences}, \emph{20}(2), 75--76. \url{https://doi.org/10.1016/j.tics.2015.10.006}

\leavevmode\vadjust pre{\hypertarget{ref-forstmann_ann_rev_2016}{}}%
Forstmann, B. U., Ratcliff, R., \& Wagenmakers, E.-J. (2016). Sequential {Sampling} {Models} in {Cognitive} {Neuroscience}: {Advantages}, {Applications}, and {Extensions}. \emph{Annual Review of Psychology}, \emph{67}(1). \url{https://doi.org/10.1146/annurev-psych-122414-033645}

\leavevmode\vadjust pre{\hypertarget{ref-freeman_chapter_2020}{}}%
Freeman, J. B., Stolier, R. M., \& Brooks, J. A. (2020). Chapter five - dynamic interactive theory as a domain-general account of social perception. In B. Gawronski (Ed.), \emph{Advances in experimental social psychology} (Vol. 61, pp. 237--287). Academic Press. \url{https://doi.org/10.1016/bs.aesp.2019.09.005}

\leavevmode\vadjust pre{\hypertarget{ref-freitas_origins_2017}{}}%
Freitas, J. D., Cikara, M., Grossmann, I., \& Schlegel, R. (2017). Origins of the belief in good true selves. \emph{Trends in Cognitive Sciences}, \emph{21}(9), 634--636. \url{https://doi.org/10.1016/j.tics.2017.05.009}

\leavevmode\vadjust pre{\hypertarget{ref-gantman_moral_2015}{}}%
Gantman, A. P., \& Bavel, J. J. V. (2015). Moral {Perception}. \emph{Trends in Cognitive Sciences}, \emph{19}(11), 631--633. \url{https://doi.org/10.1016/j.tics.2015.08.004}

\leavevmode\vadjust pre{\hypertarget{ref-gantman_see_2016}{}}%
Gantman, A. P., \& Bavel, J. J. V. (2016). See for {Yourself}: {Perception} {Is} {Attuned} to {Morality}. \emph{Trends in Cognitive Sciences}, \emph{20}(2), 76--77. \url{https://doi.org/10.1016/j.tics.2015.12.001}

\leavevmode\vadjust pre{\hypertarget{ref-gantman_moral_2014}{}}%
Gantman, A. P., \& Van Bavel, J. J. (2014). The moral pop-out effect: Enhanced perceptual awareness of morally relevant stimuli. \emph{Cognition}, \emph{132}(1), 22--29. \url{https://doi.org/10.1016/j.cognition.2014.02.007}

\leavevmode\vadjust pre{\hypertarget{ref-Goh_2016_mini}{}}%
Goh, J. X., Hall, J. A., \& Rosenthal, R. (2016). Mini meta-analysis of your own studies: Some arguments on why and a primer on how. \emph{Social and Personality Psychology Compass}, \emph{10}(10), 535--549. \url{https://doi.org/10.1111/spc3.12267}

\leavevmode\vadjust pre{\hypertarget{ref-golubickis_self-prioritization_2017}{}}%
Golubickis, M., Falben, J. K., Sahraie, A., Visokomogilski, A., Cunningham, W. A., Sui, J., \& Macrae, C. N. (2017). Self-prioritization and perceptual matching: The effects of temporal construal. \emph{Memory \& Cognition}, \emph{45}(7), 1223--1239. \url{https://doi.org/10.3758/s13421-017-0722-3}

\leavevmode\vadjust pre{\hypertarget{ref-goodwin_moral_2014}{}}%
Goodwin, G. P., Piazza, J., \& Rozin, P. (2014). Moral character predominates in person perception and evaluation. \emph{Journal of Personality and Social Psychology}, \emph{106}(1), 148--168. \url{https://doi.org/10.1037/a0034726}

\leavevmode\vadjust pre{\hypertarget{ref-Hu_2020_GoodSelf}{}}%
Hu, C.-P., Lan, Y., Macrae, C. N., \& Sui, J. (2020). Good me bad me: Does valence influence self-prioritization during perceptual decision-making? \emph{Collabra: Psychology}, \emph{6}(1), 20. \url{https://doi.org/10.1525/collabra.301}

\leavevmode\vadjust pre{\hypertarget{ref-juechems_where_2019}{}}%
Juechems, K., \& Summerfield, C. (2019). Where does value come from? \emph{Trends in Cognitive Sciences}, \emph{23}(10), 836--850. \url{https://doi.org/10.1016/j.tics.2019.07.012}

\leavevmode\vadjust pre{\hypertarget{ref-jussim_interpretations_2016}{}}%
Jussim, L., Crawford, J. T., Anglin, S. M., Stevens, S. T., \& Duarte, J. L. (2016). Interpretations and methods: {Towards} a more effectively self-correcting social psychology. \emph{Journal of Experimental Social Psychology}, \emph{66}, 116--133. \url{https://doi.org/10.1016/j.jesp.2015.10.003}

\leavevmode\vadjust pre{\hypertarget{ref-Liu_2020_JOPD}{}}%
Liu, Q., Wang, F., Yan, W., Peng, K., Sui, J., \& Hu, C.-P. (2020). Questionnaire data from the revision of a chinese version of free will and determinism plus scale. \emph{Journal of Open Psychology Data}, \emph{8}(1), 1. \url{https://doi.org/10.5334/jopd.49/}

\leavevmode\vadjust pre{\hypertarget{ref-lockwood_nat_comm_2018}{}}%
Lockwood, P. L., Wittmann, M. K., Apps, M. A. J., Klein-FlÃŒgge, M. C., Crockett, M. J., Humphreys, G. W., \& Rushworth, M. F. S. (2018). Neural mechanisms for learning self and other ownership. \url{https://doi.org/10.1038/s41467-018-07231-9}

\leavevmode\vadjust pre{\hypertarget{ref-Ma_JEPHPP_2010}{}}%
Ma, Y., \& Han, S. (2010). Why we respond faster to the self than to others? {An} implicit positive association theory of self-advantage during implicit face recognition. \emph{Journal of Experimental Psychology: Human Perception and Performance}, \emph{36}, 619--633. \url{https://doi.org/10.1037/a0015797}

\leavevmode\vadjust pre{\hypertarget{ref-makowski_indices_2019}{}}%
Makowski, D., Ben-Shachar, M. S., Chen, S. H. A., \& Lüdecke, D. (2019). Indices of {Effect} {Existence} and {Significance} in the {Bayesian} {Framework}. \emph{Frontiers in Psychology}, \emph{10}. \url{https://doi.org/10.3389/fpsyg.2019.02767}

\leavevmode\vadjust pre{\hypertarget{ref-makowski_bayestestr_2019}{}}%
Makowski, D., Ben-Shachar, M. S., \& Lüdecke, D. (2019). {bayestestR}: {Describing} {Effects} and their {Uncertainty}, {Existence} and {Significance} within the {Bayesian} {Framework}. \emph{Journal of Open Source Software}, \emph{4}(40), 1541. \url{https://doi.org/10.21105/joss.01541}

\leavevmode\vadjust pre{\hypertarget{ref-ohman_face_2001}{}}%
Ohman, A., Lundqvist, D., \& Esteves, F. (2001). The face in the crowd revisited: A threat advantage with schematic stimuli. \emph{Journal of Personality and Social Psychology}, \emph{80}(3), 381--396. \url{https://doi.org/10.1037/0022-3514.80.3.381}

\leavevmode\vadjust pre{\hypertarget{ref-pool_attentional_2016}{}}%
Pool, E., Brosch, T., Delplanque, S., \& Sander, D. (2016). Attentional bias for positive emotional stimuli: A meta-analytic investigation. \url{https://doi.org/10.1037/bul0000026}

\leavevmode\vadjust pre{\hypertarget{ref-ratcliff_tics_2016}{}}%
Ratcliff, R., Smith, P. L., Brown, S. D., \& McKoon, G. (2016). Diffusion {Decision} {Model}: {Current} {Issues} and {History}. \emph{Trends in Cognitive Sciences}, \emph{20}(4), 260--281. \url{https://doi.org/10.1016/j.tics.2016.01.007}

\leavevmode\vadjust pre{\hypertarget{ref-Rouder_2005_BHM_SDT}{}}%
Rouder, J. N., \& Lu, J. (2005). An introduction to bayesian hierarchical models with an application in the theory of signal detection {[}Journal Article{]}. \emph{Psychonomic Bulletin \& Review}, \emph{12}(4), 573--604. \url{https://doi.org/10.3758/bf03196750}

\leavevmode\vadjust pre{\hypertarget{ref-rousselet_reaction_2020}{}}%
Rousselet, G. A., \& Wilcox, R. R. (2020). Reaction times and other skewed distributions: Problems with the mean and the median. \emph{Meta-Psychology}, \emph{4}. \url{https://doi.org/10.15626/MP.2019.1630}

\leavevmode\vadjust pre{\hypertarget{ref-shadlen_decision_2016}{}}%
Shadlen, M. N., \& Shohamy, D. (2016). Decision {Making} and {Sequential} {Sampling} from {Memory}. \emph{Neuron}, \emph{90}(5), 927--939. \url{https://doi.org/10.1016/j.neuron.2016.04.036}

\leavevmode\vadjust pre{\hypertarget{ref-shore_social_2013}{}}%
Shore, D. M., \& Heerey, E. A. (2013). Do social utility judgments influence attentional processing? \emph{Cognition}, \emph{129}(1), 114--122. \url{https://doi.org/10.1016/j.cognition.2013.06.011}

\leavevmode\vadjust pre{\hypertarget{ref-Simmons_2013_life}{}}%
Simmons, J. P., Nelson, L. D., \& Simonsohn, U. (2013). \emph{Life after p-hacking} {[}Conference Proceedings{]}. \url{https://doi.org/10.2139/ssrn.2205186}

\leavevmode\vadjust pre{\hypertarget{ref-stanley_remembering_2019}{}}%
Stanley, M. L., Henne, P., \& De Brigard, F. (2019). Remembering moral and immoral actions in constructing the self. \emph{Memory \& Cognition}, \emph{47}(3), 441--454. \url{https://doi.org/10.3758/s13421-018-0880-y}

\leavevmode\vadjust pre{\hypertarget{ref-stein_no_2017}{}}%
Stein, T., Grubb, C., Bertrand, M., Suh, S. M., \& Verosky, S. C. (2017). No impact of affective person knowledge on visual awareness: Evidence from binocular rivalry and continuous flash suppression. \emph{Emotion}, \emph{17}(8), 1199--1207. \url{https://doi.org/10.1037/emo0000305}

\leavevmode\vadjust pre{\hypertarget{ref-strohminger_true_2017}{}}%
Strohminger, N., Knobe, J., \& Newman, G. (2017). The true self: A psychological concept distinct from the self: \emph{Perspectives on Psychological Science}. \url{https://doi.org/10.1177/1745691616689495}

\leavevmode\vadjust pre{\hypertarget{ref-sui_electrophysiological_2023}{}}%
Sui, J., He, X., Golubickis, M., Svensson, S. L., \& Neil Macrae, C. (2023). Electrophysiological correlates of self-prioritization. \emph{Consciousness and Cognition}, \emph{108}, 103475. \url{https://doi.org/10.1016/j.concog.2023.103475}

\leavevmode\vadjust pre{\hypertarget{ref-Sui_2012_JEPHPP}{}}%
Sui, J., He, X., \& Humphreys, G. W. (2012). Perceptual effects of social salience: Evidence from self-prioritization effects on perceptual matching. \emph{Journal of Experimental Psychology: Human Perception and Performance}, \emph{38}(5), 1105--1117. \url{https://doi.org/10.1037/a0029792}

\leavevmode\vadjust pre{\hypertarget{ref-sui_tics_2015}{}}%
Sui, J., \& Humphreys, G. W. (2015). The {Integrative} {Self}: {How} {Self}-{Reference} {Integrates} {Perception} and {Memory}. \emph{Trends in Cognitive Sciences}, \emph{19}(12), 719--728. \url{https://doi.org/10.1016/j.tics.2015.08.015}

\leavevmode\vadjust pre{\hypertarget{ref-sui_self_attention_2019}{}}%
Sui, J., \& Rotshtein, P. (2019). Self-prioritization and the attentional systems. \emph{Current Opinion in Psychology}, \emph{29}, 148--152. \url{https://doi.org/10.1016/j.copsyc.2019.02.010}

\leavevmode\vadjust pre{\hypertarget{ref-unkelbach_chapter_2020}{}}%
Unkelbach, C., Alves, H., \& Koch, A. (2020). Chapter three - negativity bias, positivity bias, and valence asymmetries: Explaining the differential processing of positive and negative information. In B. Gawronski (Ed.), \emph{Advances in experimental social psychology} (Vol. 62, pp. 115--187). Academic Press. \url{https://doi.org/10.1016/bs.aesp.2020.04.005}

\leavevmode\vadjust pre{\hypertarget{ref-willis_first_2006}{}}%
Willis, J., \& Todorov, A. (2006). First impressions: Making up your mind after a 100-ms exposure to a face. \emph{Psychological Science}, \emph{17}(7), 592--598. \url{https://doi.org/10.1111/j.1467-9280.2006.01750.x}

\leavevmode\vadjust pre{\hypertarget{ref-xiao_perceiving_2016}{}}%
Xiao, Y. J., Coppin, G., \& Bavel, J. J. V. (2016). Perceiving the world through group-colored glasses: A perceptual model of intergroup relations. \emph{Psychological Inquiry}, \emph{27}(4), 255--274. \url{https://doi.org/10.1080/1047840X.2016.1199221}

\end{CSLReferences}

\endgroup


\end{document}
