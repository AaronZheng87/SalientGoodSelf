\documentclass[man]{apa6}
\usepackage{lmodern}
\usepackage{amssymb,amsmath}
\usepackage{ifxetex,ifluatex}
\usepackage{fixltx2e} % provides \textsubscript
\ifnum 0\ifxetex 1\fi\ifluatex 1\fi=0 % if pdftex
  \usepackage[T1]{fontenc}
  \usepackage[utf8]{inputenc}
\else % if luatex or xelatex
  \ifxetex
    \usepackage{mathspec}
  \else
    \usepackage{fontspec}
  \fi
  \defaultfontfeatures{Ligatures=TeX,Scale=MatchLowercase}
\fi
% use upquote if available, for straight quotes in verbatim environments
\IfFileExists{upquote.sty}{\usepackage{upquote}}{}
% use microtype if available
\IfFileExists{microtype.sty}{%
\usepackage{microtype}
\UseMicrotypeSet[protrusion]{basicmath} % disable protrusion for tt fonts
}{}
\usepackage{hyperref}
\hypersetup{unicode=true,
            pdftitle={Open notebook of perpecptual salience of positive self},
            pdfauthor={Chuan-Peng Hu, Kaiping Peng, \& Jie Sui},
            pdfkeywords={Perceptual decision-making, Self, positive bias},
            pdfborder={0 0 0},
            breaklinks=true}
\urlstyle{same}  % don't use monospace font for urls
\usepackage{graphicx,grffile}
\makeatletter
\def\maxwidth{\ifdim\Gin@nat@width>\linewidth\linewidth\else\Gin@nat@width\fi}
\def\maxheight{\ifdim\Gin@nat@height>\textheight\textheight\else\Gin@nat@height\fi}
\makeatother
% Scale images if necessary, so that they will not overflow the page
% margins by default, and it is still possible to overwrite the defaults
% using explicit options in \includegraphics[width, height, ...]{}
\setkeys{Gin}{width=\maxwidth,height=\maxheight,keepaspectratio}
\IfFileExists{parskip.sty}{%
\usepackage{parskip}
}{% else
\setlength{\parindent}{0pt}
\setlength{\parskip}{6pt plus 2pt minus 1pt}
}
\setlength{\emergencystretch}{3em}  % prevent overfull lines
\providecommand{\tightlist}{%
  \setlength{\itemsep}{0pt}\setlength{\parskip}{0pt}}
\setcounter{secnumdepth}{0}
% Redefines (sub)paragraphs to behave more like sections
\ifx\paragraph\undefined\else
\let\oldparagraph\paragraph
\renewcommand{\paragraph}[1]{\oldparagraph{#1}\mbox{}}
\fi
\ifx\subparagraph\undefined\else
\let\oldsubparagraph\subparagraph
\renewcommand{\subparagraph}[1]{\oldsubparagraph{#1}\mbox{}}
\fi

%%% Use protect on footnotes to avoid problems with footnotes in titles
\let\rmarkdownfootnote\footnote%
\def\footnote{\protect\rmarkdownfootnote}


  \title{Open notebook of perpecptual salience of positive self}
    \author{Chuan-Peng Hu\textsuperscript{1,2}, Kaiping Peng\textsuperscript{1}, \& Jie Sui\textsuperscript{3}}
    \date{}
  
\shorttitle{Salient Positive Self}
\affiliation{
\vspace{0.5cm}
\textsuperscript{1} Tsinghua University, 100084 Beijing, China\\\textsuperscript{2} German Resilience Center, 55131 Mainz, Germany\\\textsuperscript{3} University of Bath, Bath, UK}
\keywords{Perceptual decision-making, Self, positive bias\newline\indent Word count: X}
\usepackage{csquotes}
\usepackage{upgreek}
\captionsetup{font=singlespacing,justification=justified}

\usepackage{longtable}
\usepackage{lscape}
\usepackage{multirow}
\usepackage{tabularx}
\usepackage[flushleft]{threeparttable}
\usepackage{threeparttablex}

\newenvironment{lltable}{\begin{landscape}\begin{center}\begin{ThreePartTable}}{\end{ThreePartTable}\end{center}\end{landscape}}

\makeatletter
\newcommand\LastLTentrywidth{1em}
\newlength\longtablewidth
\setlength{\longtablewidth}{1in}
\newcommand{\getlongtablewidth}{\begingroup \ifcsname LT@\roman{LT@tables}\endcsname \global\longtablewidth=0pt \renewcommand{\LT@entry}[2]{\global\advance\longtablewidth by ##2\relax\gdef\LastLTentrywidth{##2}}\@nameuse{LT@\roman{LT@tables}} \fi \endgroup}


\DeclareDelayedFloatFlavor{ThreePartTable}{table}
\DeclareDelayedFloatFlavor{lltable}{table}
\DeclareDelayedFloatFlavor*{longtable}{table}
\makeatletter
\renewcommand{\efloat@iwrite}[1]{\immediate\expandafter\protected@write\csname efloat@post#1\endcsname{}}
\makeatother
\usepackage{lineno}

\linenumbers

\authornote{Chuan-Peng Hu, Department of Psychology, Tsinghua University, 100084 Beijing, China; Germany Resilience Center, 55131 Mainz, Germany.
Kaiping Peng, Department of Psychology, Tsinghua University, 100084 Beijing, China.
Jie Sui, Department of Psychology, the University of Bath, Bath, UK.

Authors contriubtion: CPH, JS, \& KP design the study, CPH collected the data, CPH analyzed the data and drafted the manuscript. KP \& JS supported this project.

Correspondence concerning this article should be addressed to Chuan-Peng Hu, Langenbeckstr. 1, Neuroimaging Center, University Medical Center Mainz, 55131 Mainz, Germany. E-mail: \href{mailto:hcp4715@gmail.com}{\nolinkurl{hcp4715@gmail.com}}}

\abstract{
Perception is not just a passive process but actively constructing. There are rich literature on different factors that impact the processing efficiency of our perceptual system. One prominent example is the self-prioritization. However, the self is multifaceted and may include different aspsects and the moral self-concept pop out as the most importance one for people's identity. To examine whether the morally positive self facilitate the perceptual decision-making, we used a social associative paradigm (Sui et al., 2012). Through 11 experiments, we found that there was an robust moral-self prioritization effect, which may drives self-priortization effect when the self was not explicitly separated into different aspects. These results was consistent with previous studies that good-self is core representation of the self and people's motivation to maitain a positive-self may influence perceptual process.These results suggested a positive self-bias in perception. That is, people literally ``see'' the positive side of themselves better.


}

\begin{document}
\maketitle

\hypertarget{general-methods}{%
\section{General Methods}\label{general-methods}}

\hypertarget{participants.}{%
\subsection{Participants.}\label{participants.}}

All experiments (1a \textasciitilde{} 6b, except experiment 3b) reported in the current study were first finished between 2014 to 2016 in Tsinghua University, Beijing, participants of these experiments were recruited in Tsinghua University community. To increase the power by adding more data so that each experiment has 50 or more valid data (Simmons, Nelson, \& Simonsohn, 2013) , we recruited additional participants in Wenzhou University, Wenzhou, China in 2017. Due to the limited time and resources, additional data were collected for experiment 1a, 1b, 4a, and 4b, but not other experiments. We report all the variables measured in our experiments.

\hypertarget{material-and-procedure}{%
\subsection{Material and Procedure}\label{material-and-procedure}}

In the current study, we used the social associative learning paradigm (or self-tagging paradigm)(Sui, He, \& Humphreys, 2012), in which participants first learn the associations between geometric shapes and labels of person with different moral valence (e.g., in first three studies, the triangle, square, and circle and good person, neutral person, and bad person, respectively). The associations of the shapes and label were counterbalanced across participants. After learning phase, participants finish a practice phase to familiar with the task, in which they viewed one of the shapes upon the fixation while one of the labels below the fixation and judged whether the shape and the label were matched. When participants can get 60\% or higher accuracy at the end of the practicing session, they can start the experimental task which is the same as in the practice phase.

If not noted, E-prime 2.0 was used in all experiments. For participants recruited in Tsinghua University, they finished the experiment individually in a dim-lighted chamber, stimuli were presented on 22-inch CRT monitors, with a chin-rest brace. The visual angle of geometric shapes was about 3.7º × 3.7º, the finxation cross is of (0.8º × 0.8º of visual angle) at the center of the screen. The words were of 3.6º × 1.6º visual angle. The distance between the center of the shape or the word and the fixation cross was 3.5º of visual angle. Participant fixed their head on a chin-fixation, about 60 cm from the screen.

For participants recruited in Wenzhou University, they finished the experiment in a group consist of 3 \textasciitilde{} 12 participants in a dim-lighted testing room. Participants were required to finished the whole experiment independently. Also, they were instructed to start the experiment at the same time, so that the distraction between participants were minimized. The stimuli were presented on 19-inch CRT monitor. The visual angles are could not be exactly controlled because participants's chin were not fixed.

In most of these experiments, participant were also asked to fill a battery of questionnaire after they finish the cognitive tasks. More specificially, \ldots{}\ldots{}. All the questionnaire data are open (see, Liu et al., 2019)

\hypertarget{data-analysis}{%
\subsection{Data analysis}\label{data-analysis}}

We reported all the measurements, analysis and results in all the experiments in the current study. All data were first pre-processed using R R (Version 3.6.1; R Core Team, 2018) and the R-packages \emph{afex} (Version 0.23.0; Singmann, Bolker, Westfall, \& Aust, 2019), \emph{BayesFactor} (Version 0.9.12.4.2; Morey \& Rouder, 2018), \emph{boot} (Version 1.3.22; Davison \& Hinkley, 1997; Gerlanc \& Kirby, 2015), \emph{bootES} (Version 1.2; Gerlanc \& Kirby, 2015), \emph{coda} (Version 0.19.3; Plummer, Best, Cowles, \& Vines, 2006), \emph{corrplot2017} (Wei \& Simko, 2017), \emph{dplyr} (Version 0.8.0.1; Wickham et al., 2019), \emph{emmeans} (Version 1.3.4; Lenth, 2019), \emph{forcats} (Version 0.4.0; Wickham, 2019a), \emph{Formula} (Version 1.2.3; Zeileis \& Croissant, 2010), \emph{ggformula} (Version 0.9.1; Kaplan \& Pruim, 2019), \emph{ggplot2} (Version 3.2.0; Wickham, 2016), \emph{ggstance} (Version 0.3.1; Henry, Wickham, \& Chang, 2018), \emph{ggstatsplot} (Version 0.0.10; Patil \& Powell, 2018), \emph{here} (Version 0.1; Müller, 2017), \emph{Hmisc} (Version 4.2.0; Harrell Jr, Charles Dupont, \& others., 2019), \emph{lattice} (Version 0.20.38; Sarkar, 2008), \emph{lme4} (Version 1.1.21; Bates, Mächler, Bolker, \& Walker, 2015), \emph{MASS} (Version 7.3.51.4; Venables \& Ripley, 2002), \emph{Matrix} (Version 1.2.17; Bates \& Maechler, 2019), \emph{MBESS} (Version 4.4.3; Kelley, 2018), \emph{mosaic} (Version 1.5.0; Pruim, Kaplan, \& Horton, 2017, 2018), \emph{mosaicData} (Version 0.17.0; Pruim et al., 2018), \emph{multcomp} (Version 1.4.10; Hothorn, Bretz, \& Westfall, 2008), \emph{mvtnorm} (Version 1.0.11; Genz \& Bretz, 2009), \emph{papaja} (Version 0.1.0.9842; Aust \& Barth, 2018), \emph{plyr} (Version 1.8.4; Wickham et al., 2019; Wickham, 2011), \emph{psych} (Version 1.8.12; Revelle, 2018), \emph{purrr} (Version 0.3.2; Henry \& Wickham, 2019), \emph{RColorBrewer} (Version 1.1.2; Neuwirth, 2014), \emph{readr} (Version 1.3.1; Wickham, Hester, \& Francois, 2018), \emph{reshape2} (Version 1.4.3; Wickham, 2007), \emph{stringr} (Version 1.4.0; Wickham, 2019b), \emph{survival} (Version 2.44.1.1; Terry M. Therneau \& Patricia M. Grambsch, 2000), \emph{TH.data} (Version 1.0.10; Hothorn, 2019), \emph{tibble} (Version 2.1.3; Müller \& Wickham, 2019), \emph{tidyr} (Version 0.8.3; Wickham \& Henry, 2019), and \emph{tidyverse} (Version 1.2.1; Wickham, 2017). The clean data were analyzed using JASP (0.8.6.0, www.jasp-stats.org, (Love et al., 2019)). Participants whose overall accuracy lower than 60\% were excluded from analysis. Also, the accurate responses with less than 200ms reaction times were excluded from the analysis.

We analyzed accuracy performance using a signal detection approach, as in Sui et al.~(2012). The performance in each match condition was combined with that in the nonmatching condition with the same shape to form a measure of d'. Trials without response were coded either as \enquote{miss} (matched trials) or \enquote{false alarm} (mismatched trials). The d' were then analyzed using repeated measures analyses of variance (repeated measures ANOVA).

The reaction times of accurate trials were also analyzed using repeated measures ANOVA. To control the false positive when conducting the post-hoc comparisons, we used Bonferroni correction. Please note that in the first two experiment (experiment 1a and 1b), we included the variable matchness (matched vs.~mismatched) in our ANOVA of reaction times and then examine matched trials and mismatched trials separately when the interaction between matchness and other variables are significant. In both experiments, we found significant interaction between matchness and valence. Then, as previous study, we focused on the matched trial for the rest of the experiment (Sui et al., 2012).

We reported the effect size of repeated measures ANOVA (omega squared) (Bakeman, 2005; Lakens, 2013). Also, we reported Cohen's d and its 95\% confidence intervals for the post-hoc comparisons. To provide more information about the results, we also reported the Bayes Factor using JASP (Hu, Kong, Wagenmakers, Ly, \& Peng, 2018; Wagenmakers et al., 2018). The Bayes factor is the ratio of the probability of the current data pattern under alternative hypothesis (H1) and the probability of the current data pattern under null hypothesis (H0), which index the relative evidence for these two hypotheses from the current data. The BF10 represents the evidence for alternative hypothesis (H1) vs.~evidence for null hypothesis (H0); in contrast, BF01 represents that evidence for null hypothesis over the evidence for althernative hypothesis. We used the default prior in JASP for all the Bayes Factor analyses, and used Jeffreys (1961)'s convention for the strength of evidence: the BF10 \textgreater{} 3 means there are some evidence for H1 as compared with H0, BF10 great or equal to 10 means strong evidence for H1.

To assess the individual difference, we explored correlation between self-reported psychological distance and more objective responses bias (i.e., reaction times and d prime). To do this, we first normalized the personal distance by taking the percentage of the mean distance between each two persons in the sum of all 6 distances (self-good, self-normal, self-bad, good-normal, good-bad, normal-bad), and then calculated the bias score (indexed by the differences between good-normal, good-bad). Also, as exploratory analysis, we analyzed the correlation between behavioral response and moral identity, self-esteem, if data are available. As recent study showed that small size leads to unstable correlation estimates (Schönbrodt \& Perugini, 2013), we only reported the correlation based on data pooled from all experiments, while the results of each experiment were reported in supplementary results.

\hypertarget{experiment-1a}{%
\section{Experiment 1a}\label{experiment-1a}}

\hypertarget{methods}{%
\subsection{Methods}\label{methods}}

\hypertarget{participants}{%
\subsubsection{Participants}\label{participants}}

57 college students (38 female, age = 20.75 \(\pm\) 2.54 years) participated. 39 of them were recruited from Tsinghua University community in 2014; 18 were recruited from Wenzhou University in 2017. All participants were right-handed except one, and all had normal or corrected-to-normal vision. Informed consent was obtained from all participants prior to the experiment according to procedures approved by the local ethics committees. 6 participant's data were excluded from analysis because nearly random level of accuracy, leaving 51 participants (34 female, age = 20.72 \(\pm\) 2.44 years).

\hypertarget{stimuli-and-tasks}{%
\subsubsection{Stimuli and Tasks}\label{stimuli-and-tasks}}

Three geometric shapes were used in this experiment: triangle, square, and circle. These shapes were paired with three labels (bad person, good person or neutral person). The pairs were counterbalanced across participants.

\hypertarget{procedure}{%
\subsubsection{Procedure}\label{procedure}}

As we describe in general method part, this experiment had two phases. First, there was a learning stage. Participants were asked to learn the relationship between geometric shapes (triangle, square, and circle) and different person (bad person, a good person, or a neutral person). For example, a participant was told, \enquote{bad person is a circle; good person is a triangle; and a neutral person is represented by a square.} After participant remember the associations (usually in a few minutes), participants started a practicing phase of matching task which has the exact task as in the experimental task.
In the experimental task, participants judged whether shape--label pairs, which were subsequently presented, were correct. Each trial started with the presentation of a central fixation cross for 500 ms. Subsequently, a pairing of a shape and label (good person, bad person, and neutral person) was presented for 100 ms. The pair presented could confirm to the verbal instruction for each pairing given in the training stage, or it could be a recombination of a shape with a different label, with the shape--label pairings being generated at random. The next frame showed a blank for 1100ms. Participants were expected to judge whether the shape was correctly assigned to the person by pressing one of the two response buttons as quickly and accurately as possible within this timeframe (to encourage immediate responding). Feedback (correct or incorrect) was given on the screen for 500 ms at the end of each trial, if no response detected, \enquote{too slow} was presented to remind participants to accelerate. Participants were informed of their overall accuracy at the end of each block. The practice phase finished and the experimental task began after the overall performance of accuracy during practice phase achieved 60\%.
For pariticpants from the Tsinghua community, they completed 6 experimental blocks of 60 trials. Thus, there were 60 trials in each condition (bad-person matched, bad-person nonmatching, good-person matched, good-person nonmatching, neutral-person matched, and neutral-person nonmatching). For the participants from Wenzhou Univeristy, they finished 6 blocks of 120 trials, therefore, 120 trials for each condition.

\hypertarget{questionnaires}{%
\subsubsection{Questionnaires}\label{questionnaires}}

After the experiment, part of the participants in Tsinghua University also finished psychological distance, trait social justice (Bai, 2013), cognitive reflection test (Frederick, 2005), and disgust senstivity (Tan, Cong, \& Lu, 2007). The psychological distance measurement finished by indicating the the psychological distance between self, good person, bad person and neutral person, through two points on a horizontal line. This procedure is presented by Matlab. This method had been proven been an effective way to measure the psychological distance (Enock, Sui, Hewstone, \& Humphreys, 2018).
For all participants from Wenzhou University, they finished following questionnaires online immediately after the experiment: objective and subjective socioeconomic status (the objective SES measured by parents' education and occupation (Shi \& Shen, 2007), the subjective SES measured by ladder task (Ostrove, Adler, Kuppermann, \& Washington, 2000)), psychological distance (Enock et al., 2018), sensitivity to justice (Wu et al., 2014), cognitive reflection test (Frederick, 2005), disgust senstivity scale (Tan et al., 2007), belief in just world (short) (Wu et al., 2011), a short version of big five personality (John \& Srivastava, 1999), trait self-esteem (Rosenberg, 1965), locus of control (Levenson,1981), Free will and determinism plus (FAD+) (translated version) (Liu, Jian, Hu, \& Peng, 2015; Paulhus \& Carey, 2010), moral identity (Aquino \& Reed II, 2002), and moral self image (translated version) (Jordan, Leliveld, \& Tenbrunsel, 2015). Only the psychological distance data were analyized in the current study.

\hypertarget{data-analysis-1}{%
\subsubsection{Data analysis}\label{data-analysis-1}}

As we describe in the general method section.

\hypertarget{results}{%
\subsection{Results}\label{results}}

\begin{figure}
\centering
\includegraphics{Notebook_Pos_Self_Salience_APA_files/figure-latex/ex1a-dprime-rt-1.pdf}
\caption{\label{fig:ex1a-dprime-rt}RT and \emph{d} prime of Experiment 1a.}
\end{figure}

\hypertarget{d-prime}{%
\subsubsection{d prime}\label{d-prime}}

Figure \ref{fig:ex1a-dprime-rt} shows \emph{d} prime and reaction times during the perceptual matching task. We conducted a single factor (valence: good, neutral, bad) repeated measure ANOVA.

We found the effect of Valence (\(F(2, 100) = 6.19\), \(\mathit{MSE} = 0.26\), \(p = .003\), \(\hat{\eta}^2_G = .020\)). The post-hoc comparison with multiple comparison correction revealed that the shapes associated with Good-person (2.11, SE = 0.14) has greater \emph{d} prime than shapes associated with Bad-person (1.75, SE = 0.14), \emph{t}(50) = 3.304, \emph{p} = 0.0049. The Good-person condition was also greater than the Netural-person condition (1.95, SE = 0.16), but didn't reach statical significant, \emph{t}(50) = 1.54, \emph{p} = 0.28. Neither the Neutral-person conditiona is significantly greater than the Bad-person conidition, \emph{t}(50) = 2.109, \emph{p} = .098.

\hypertarget{reaction-time}{%
\subsubsection{Reaction time}\label{reaction-time}}

We conducted 2 (Matchness: Match v. Mismatch) by 3 (Valence: good, neutral, bad) repeated measure ANOVA. We found the main effect of Matchness (\(F(1, 50) = 232.39\), \(\mathit{MSE} = 948.92\), \(p < .001\), \(\hat{\eta}^2_G = .104\)), main effect of valence (\(F(1.87, 93.31) = 9.62\), \(\mathit{MSE} = 1,673.86\), \(p < .001\), \(\hat{\eta}^2_G = .016\)), and intercation between Matchness and Valence (\(F(1.73, 86.65) = 8.52\), \(\mathit{MSE} = 1,441.75\), \(p = .001\), \(\hat{\eta}^2_G = .011\)).

We then carried out two separate ANOVA for Match and Mismatched trials. For matched trials, we found the effect of valence . We further examined the effect of valence for both self and other for mached trials. We found that shapes associated with Good Person (684 ms, SE = 11.5) responded faster than Neutral (709 ms, SE = 11.5), \emph{t}(50) = -2.265, \emph{p} = 0.0702) and Bad Person (728 ms, SE = 11.7), \emph{t}(50) = -4.41, \emph{p} = 0.0002), and the Neutral condition was faster than the Bad condition, \emph{t}(50) = -2.495, \emph{p} = 0.0415). For non-matched trials, there was no significant effect of Valence ().

\hypertarget{experiment-1b}{%
\section{Experiment 1b}\label{experiment-1b}}

In this study, we aimed at excluding the potential confouding factor of the familarity of words we used in experiment 1a, by matching the familiarity of the words.

\hypertarget{method}{%
\subsection{Method}\label{method}}

\hypertarget{participants-1}{%
\subsubsection{Participants}\label{participants-1}}

72 college students (49 female, age = 20.17 \(\pm\) 2.08 years) participated. 39 of them were recruited from Tsinghua University community in 2014; 33 were recruited from Wenzhou University in 2017. All participants were right-handed except one, and all had normal or corrected-to-normal vision. Informed consent was obtained from all participants prior to the experiment according to procedures approved by the local ethics committees. 20 participant's data were excluded from analysis because nearly random level of accuracy, leaving 52 participants (36 female, age = 20.25 \(\pm\) 2.31 years).

\hypertarget{stimuli-and-tasks-1}{%
\subsubsection{Stimuli and Tasks}\label{stimuli-and-tasks-1}}

Three geometric shapes (triangle, square, and circle, with 3.7º × 3.7º of visual angle) were presented above a white fixation cross subtending 0.8º × 0.8º of visual angle at the center of the screen. The three shapes were randomly assigned to three labels with different moral valence: a morally bad person (\enquote{恶人}, ERen), a morally good person (\enquote{善人}, ShanRen) or a morally neutral person (\enquote{常人}, ChangRen). The order of the associations between shapes and labels was counterbalanced across participants.
Three labels used in this experiment is selected based on the rating results from an independent survey, in which participants rated the familiarity, frequency, and concreteness of eight different words online. Of the eight words, three of them are morally positive (HaoRen, ShanRen, Junzi), two of them are morally neutral (ChangRen, FanRen), and three of them are morally negative (HuaiRen, ERen, LiuMang). An independent sample consist of 35 participants (22 females, age 20.6 ± 3.11) were recruited to rate these words. Based on the ratings (see supplementary materials Figure S1), we selected ShanRen, ChangRen, and ERen to represent morally positve, neutral, and negative person.

\hypertarget{procedure-1}{%
\subsubsection{Procedure}\label{procedure-1}}

For participants from both Tsinghua community and Wenzhou community, the procedure in the current study was exactly same as in experiment 1a. For participants in Tsinghua community, they finished a survey suite include personal distance, objective and subjective SES, belief in just world (Wu et al., 2011), disgust senstivity scale (谭永红 et al., 2007), trait justice (Wu et al., 2014), and cognitive reflection test (Frederick, 2005). For participants from Wenzhou community, they finished exactly the same questionnaires as the participants from Wenzhou University in experiment 1a.

\hypertarget{data-analysis-2}{%
\subsection{Data Analysis}\label{data-analysis-2}}

Data was analyzed as in experiment 1a.

\hypertarget{results-1}{%
\subsection{Results}\label{results-1}}

\begin{figure}
\centering
\includegraphics{Notebook_Pos_Self_Salience_APA_files/figure-latex/ex1b-dprime-rt-1.pdf}
\caption{\label{fig:ex1b-dprime-rt}RT and \emph{d} prime of Experiment 1b.}
\end{figure}

Figure \ref{fig:ex1b-dprime-rt} shows \emph{d} prime and reaction times of experiment 1b.

\hypertarget{d-prime-1}{%
\subsubsection{\texorpdfstring{\emph{d} prime}{d prime}}\label{d-prime-1}}

Repeated measures ANOVA revealed main effect of valence, \(F(2, 102) = 14.98\), \(\mathit{MSE} = 0.16\), \(p < .001\), \(\hat{\eta}^2_G = .053\). Paired t test showed that the Good-Person condition (1.87 \(\pm\) 0.102) was with greater \emph{d} prime than Netural condition (1.44 \(\pm\) 0.101, \emph{t}(51) = 5.945, \emph{p} \textless{} 0.001). We also found that the Bad-Person condition (1.67 \(\pm\) 0.11) has also greater \emph{d} prime than neutral condition , \emph{t}(51) = 3.132, \emph{p} = 0.008). There Good-person condition was also slightly greater than the bad conidition, \emph{t}(51) = 2.265, \emph{p} = 0.0701.

\hypertarget{reaction-time-1}{%
\subsubsection{Reaction time}\label{reaction-time-1}}

We found intercation between Matchness and Valence (\(F(1.95, 99.31) = 19.71\), \(\mathit{MSE} = 960.92\), \(p < .001\), \(\hat{\eta}^2_G = .031\)) and then analyzed the matched trials and mismatched trials separately, as in experiment 1a. For matched trials, we found the effect of valence \(F(1.94, 99.1) = 33.97\), \(\mathit{MSE} = 1,343.19\), \(p < .001\), \(\hat{\eta}^2_G = .115\). Post-hoc \emph{t}-tests revealed that shapes associated with Good Person (684 \(\pm\) 8.77) were responded faster than Neutral-Person (740 \(\pm\) 9.84), (\emph{t}(51) = -8.167, \emph{p} \textless{} 0.001) and Bad Person (728 \(\pm\) 9.15), \emph{t}(51) = -5.724, \emph{p} \textless{} 0.0001). While there was no significant differences between Neutral and Bad-Person condition (\emph{t}(51) = 1.686, \emph{p} = 0.221). For non-matched trials, there was no significant effect of Valence (\(F(1.9, 97.13) = 1.80\), \(\mathit{MSE} = 430.15\), \(p = .173\), \(\hat{\eta}^2_G = .003\)).

\hypertarget{discussion}{%
\subsection{Discussion}\label{discussion}}

These results confirmed the facilitation effect of positive moral valence on the perceptual matching task. This pattern of results mimic prior results demonstrating self-bias effect on perceptual matching (Sui et al., 2012) and in line with previous studies that indirect learning of other's moral reputation do have influence on our subsequence behavior (Fouragnan et al., 2013).

\hypertarget{experiment-1c}{%
\section{Experiment 1c}\label{experiment-1c}}

In this study, we further control the valence of words using in our experiment. Instead of using label with moral valence, we used valence-neutral names in China. Participant first learn behaviors of the different person, then, they associate the names and shapes. And then they perform a name-shape matching task.

\hypertarget{method-1}{%
\subsection{Method}\label{method-1}}

\hypertarget{participants-2}{%
\subsubsection{Participants}\label{participants-2}}

23 college students (15 female, age = 22.61 \(\pm\) 2.62 years) participated. All of them were recruited from Tsinghua University community in 2014. Informed consent was obtained from all participants prior to the experiment according to procedures approved by the local ethics committees. No participant was excluded because they overall accuracy were above 0.6.

\hypertarget{stimuli-and-tasks-2}{%
\subsubsection{Stimuli and Tasks}\label{stimuli-and-tasks-2}}

Three geometric shapes (triangle, square, and circle, with 3.7º × 3.7º of visual angle) were presented above a white fixation cross subtending 0.8º × 0.8º of visual angle at the center of the screen. The three most common names were chosen, which are neutral in moral valence before the manipulation.
Three names (Zhang, Wang, Li) were first paired with three paragraphs of behavioral description. Each description includes one sentence of biographic infomration and four sentences that describing the moral behavioral under that name. To assess the that these three descriptions represented good, neutral, and bad valence, we collected the ratings of three person on six dimensions: morality, likability, trustworthiness, dominance, competence, and aggressiviess, from an independent sample (n = 34, 18 female, age = 19.6 ± 2.05). The rating results showed that the person with morally good behavioral description has higher score on morality (M = 3.59, SD = 0.66) than neutral (M = 0.88, SD = 1.1), \emph{t}(33) = 12.94, \emph{p} \textless{} .001, and bad conditions (M = -3.4, SD = 1.1), \emph{t}(33) = 30.78, \emph{p} \textless{} .001. Neutral condition was also significant higher than bad conditions \emph{t}(33) = 13.9, \emph{p} \textless{} .001 (See supplementary materials).

\hypertarget{procedure-2}{%
\subsubsection{Procedure}\label{procedure-2}}

After arriving the lab, participants were informed to complete two experimental tasks, first a social memory task to remember three person and their behaviors, after tested for their memory, they will finish a perceptual matching task.
In the social memory task, the descriptions of three person were presented without time limitation. Participant self-paced to memorized the behaviors of each person. After they memorizing, a recognition task was used to test their memory effect. Each participant was required to have over 95\% accuracy before preceding to matching task.
The perceptual learning task was followed, three names were randomly paired with geometric shapes. Participants were required to learn the association and perform a practicing task before they start the formal experimental blocks. They kept practicing util they reached 70\% accuracy. Then, they would start the perceptual matching task as in experiment 1a. They finished 6 blocks of perceptual matching trials, each have 120 trials.

\hypertarget{data-analysis-3}{%
\subsection{Data Analysis}\label{data-analysis-3}}

Data was analyzed as in experiment 1a.

\hypertarget{results-2}{%
\subsection{Results}\label{results-2}}

\begin{figure}
\centering
\includegraphics{Notebook_Pos_Self_Salience_APA_files/figure-latex/ex1c-dprime-rt-1.pdf}
\caption{\label{fig:ex1c-dprime-rt}RT and \emph{d} prime of Experiment 1c.}
\end{figure}

Figure \ref{fig:ex1c-dprime-rt} shows \emph{d} prime and reaction times of experiment 1c. We conducted same analysis as in Experiment 1a. Our analysis didn't should effect of valence on \emph{d} prime, \(F(2, 44) = 0.23\), \(\mathit{MSE} = 0.40\), \(p = .798\), \(\hat{\eta}^2_G = .005\). Neither the effect of valence on RT (\(F(1.63, 35.81) = 0.22\), \(\mathit{MSE} = 2,212.71\), \(p = .761\), \(\hat{\eta}^2_G = .001\)) or interaction between valence and matchness on RT (\(F(1.79, 39.43) = 1.20\), \(\mathit{MSE} = 1,973.91\), \(p = .308\), \(\hat{\eta}^2_G = .005\)).

\hypertarget{discussion-1}{%
\subsection{Discussion}\label{discussion-1}}

Experiment 1c was conducted in a old way, i.e., we peeked the data when we collected around 20 participants, and decided to stop because the non-signficant results. (move this experiment to supplementary?)

\hypertarget{experiment-2-sequential-presenting}{%
\section{Experiment 2: Sequential presenting}\label{experiment-2-sequential-presenting}}

Experiment 2 was conducted for two purpose: (1) to further confirm the facilitation effect of positive moral associations; (2) to test the effect of expectation of occurrence of each pair. In this experiment, after participant learned the assocation between labels and shapes, they were presented a label first and then a shape, they then asked to judge whether the shape matched the label or not (see (Sui, Sun, Peng, \& Humphreys, 2014). Previous studies showed that when the labels presented before the shapes, participants formed expectations about the shape, and therefore a top-down process were introduced into the perceptual matching processing. If the facilitation effect of postive moral valence we found in experiment 1 was mainly drive by top-down processes, this sequential presenting paradigm may eliminate or attenuate this effect; if, however, the facilitation effect ocured because of button-up processes, then, similar facilitation effect will appear even with sequential presenting paradigm.

\hypertarget{method-2}{%
\subsection{Method}\label{method-2}}

\hypertarget{participants-3}{%
\subsubsection{Participants}\label{participants-3}}

35 participants (17 female, age = 21.66 \(\pm\) 3.03) were recruited. 24 of them had participated in Experiment 1a (9 male, mean age = 21.9, s.d. = 2.9), and the time gap between these experiment 1a and experiment 2 is at least six weeks. The results of 1 participants were excluded from analysis because of less than 60\% overall accuracy, remains 34 participants (17 female, age = 21.74 \(\pm\) 3.04).

\hypertarget{procedure-3}{%
\subsubsection{Procedure}\label{procedure-3}}

In Experiment 2, the sequential presenting makes the matching task much easier than experiment 1. To avoid ceiling effect on behavioral data, we did a few pilot experiments to get optimal parameters, i.e., the conditions under which participant have similar accuracy as in Experiment 1 (around 70 \textasciitilde{} 80\% accuracy).
In the final procedure, the label (good person, bad person, or neutral person) was presented for 50 ms and then masked by a scrambled image for 200 ms. A geometric shape followed the scrambled mask for 50 ms in a noisy background (which was produced by first decomposing a square with ¾ gray area and ¼ white area to small squares with a size of 2 × 2 pixels and then re-combine these small pieces randomly), instead of pure gray background in Experiment 1. After that, a blank screen was presented 1100 ms, during which participants should press a button to indicate the label and the shape match the original association or not. Feedback was given, as in study 1. The next trial then started after 700 \textasciitilde{} 1100 ms blank. Other aspects of study 2 were identical to study 1.

\hypertarget{analysis}{%
\subsubsection{Analysis}\label{analysis}}

Data was analyzed as in study 1a.

\hypertarget{results-3}{%
\subsection{Results}\label{results-3}}

\begin{figure}
\centering
\includegraphics{Notebook_Pos_Self_Salience_APA_files/figure-latex/ex2-dprime-rt-1.pdf}
\caption{\label{fig:ex2-dprime-rt}RT and \emph{d} prime of Experiment 2.}
\end{figure}

Figure \ref{fig:ex2-dprime-rt} shows \emph{d} prime and reaction times of experiment 2. Less than 0.2\% correct trials with less than 200ms reaction times were exlucded.

\hypertarget{d-prime.}{%
\subsubsection{\texorpdfstring{\emph{d} prime.}{d prime.}}\label{d-prime.}}

There was evidence for the main effect of valence, \(F(2, 66) = 14.41\), \(\mathit{MSE} = 0.21\), \(p < .001\), \(\hat{\eta}^2_G = .066\). Paired t test showed that the Good-Person condition (2.79 \(\pm\) 0.17) was with greater \emph{d} prime than Netural condition (2.21 \(\pm\) 0.16, \emph{t}(33) = 4.723, \emph{p} = 0.001) and Bad-person condition (2.41 \(\pm\) 0.14), \emph{t}(33) = 4.067, \emph{p} = 0.008). There was no-significant difference between Neutral-person and Bad-person conidition, \emph{t}(33) = -1,802, \emph{p} = 0.185.

\hypertarget{reaction-time-2}{%
\subsubsection{Reaction time}\label{reaction-time-2}}

The results of reaction times of matchness trials showed similiar pattern as the \emph{d} prime data.

We found intercation between Matchness and Valence (\(F(1.99, 65.7) = 9.53\), \(\mathit{MSE} = 605.36\), \(p < .001\), \(\hat{\eta}^2_G = .017\)) and then analyzed the matched trials and mismatched trials separately, as in experiment 1a. For matched trials, we found the effect of valence \(F(1.99, 65.76) = 10.57\), \(\mathit{MSE} = 1,192.65\), \(p < .001\), \(\hat{\eta}^2_G = .067\). Post-hoc \emph{t}-tests revealed that shapes associated with Good Person (548 \(\pm\) 9.4) were responded faster than Neutral-Person (582 \(\pm\) 10.9), (\emph{t}(33) = -3.95, \emph{p} = 0.0011) and Bad Person (582 \(\pm\) 10.2), \emph{t}(33) = -3.9, \emph{p} = 0.0013). While there was no significant differences between Neutral and Bad-Person condition (\emph{t}(33) = -0.01, \emph{p} = 0.999). For non-matched trials, there was no significant effect of Valence (\(F(1.99, 65.83) = 0.17\), \(\mathit{MSE} = 489.80\), \(p = .843\), \(\hat{\eta}^2_G = .001\)).

\hypertarget{discussion-2}{%
\subsection{Discussion}\label{discussion-2}}

In this experiment, we repeated the results pattern that the positive moral valenced stimuli has an advantage over the neutral or the negative valenced association. Moreover, with a croass task analysis, we didn't found evidence that the experiment task interacted with moral valence, suggesting that the effect might not be effect by experiment task.
These findings suggested that the facilitation effect of positive moral valence is robust and not affected by task. This robust effect detected by the associative learning is unexpected.

\hypertarget{experiment-3a}{%
\section{Experiment 3a}\label{experiment-3a}}

To examine the modulation effect of positive valence was an intrinsic, self-referential process, we designed study 3. In this study, moral valence was assigned to both self and a stranger. We hypothesized that the modulation effect of moral valence will be stronger for the self than for a stranger.

\hypertarget{method-3}{%
\subsection{Method}\label{method-3}}

\hypertarget{participants-4}{%
\subsubsection{Participants}\label{participants-4}}

38 college students (15 female, age = 21.92 \(\pm\) 2.16) participated in experiment 3a. All of them were right-handed, and all had normal or correted-to-normal vision. Informed consent was obtained from all participants prior to the experiment according to procedures approved by a local ethics committee. One female and one male student did not finish the experiment, and 1 participants' data were excluded from analysis because less than 60\% overall accuracy, remains 35 participants (13 female, age = 22.11 \(\pm\) 2.13).

\hypertarget{design}{%
\subsubsection{Design}\label{design}}

Study 3a combined moral valence with self-relevance, hence the experiment has a 2 × 3 × 2 within-subject design. The first variable was self-relevance, include two levels: self-relevance vs.~stranger-relevance; the second variable was moral valence, include good, neutral and bad; the third variable was the matching between shape and label: match vs.~mismatch.

\hypertarget{stimuli}{%
\subsubsection{Stimuli}\label{stimuli}}

The stimuli used in study 3a share the same parameters with experiment 1 \& 2. 6 shapes were included (triangle, square, circle, trapezoid, diamond, regular pentagon), as well as 6 labels (good self, neutral self, bad self, good person, bad person, neutral person). To match the concreteness of the label, we asked participant to chosen an unfamiliar name of their own gender to be the stranger.

\hypertarget{procedure-4}{%
\subsubsection{Procedure}\label{procedure-4}}

After being fully explained and signed the informed consent, participants were instructed to chose a name that can represent a stranger with same gender as the participant themselves, from a common Chinese name pool. Before experiment, the experimenter explained the meaning of each label to participants. For example, the \enquote{good self} mean the morally good side of themselves, them could imagine the moment when they do something's morally applauded, \enquote{bad self} means the morally bad side of themselves, they could also imagine the moment when they doing something morally wrong, and \enquote{neutral self} means the aspect of self that doesn't related to morality, they could imagine the moment when they doing something irrelevant to morality. In the same sense, the \enquote{good other}, \enquote{bad other}, and \enquote{neutral other} means the three different aspects of the stranger, whose name was chosen before the experiment. Then, the experiment proceeded as study 1a. Each participant finished 6 blocks, each have 120 trials. The sequence of trials was pseudo-randomized so that there are 10 matched trials for each condition and 10 non-matched trials for each condition (good self, neutral sef, bad self, good other, neutral other, bad other) for each block.

\hypertarget{data-analysis-4}{%
\subsubsection{Data Analysis}\label{data-analysis-4}}

Data analysis followed strategies described in the general method section. Reaction times and \emph{d} prime data were analyzed as in study 1 and study 2, except that one more within-subject variable (i.e., self-relevance) was included in the repeated measures ANOVA.

\hypertarget{results-4}{%
\subsection{Results}\label{results-4}}

\begin{figure}
\centering
\includegraphics{Notebook_Pos_Self_Salience_APA_files/figure-latex/ex3a-dprime-rt-1.pdf}
\caption{\label{fig:ex3a-dprime-rt}RT and \emph{d} prime of Experiment 3a.}
\end{figure}

Figure \ref{fig:ex3a-dprime-rt} shows \emph{d} prime and reaction times of experiment 3a. Less than 5\% correct trials with less than 200ms reaction times were exlucded.

\hypertarget{d-prime-2}{%
\subsubsection{\texorpdfstring{\emph{d} prime}{d prime}}\label{d-prime-2}}

There was evidence for the main effect of valence, \(F(2, 68) = 11.09\), \(\mathit{MSE} = 0.22\), \(p < .001\), \(\hat{\eta}^2_G = .039\), and main effect of self-relevance, \(F(1, 34) = 3.22\), \(\mathit{MSE} = 0.54\), \(p = .082\), \(\hat{\eta}^2_G = .015\), as well as the interaction, \(F(2, 68) = 3.39\), \(\mathit{MSE} = 0.38\), \(p = .039\), \(\hat{\eta}^2_G = .022\).

We then conducted separated ANOVA for self-referential and other-referential trials. The valence effect was shown for the self-referential conditions, \(F(2, 68) = 13.98\), \(\mathit{MSE} = 0.25\), \(p < .001\), \(\hat{\eta}^2_G = .119\). Post-hoc test revealed that the Good-Self condition (1.97 \(\pm\) 0.14) was with greater \emph{d} prime than Netural condition (1.41 \(\pm\) 0.12, \emph{t}(34) = 4.505, \emph{p} = 0.0002), and Bad-self condition (1.43 \(\pm\) 0.102), \emph{t}(34) = 3.856, \emph{p} = 0.0014. There was difference between neutral and bad conidition, \emph{t}(34) = -0.238, \emph{p} = 0.9694. However, no effect of valence was found for the other-referential condition \(F(2, 68) = 0.38\), \(\mathit{MSE} = 0.34\), \(p = .683\), \(\hat{\eta}^2_G = .004\).

\hypertarget{reaction-time-3}{%
\subsubsection{Reaction time}\label{reaction-time-3}}

We found intercation between Matchness and Valence (\(F(1.98, 67.44) = 26.29\), \(\mathit{MSE} = 730.09\), \(p < .001\), \(\hat{\eta}^2_G = .025\)) and then analyzed the matched trials and mismatched trials separately, as in previous experiments.

For the matched trials, we found that the interaction between identity and valence, \(F(1.72, 58.61) = 3.89\), \(\mathit{MSE} = 2,750.19\), \(p = .032\), \(\hat{\eta}^2_G = .019\), as well as the main effect of valence \(F(1.98, 67.34) = 35.76\), \(\mathit{MSE} = 1,127.25\), \(p < .001\), \(\hat{\eta}^2_G = .079\), but not the effect of identity \(F(1, 34) = 0.20\), \(\mathit{MSE} = 3,507.14\), \(p = .660\), \(\hat{\eta}^2_G = .001\). As for the \emph{d} prime, we separated analyzed the self-referential and other-referential trials. For the Self-referential trials, we found the main effect of valence, \(F(1.8, 61.09) = 30.39\), \(\mathit{MSE} = 1,584.53\), \(p < .001\), \(\hat{\eta}^2_G = .159\); for the other-referential trials, the effect of valence is weaker, \(F(1.86, 63.08) = 2.85\), \(\mathit{MSE} = 2,224.30\), \(p = .069\), \(\hat{\eta}^2_G = .024\). We then focused on the self conditions: the good-self condition (713 \(\pm\) 12) is faster than neutral- (776 \(\pm\) 11.8), \emph{t}(34) = -7.396, \emph{p} \textless{} .0001 , and bad-self (772 \(\pm\) 10.1) conditions, \emph{t}(34) = -5.66, \emph{p} \textless{} .0001. But there is not difference between neutral- and bad-self conditions, \emph{t}(34) = 0.481, \emph{p} = 0.881.

For the mismatched trials, we didn't found any strong effect: identity, \(F(1, 34) = 3.43\), \(\mathit{MSE} = 660.02\), \(p = .073\), \(\hat{\eta}^2_G = .004\), valence \(F(1.89, 64.33) = 0.40\), \(\mathit{MSE} = 444.10\), \(p = .661\), \(\hat{\eta}^2_G = .001\), or interaction between the two \(F(1.94, 66.02) = 2.42\), \(\mathit{MSE} = 817.35\), \(p = .099\), \(\hat{\eta}^2_G = .007\).

\hypertarget{experiment-3b}{%
\section{Experiment 3b}\label{experiment-3b}}

In study 3a, participants had to remember 6 pairs of association, which cause high cogitive load during the whole exepriment. To eliminate the influence of cognitive load, we conducted study 3b, in which participant learn three aspect of self and stranger seperately in to consecutive task. We hypothesize that we will replicate the pattern of study 3a, i.e., the effect of moral valence only occurs for self-relevant conditions.

\hypertarget{method-4}{%
\subsection{Method}\label{method-4}}

\hypertarget{participants-5}{%
\subsubsection{Participants}\label{participants-5}}

Study 3b were finished in 2017, at that time we have calculated that the effect size (Cohen's d) of good-person (or good-self) vs.~bad-person (or bad-other) was between 0.47 \textasciitilde{} 0.53, based on study 1a, 1b, 2, 3a, 4a, and 4b. Based on this effect size, we estimated that 54 participants would allow we to detect the effec size of Cohen's = 0.5 with 95\% power and alpha = 0.05, using G*power 3.192 (Faul, Erdfelder, Buchner, \& Lang, 2009; Faul, Erdfelder, Lang, \& Buchner, 2007). Therefore, we planned to stop after we arrived this number. During the data collected at Wenzhou University, 61 participants (45 females; 19 to 25 years of age, age = 20.42 \(\pm\) 1.77) came to the testing room and we tested all of them during a single day. All participants were right-handed, and all had normalneutral or corrected-to-normal vision. Informed consent was obtained from all participants prior to the experiment according to procedures approved by a local ethics committee. 4 participants' data were excluded from analysis because their over all accuracy was lower than 60\%, 1 more participant waw excluded because of zero hit rate for one condition, leaving 56 participants (43 females; 19 to 25 years old, age = 20.27 \(\pm\) 1.60).

\hypertarget{design-1}{%
\subsubsection{Design}\label{design-1}}

Study 3b has the same experimental design as 3a, with a 2× 3× 2 within-subject design. The first variable was self-relevance, include two levels: self-relevant vs.~stranger-relevant; the second variable was moral valence, include good, neutral and bad; the third variable was the matching between shape and label: match vs.~mismatch.
Stimuli. The stimuli used in study 3b share the same parameters with experiment 3a. 6 shapes were included (triangle, square, circle, trapezoid, diamond, regular pentagon), as well as 6 labels, but the labels changed to \enquote{good self}, \enquote{neutral self}, \enquote{bad self}, \enquote{good him/her}, bad him/her'', \enquote{neutral him/her}, the stranger's label is consistent with participants' gender. Same as study 3a, we asked participant to chosen an unfamiliar name of their own gender to be the stranger before showing them the relationship. Note, because of implementing error, the personal distance data didn't collect for this experiment.

\hypertarget{procedure-5}{%
\subsubsection{Procedure}\label{procedure-5}}

In this experiment, participants finished two matching tasks, i.e., self-matching task, and other-matching task. In the self-matching task, participants first associate the three aspects of self to three different shapes, and then perform the matching task. In the other-matching task, participants first associate the three aspects of the stranger to three different shapes, and then perform the matching task. The order of self-task and other-task are counter-balanced among participants.
Different from experiment 3a, after presenting the stimuli pair for 100ms, participant has 1900 ms to response, and they were feedbacked with both accuracy and reaction time.
As in study 3a, before each task, the intruction showed the meaning of each label to participants. The self-matching task and other-matching task were randomized between participants. Each participant finished 6 blocks, each have 120 trials.

\hypertarget{data-analysis-5}{%
\subsubsection{Data Analysis}\label{data-analysis-5}}

Data analysis is the same as study 3a.

\hypertarget{results-5}{%
\subsection{Results}\label{results-5}}

\begin{figure}
\centering
\includegraphics{Notebook_Pos_Self_Salience_APA_files/figure-latex/ex3b-dprime-rt-1.pdf}
\caption{\label{fig:ex3b-dprime-rt}RT and \emph{d} prime of Experiment 3b.}
\end{figure}

Figure \ref{fig:ex3b-dprime-rt} shows \emph{d} prime and reaction times of experiment 3b. Less than 5\% correct trials with less than 200ms reaction times were exlucded.

\hypertarget{d-prime-3}{%
\subsubsection{\texorpdfstring{\emph{d} prime}{d prime}}\label{d-prime-3}}

There was evidence for the main effect of valence, \(F(2, 110) = 5.29\), \(\mathit{MSE} = 0.26\), \(p = .006\), \(\hat{\eta}^2_G = .011\), and main effect of self-relevance, \(F(1, 55) = 5.07\), \(\mathit{MSE} = 0.88\), \(p = .028\), \(\hat{\eta}^2_G = .019\), as well as the intercation, \(F(2, 110) = 16.18\), \(\mathit{MSE} = 0.24\), \(p < .001\), \(\hat{\eta}^2_G = .033\). Therefore we conducted repeated measure ANOVA for self and other conditions separately. We found the valence effect for self-referential condition (\(F(2, 110) = 18.32\), \(\mathit{MSE} = 0.24\), \(p < .001\), \(\hat{\eta}^2_G = .090\)) and other-referential condition (\(F(2, 110) = 3.26\), \(\mathit{MSE} = 0.26\), \(p = .042\), \(\hat{\eta}^2_G = .011\)). Post-hoc comparison for the self-referential conditions revealed that \emph{d}' was larger for good self (2.23 \(\pm\) 0.1087) than for bad self (1.66 \(\pm\) 0.098), \emph{t}(55) = 6.11, \emph{p} \textless{} 0.0001, Cohen's \emph{d} = 0.817, 95\% CI {[}0.511 1.117{]} , BF10 = 8.43, and neutral self (1.91 \(\pm\) 0.088), \emph{t}(55) = 3.03, \emph{p} = 0.0104, Cohen's \emph{d} = 0.404, 95\%CI {[}0.13 0.675{]} , BF10 = 1.33e+5. There was also higher d' for neutral-self condition than bad-self conditions, \emph{t}(55) = 3.02, \emph{p} = 0.0106, Cohen's \emph{d} = 0.403, 95\% CI {[}0.129 0.674{]}, BF10 = 8.22. For other-referential conditions, good-other (2.03 \(\pm\) 0.12) was smaller than neutral-other (2.28 \(\pm\) 0.133), \emph{t}(55) = -2.262, \emph{p} = 0.0699, but not the bad-other condition (2.19 \(\pm\) 0.122), \emph{t}(55) = -1.559, \emph{p} = 0.272. Neither evidence for the neutral and bad-other condition \emph{t}(55) = 1.11, \emph{p} = 0.51.

We also tested the effect of personal association by comparing \emph{d'} values for difference moral valence level. The results showed that the bad-self association condition was responded worse than the bad-other association condition, \emph{t}(55) = -4.1, \emph{p} \textless{} 0.001, Cohen's \emph{d} = -0.548, 95\% CI {[}-0.827 -0.265{]}, BF10 = 167. The neutral-self was also worse than the neutral-other, \emph{t}(55) = -3.15, \emph{p} = 0.0026 , Cohen's \emph{d} = -0.422, 95\% CI {[}-0.693 -0.146{]}, BF10 = 11.7. While the good-self association condition and good-stranger conditions are not differ from each other, \emph{t}(55) = 1.394, \emph{p} = 0.169, Cohen's \emph{d} = 0.186, 95\% CI{[}-0.079 0.449{]}, BF10 = 0.364.

\hypertarget{reaction-time-4}{%
\subsubsection{Reaction time}\label{reaction-time-4}}

The results of reaction times of matchness trials showed similiar pattern as the \emph{d} prime data. We found three-way intercation between matchness, and valence (\(F(1.99, 109.63) = 55.67\), \(\mathit{MSE} = 4,203.96\), \(p < .001\), \(\hat{\eta}^2_G = .032\)) and then analyzed the matched trials and mismatched trials separately, as in previous experiments.

For the matched trials, we found that interaction between of valence and identity, \(F(1.67, 92.11) = 6.14\), \(\mathit{MSE} = 6,472.48\), \(p = .005\), \(\hat{\eta}^2_G = .009\), BF10 = 1.17. Then, the matched trials were analyzed for the self-relevance and other-relevance pairs separately. The results showed a significant effect of moral valence for the self condition, \(F(1.66, 91.38) = 23.98\), \(\mathit{MSE} = 6,965.61\), \(p < .001\), \(\hat{\eta}^2_G = .100\), BF10 = 4.54e+6. Paired \emph{t}-tests showed that responses to the good-self association (817 ± 119) were faster than to bad-self associations (915 ± 132), \emph{t}(55) = -8.78, \emph{p} \textless{} 0.001, Cohen's \emph{d} = -1.173, 95\% CI {[}-1.511 -0.828{]}, BF10 = 1.84e+9, and to neutral-self association (880 ± 116), \emph{t}(55) = 3.748, \emph{p} \textless{} 0.0001, Cohen's \emph{d} = -0.501, 95\% CI {[}-0.777 -0.221{]}, BF10 = 58.7. The neutral-self was faster than the bad-self associations, \emph{t}(55) = -2.41, \emph{p} = 0.019 , Cohen's \emph{d} = -0.321, 95\% CI {[}-0.589 -0.051{]}, BF10 = 2.03.

The effect of moral valence was also significant for the other-relevance conditions, \(F(1.89, 103.94) = 5.96\), \(\mathit{MSE} = 5,589.90\), \(p = .004\), \(\hat{\eta}^2_G = .014\), BF10 = 8.55. the good-other condition (734 ± 158) didn't differ from neutral-other condition (735 ±160), t(55) = -0.07, p = 0.946, Cohen's \emph{d} = -0.009, 95\% CI {[}-0.271 -0.293{]}, BF10 = 0.15, but faster than the bad other condition (776 ± 173), \emph{t}(55) = -3.14, \emph{p} = 0.0027, Cohen's \emph{d} = -0.419, 95\% CI {[}-0.691 -0.144{]}, BF10 = 11.3. The neutral-other condition also faster than the bad-other condition, \emph{t}(55) = -3.232, \emph{p} = 0.0021, Cohen's \emph{d} = -0.432, 95\% CI {[}-0.704 -0.156{]}, BF10 = 14.3.

We also analyzed the effect of self-relevance on the different moral valence levels. The results showed that for all three different valence levels, there the self condition was responded slower than other condition: good-self vs.~good-stranger, \emph{t}(55) = 4.29, \emph{p} \textless{} 0.001, Cohen's \emph{d} = 0.573, 95\% CI {[}0.288 0.854{]}, BF10 = 297.2; neutral-self vs.~neutral -stranger, \emph{t}(55) = 7.17, \emph{p} \textless{} 0.001, Cohen's \emph{d} = 0.958, 95\% CI {[}0.638 1.272{]}), BF10 = 5.77e+6; bad-self vs.~bad-other, \emph{t}(55) = 6.03, \emph{p} \textless{} 0.001, Cohen's \emph{d} = 0.806, 95\% CI {[}0.5 1.11{]}, BF10 = 100208.03.

For the mismacthed trials, we also found interaction between of valence and identity, \(F(1.84, 100.94) = 35.20\), \(\mathit{MSE} = 2,879.80\), \(p < .001\), \(\hat{\eta}^2_G = .026\). Further analysis showed that there was effect of valence for other-referential trials (\(F(1.79, 98.63) = 62.42\), \(\mathit{MSE} = 3,489.92\), \(p < .001\), \(\hat{\eta}^2_G = .077\)) but not for the self-referential trials (\(F(1.84, 101.38) = 0.66\), \(\mathit{MSE} = 2,283.19\), \(p = .508\), \(\hat{\eta}^2_G = .001\)). Post-hoc comparison revealed that bad-other trials (857 \(\pm\) 20.9) were responded faster than good (967 \(\pm\) 25.1), \emph{t} = -9.625, \emph{p} \textless{} 0.0001, and neutral (876 \(\pm\) 21.4) trials, \emph{t} = -2.263, \emph{p} = 0.697, good-other conditions were also slower than neutral-other, \emph{t} = -8.016, \emph{p} \textless{} 0.0001

\hypertarget{discussion-3}{%
\subsection{Discussion}\label{discussion-3}}

In experiment 3b, we separated the self-referential and other-referential tasks into different blocks so that participants had lower cognitive load when finish the task. We replicated the pattern from experiment 3a that the valence effect is stronger when the stimuli were self-referential. Contrast to experiment 3a, however, we found that the self-referential condition is out-performed by other-referential conditions. This pattern suggest that the self-referential enhanced the valence effect, and the advantage to self might be relative instead of absolute.

\hypertarget{experiment-4a}{%
\section{Experiment 4a}\label{experiment-4a}}

In study 1-3 participants made explicit judgements about moral associations. In Experiment 4, we examined whether the interaction between moral valence and identity occur even when one of the variable was irrelevent to the task. In experiment 4a, participants learnt associations between shapes and self/other labels, then made perceptual match judgements only about the self or other conditions labels and shapes (cf.~Sui et al., 2012). However, we presented labels of different moral valence in the shapes, which means that the moral valence factor become task irrelevant. If the binding between moral good and self is intrinsic and automatic, then we will observe that facilitating effect of moral good for self conditions, but not for other conditions.

In study 4b, we changed the role of valence and identity in task. In this experiment, participants learn the association between moral valence and the made perceptual match judgements to associations between different moral valence and shapes as in study 1-3. Different from experiment 1 \textasciitilde{} 3, we made put the labels of \enquote{self/other} in the shapes so that identity served as an task irrelevant variable. As in experiment 4a, we also hypothesized that the instrinc binding between morally good and self will enhance the performance of good self condition, even identity is irrelevant to the task.

\hypertarget{methods-1}{%
\subsection{Methods}\label{methods-1}}

\hypertarget{participants-6}{%
\subsubsection{Participants}\label{participants-6}}

64 participants (37 female, age = 19.70 \(\pm\) 1.22) participated the current study, 32 of them were from Tsinghua Universtiy in 2015, 32 were from Wenzhou University parpticipated in 2017. All participants were right-handed, and all had normalneutral or corrected-to-normalneutral vision. Informed consent was obtained from all participants prior to the experiment according to procedures approved by a local ethics committee. The data from 5 participants from Wenzhou site were excluded from analysis because their accuracy was close to chance (\textless{} 0.6). The results for the remaining 59 participants (33 female, age = 19.78 \(\pm\) 1.20) were analyzed and reported.

\hypertarget{experimental-design}{%
\subsubsection{Experimental design}\label{experimental-design}}

As in Experiment 3, a 2× 3× 2 within-subject design was used. The first variable was self-relevance (self and stranger associations); the second variable was moral valence (good, normalneutral and bad associations); the third variable was the matching between shape and label (matching vs.~non-match for the personal association).
However, in this the task, participants only learn the association between two geometric shapes and two labels (self and other), i.e., only self-relevance were related to the task. The moral valence manipulation was achieved by embeding the personal label of the labels in the geometric shapes, see below. For simplicity, the trials where shapes where paired with self and with a word of \enquote{good person} inside were shorted as good-self condition, similarly, the trials where shapes paired with the self and with a word of \enquote{bad person} inside were shorted as bad-self condition. Hence, we also have six conditions: good-self, neutral-self, bad-self, good-other, neutral-other, and bad-other.

\hypertarget{stimuli-1}{%
\subsubsection{Stimuli}\label{stimuli-1}}

2 shapes were included (circle, square) and each appeared above a central fixation cross with the personal label appearing below. However, the shapes were not empty but with a two-Chinese-character word in the middle, the word was one of three labels with different moral valence: \enquote{good person}, \enquote{bad person} and \enquote{neutral person}. Before the experiment, participants learned the self/other association, and were informed to only response to the association between shapes' configure and the labels below the fixation, but ignore the words within shapes. Besides the behavioral experiments, participants from Tsinghua community also finished questionnaires as Experiments 3, and participants from Wenzhou community finished a series of questionnaire as the other experiment finished in Wenzhou.

\hypertarget{procedure-6}{%
\subsubsection{Procedure}\label{procedure-6}}

The procedure was similar to Experiment 1. There were 6 blocks of trial, each with 120 trials for 2017 data. Due to procedure error, the data collected in 2015 in Tsinghua community only have 60 trials for each block, i.e., 30 trials per condition.

\hypertarget{data-analysis-6}{%
\subsubsection{Data analysis}\label{data-analysis-6}}

The data were analyzed in the same way as in experiment 3a and 3b.

\hypertarget{results-6}{%
\subsection{Results}\label{results-6}}

\begin{figure}
\centering
\includegraphics{Notebook_Pos_Self_Salience_APA_files/figure-latex/ex4a-dprime-rt-1.pdf}
\caption{\label{fig:ex4a-dprime-rt}RT and \emph{d} prime of Experiment 4a.}
\end{figure}

Figure \ref{fig:ex4a-dprime-rt} shows \emph{d} prime and reaction times of experiment 4a.

\hypertarget{d-prime.-1}{%
\subsubsection{\texorpdfstring{\emph{d} prime.}{d prime.}}\label{d-prime.-1}}

We conducted 2 (Idenity: self v. other) by 3 (morlaity: good, neutral, bad) repeated measures ANOVA. The effect of identity (\(F(1, 58) = 121.04\), \(\mathit{MSE} = 0.48\), \(p < .001\), \(\hat{\eta}^2_G = .189\)), also the interaction between identity and valence, (\(F(2, 116) = 4.12\), \(\mathit{MSE} = 0.14\), \(p = .019\), \(\hat{\eta}^2_G = .004\)). But not the effect of valence was not found, \(F(2, 116) = 0.53\), \(\mathit{MSE} = 0.11\), \(p = .587\), \(\hat{\eta}^2_G = .000\). We further examined the effect of valence for both self and other. For the self-referential trials, there was a weak effect of valence \(F(2, 116) = 3.01\), \(\mathit{MSE} = 0.15\), \(p = .053\), \(\hat{\eta}^2_G = .008\). Post-hoc analysis showed that good self condition (2.55 \(\pm\) 0.111) is slightly higher than bad self condition (2.38 \(\pm\) 0.105) , \emph{t}(58) = 2.339, \emph{p} = .0583. But there was no differences between good-self and neutral self (2.45 \(\pm\) 0.101), \emph{t}(58) = 1.575, \emph{p} = .264, neither between neutral-self and bad-self, \emph{t}(58) = 0.966, \emph{p} = .601. As for the other-referential conditions, the result didn't show effect of valence \(F(2, 116) = 1.75\), \(\mathit{MSE} = 0.10\), \(p = .178\), \(\hat{\eta}^2_G = .003\).

\hypertarget{reaction-times}{%
\subsubsection{Reaction times}\label{reaction-times}}

We conducted 2 (Matchness: match v. mismatch) by 2 (Idenity: self v. other) by 3 (morlaity: good, neutral, bad) repeated measure ANOVA. There was a main effect of Matchness (\(F(1, 58) = 79.85\), \(\mathit{MSE} = 4,817.48\), \(p < .001\), \(\hat{\eta}^2_G = .123\)) and intercation between Matchness and Identity(\(F(1, 58) = 81.15\), \(\mathit{MSE} = 3,226.69\), \(p < .001\), \(\hat{\eta}^2_G = .087\))

We carried out two separate ANOVA for both matched and mismatched trials. The results showed that for matched trials, there was an interaction between valence and identity, \(F(1.9, 110.18) = 4.41\), \(\mathit{MSE} = 465.91\), \(p = .016\), \(\hat{\eta}^2_G = .003\). However, there is no main effect of valence (\(F(1.98, 114.82) = 0.94\), \(\mathit{MSE} = 606.30\), \(p = .392\), \(\hat{\eta}^2_G = .001\)). We futher broke down the interaction by analyzing the data for self and other pairs separately. There was a significant effect of moral valence for self-stimuli, \(F(1.97, 114.32) = 6.29\), \(\mathit{MSE} = 367.25\), \(p = .003\), \(\hat{\eta}^2_G = .006\), BF10 = 11.16. Paired \emph{t}-tests showed that good-self condition (654 ± 67) were faster relative to bad-self condition (665 ± 64.6), \emph{t}(58) = -3.47, \emph{p} = 0.0028, Cohen's \emph{d} = -0.451 CI {[}-0.718 -0.182{]}, BF10 = 27.0, and over neutral-self condition (664 ± 64), \emph{t}(58) = -2.78, \emph{p} = 0.013, Cohen's \emph{d} = -0.362, 95\% CI {[}-0.624 -0.097{]}, BF10 = 4.63. The neutral-self and bad-self conditionsdid not differ, \emph{t}(58) = -0.44, \emph{p} = 0.89, Cohen's \emph{d} = 0.0499, CI {[}-0.305 0.206{]}, BF10 = 0.153. For the stranger condition, the results showed that there was no difference among these conditions, \(F(1.95, 112.89) = 0.35\), \(\mathit{MSE} = 699.50\), \(p = .699\), \(\hat{\eta}^2_G = .001\), BF10 = 0.077.

For non-matched trials, there was no significant effect. Morality (\(F(1, 58) = 0.16\), \(\mathit{MSE} = 1,547.37\), \(p = .692\), \(\hat{\eta}^2_G = .000\)), Identity(\(F(1.96, 113.52) = 0.68\), \(\mathit{MSE} = 390.26\), \(p = .508\), \(\hat{\eta}^2_G = .000\)), interaction (\(F(1.9, 110.27) = 0.04\), \(\mathit{MSE} = 585.80\), \(p = .953\), \(\hat{\eta}^2_G = .000\)).

\hypertarget{experiment-4b}{%
\section{Experiment 4b}\label{experiment-4b}}

\hypertarget{method-5}{%
\subsection{Method}\label{method-5}}

\hypertarget{participants-7}{%
\subsubsection{Participants}\label{participants-7}}

53 college students (39 female, age = 20.57 \(\pm\) 1.81) participated the current study, 34 of them were from Tsinghua Universtiy in 2015 19 were from Wenzhou University parpticipated in 2017. All participants were right-handed, and all had normalneutral or corrected-to-normalneutral vision. Informed consent was obtained from all participants prior to the experiment according to procedures approved by a local ethics committee. The data from 8 participants were excluded from analysis because their accuracy was close to chance (\textless{} 0.6). The results for the remaining 45 participants (33 female, age = 20.78 \(\pm\) 1.76) were analyzed and reported.

\hypertarget{experimental-design-1}{%
\subsubsection{Experimental design}\label{experimental-design-1}}

The experimental design of this experiment is same as experiment 4a: a 3× 2 × 2 within-subject design with moral valence (good, normalneutral and bad associations), self-relatedness (self vs.~other), and matchness between shape and label (match vs.~mismatch for the personal association) as within-subject variables. However, in the current task, the participants learned the associations between three shapes and three labels with different moral valence: good-person, neutral-person, and bad-person. While the word \enquote{self} or \enquote{other} were presented in the shapes (see below).

\hypertarget{stimuli-2}{%
\subsubsection{Stimuli}\label{stimuli-2}}

In this task, 3 shapes were included (circle, square, and trapezoid) and were presented above a central fixation cross, as in previous experiments. Similar to experiment 4a, the shapes were not empty but with a two-Chinese-character word in the middle corresponding to the labels \enquote{self} and \enquote{other}. Before the experiment, we informed participants only response to the relationship between shapes'shapes configure and the labels below the fixation, ignoring the wordswithin each shape. Besides the behavioral experiments, participants also finished questionnaires as Experiments 1-3.

\hypertarget{procedure-7}{%
\subsubsection{Procedure}\label{procedure-7}}

The procedure was similar to Experiment 4 a. Both samples of participants finished 6 blocks of trial, each with 120 trials.

\hypertarget{data-analysis-7}{%
\subsubsection{Data analysis}\label{data-analysis-7}}

The data were analyzed as in experiment 4a.

\hypertarget{results-7}{%
\subsection{Results}\label{results-7}}

\begin{figure}
\centering
\includegraphics{Notebook_Pos_Self_Salience_APA_files/figure-latex/ex4b-dprime-rt-1.pdf}
\caption{\label{fig:ex4b-dprime-rt}RT and \emph{d} prime of Experiment 4b.}
\end{figure}

Figure \ref{fig:ex4b-dprime-rt} shows \emph{d} prime and reaction times of experiment 4b.

\hypertarget{d-prime-4}{%
\subsubsection{\texorpdfstring{\emph{d} prime}{d prime}}\label{d-prime-4}}

We conducted 2 (Idenity: self v. other) by 3 (morlaity: good, neutral, bad) repeated measure ANOVA. The results revealed no effect of valence (\(F(1.59, 69.94) = 2.34\), \(\mathit{MSE} = 0.48\), \(p = .115\), \(\hat{\eta}^2_G = .010\)), or identity, \(F(1, 44) = 0.00\), \(\mathit{MSE} = 0.08\), \(p = .994\), \(\hat{\eta}^2_G = .000\), or their interactions, \(F(1.96, 86.41) = 0.53\), \(\mathit{MSE} = 0.10\), \(p = .585\), \(\hat{\eta}^2_G = .001\).

\hypertarget{reaction-times-1}{%
\subsubsection{Reaction times}\label{reaction-times-1}}

We conducted 2 (Matchness: match v. mismatch) by 2 (Idenity: self v. other) by 3 (morlaity: good, neutral, bad) repeated measure ANOVA:

There was a main effect of Matchness (\(F(1, 44) = 210.69\), \(\mathit{MSE} = 1,969.60\), \(p < .001\), \(\hat{\eta}^2_G = .180\)), main effect of valence (\(F(1.95, 85.74) = 5.35\), \(\mathit{MSE} = 2,247.66\), \(p = .007\), \(\hat{\eta}^2_G = .012\)), intercation between Matchness and Valence(\(F(1.87, 82.5) = 18.58\), \(\mathit{MSE} = 1,291.12\), \(p < .001\), \(\hat{\eta}^2_G = .023\)), and three way interaction (\(F(1.95, 85.99) = 3.70\), \(\mathit{MSE} = 285.25\), \(p = .030\), \(\hat{\eta}^2_G = .001\)).

We futher broke down the interaction by analyzing the data for self and other pairs separately. There was a significant effect of moral valence for self-stimuli, \(F(1.74, 76.48) = 13.69\), \(\mathit{MSE} = 1,732.08\), \(p < .001\), \(\hat{\eta}^2_G = .079\), BF10 = 11.16. Paired \emph{t} tests showed that good-self association (680 ± 9.79) were faster than bad-self associations (721 ± 8.97), \emph{t}(44) = -4.22, \emph{p} \textless{} .001, Cohen's \emph{d} = -0.629 CI {[}-0.947 -0.306{]}, BF10 = 200, and neutral-self association (713 \(\pm\) 8.19), \emph{t}(44) = -4.67, \emph{p} \textless{} 0.001, Cohen's \emph{d} = -0.696, 95\% CI {[}-1.019 -0.367{]}, BF10 = 745.3. The neutral-self and bad-self associations did not differ, \emph{t}(44) = -1.04, \emph{p} = .31, Cohen's \emph{d} = -0.155, 95\%CI {[}-0.448 0.14{]}, BF10 = 0.267. RTs in good-self condition were facilitated but without performance being impaired for bad-self associations (relative to the normal neutral self)(see Figure 5).
For other-association condition, the main effect of moral valence was also significant, \(F(1.87, 82.44) = 7.09\), \(\mathit{MSE} = 1,527.43\), \(p = .002\), \(\hat{\eta}^2_G = .043\), BF10 = 21. The RT for good-other association condition (688 ± 66.9) is faster than the bad-other association condition (718 ± 49.7), t(44) = -3.353, p = 0.0017, Cohen's d = -0.4999, 95\%CI {[}-0.8075 -0.1872{]}, BF10 = 18.84. The RT for good-other condition is slightly faster than neutral-other condition (704 ± 57.1), but the evidence is not strong, t (44) = -2.21, p = 0.0324, Cohen's d = -0.3294, 95\%CI {[}-6278 -0.0275{]}, BF10 = 1.454. While there is is no strong evidence about the differences between bad-other vs.~neutral-other conditions, t(44) = -1.8267, p = 0.0745, Cohen's d = -0.2723, 95\%CI {[}-0.5685 0.0268{]}, BF10 = 0.743.

We also comparied the reaction times for self- and other- association in different valence condition. we found that Good-self condition is faster than good-other condition, \emph{t}(44) = -2.165, \emph{p} = 0.0358; neutral self is slower than neutral-other condition, \emph{t}(44) = 3.064, \emph{p} = 0.0037. Bad-self and bad-other did not show difference, \emph{t}(44) = 0.623, \emph{p} = 0.5363

For non-matched trials, there was no significant effect. Idneity (\(F(1, 44) = 1.96\), \(\mathit{MSE} = 319.47\), \(p = .169\), \(\hat{\eta}^2_G = .001\)), interaction (\(F(1.88, 82.57) = 0.31\), \(\mathit{MSE} = 316.96\), \(p = .718\), \(\hat{\eta}^2_G = .000\)). But here are effect of Valence (\(F(1.69, 74.54) = 6.59\), \(\mathit{MSE} = 886.19\), \(p = .004\), \(\hat{\eta}^2_G = .010\))

\hypertarget{discussion-4}{%
\subsection{Discussion}\label{discussion-4}}

In experiment 4, we manipulated the task so that the moral valence (experiment 4a) or the self-relatedness (experiment 4b) become irrelevant to the task. We found robust effects of the tasks: when the self-relatedness is task related, the results showed a strong effect of self-relatedness; in contrast, when moral valence become task related the main effect of moral valence was strong. However, the task irrelevant stimuli in the shape also had influence on the performance. The good self conditions (the shape associated the self and with a \enquote{good person} within the shape) performed better than bad self conditions even when the self was the only task relevant stimuli. Also, good-self showed advantage over good-other when valence is the only task relevant variable while idenity was not. Together, these results suggest that moral valence and self-referential can still couple together and facilitated the perceptual decision making even one feature of them is implicit.

\hypertarget{experiment-5-generalization-of-positive-effect}{%
\section{Experiment 5: Generalization of positive effect}\label{experiment-5-generalization-of-positive-effect}}

So far, we have considered the modulation effect of morality and found that the positive moral valence could enhance the perception. However, we still not sure whether this effect was moral specific or reflecting a more general mechanism of effect of positive valence. To test the specificity of morality, we conducted experiment 5, in which three more categories of stimuli were used (people of different attractiveness, scene of diffenent attractivness, and emotional words with different valence). In this study, participants finished 4 session of association task, each with different categories of stimuli.

\hypertarget{method-6}{%
\subsection{Method}\label{method-6}}

\hypertarget{participants-8}{%
\subsubsection{Participants}\label{participants-8}}

43 participant recruited from Tsinghua University university community (21 females; age = 22.47 \(\pm\) 2.48). All participants were right-handed, and all had normal or corrected-to-normal vision. Informed consent was obtained from all participants prior to the experiment according to procedures approved by the local ethics committee. The data from 5 participants were excluded from analysis, 1 participant didn't finished the experiment, and the other 4 were exclued because of the overall accuracy was less than 60\%. The results for the remaining 38 subjects (18 female, age = 22.32 \(\pm\) 2.41) were included in data analyses.

\hypertarget{experimental-design-2}{%
\subsubsection{Experimental design}\label{experimental-design-2}}

A 4 × 3 × 2 within-subject design was used. The first independent variable was stimuli categories (morality, atttractiveness of people, attractiveness of scene, and emotional words); the second independent variables is valence (positive, neutral and negative); the third variable was the matching between shape and label (match vs.~mismatch for the association). The task was to learn the association between each geometric shape and the self/other label.

\hypertarget{stimuli-3}{%
\subsubsection{Stimuli}\label{stimuli-3}}

4 sets of shapes were included (three circle, three rectangle, three kind of triangle, and three kinds of quadrangle), each set of shape were paired with one category of label, counter-balanced across subjects. Besides the behavioral experiments, participants also finished questionnaires XXXXXXXX.

\hypertarget{procedure-8}{%
\subsubsection{Procedure}\label{procedure-8}}

Participants finish 4 session of experiment, and each include one experiment as in experiment 1. And the order of each category was randomnized for each participants. Each session started with a practice, and proceed to formal experiment when reached over 60\% accuracy. Each session included 6 blocks of trial, each with 120 trials.

\hypertarget{results-8}{%
\subsection{Results}\label{results-8}}

\begin{figure}
\centering
\includegraphics{Notebook_Pos_Self_Salience_APA_files/figure-latex/ex5-dprime-rt-1.pdf}
\caption{\label{fig:ex5-dprime-rt}RT and \emph{d} prime of Experiment 5.}
\end{figure}

Figure \ref{fig:ex5-dprime-rt} shows \emph{d} prime and reaction times of experiment 5.

\hypertarget{d-prime-5}{%
\subsubsection{\texorpdfstring{\emph{d} prime}{d prime}}\label{d-prime-5}}

We conducted 4 (task type: morality, emotion, person, scene) by 3 (morlaity: good, neutral, bad) repeated measure ANOVA. The results revealed no effect of task type (\(F(2.95, 108.97) = 1.18\), \(\mathit{MSE} = 0.76\), \(p = .321\), \(\hat{\eta}^2_G = .006\)), but revealed effect of valence, \(F(1.85, 68.49) = 29.70\), \(\mathit{MSE} = 0.29\), \(p < .001\), \(\hat{\eta}^2_G = .034\), and their interactions, \(F(4.68, 173.09) = 5.85\), \(\mathit{MSE} = 0.31\), \(p < .001\), \(\hat{\eta}^2_G = .019\).

We then conducted ANOVA separately for four different tasks. For the morality task, the valence effect \(F(1.96, 72.66) = 2.22\), \(\mathit{MSE} = 0.21\), \(p = .116\), \(\hat{\eta}^2_G = .008\).

For emotion conditions, we found that main effect of valence \(F(1.69, 62.65) = 5.28\), \(\mathit{MSE} = 0.30\), \(p = .011\), \(\hat{\eta}^2_G = .030\). Post-hoc comparison showed that good (2.13, s.e. = 0.161) was not different from neutral condition (2.10, se = 0.125), \emph{t}(37) = 0.163, \emph{p} = 0.9854, but both are higher than than bad condition (1.79, se = 0.148) (neutral \textgreater{} bad, \emph{t}(37) = 3.588, \emph{p} = 0.0027; good \textgreater{} bad, \emph{t}(37) = 2.62, \emph{p} = 0.0332).

For the person appearance, the main effect of valence is significant, \(F(1.73, 64.02) = 27.22\), \(\mathit{MSE} = 0.26\), \(p < .001\), \(\hat{\eta}^2_G = .088\). Post-hoc analysis found that good condition (2.4, se = 0.167) was higher than both neutral (1.71, se = 0.175, \emph{t}(37) = 5.482, \emph{p} \textless{} .0001) and bad (1.70, se = 0.182, \emph{t}(37) = 6.365, \emph{p} \textless{} .0001), but neutral and bad condition are not different (\emph{t}(37) = 0.197, \emph{p} =0.9788).

For scene appearance, the main effect of valence is significnat \(F(1.79, 66.4) = 14.02\), \(\mathit{MSE} = 0.34\), \(p < .001\), \(\hat{\eta}^2_G = .072\). Post-hoc analysis revealed the same pattern as in person task: good condition (2.23, se = 0.156) is higher than both neutral (1.57, se = 0.178, \emph{t}(37) =4.683, \emph{p} = 0.0001) and bad (1.77, se = 0.148, \emph{t}(37) = 3.414, \emph{p} = 0.0044), but neutral and bad conditions are not different (\emph{t}(37) = -1.893, \emph{p} = 0.1549)

\hypertarget{reaction-time-5}{%
\subsubsection{Reaction time}\label{reaction-time-5}}

As in previous experiment, we focused our analysis on the matched trials. We conducted 4 (task type: morality, emotion, beauty of person, beauty of scene) by 3 (Valence: good, neutral, bad) repeated measure ANOVA for matched trials only.

We din't found the main effect of task type (\(F(2.06, 76.12) = 0.56\), \(\mathit{MSE} = 7,041.60\), \(p = .577\), \(\hat{\eta}^2_G = .004\)), but the main effect of valence (\(F(1.82, 67.23) = 53.98\), \(\mathit{MSE} = 1,911.31\), \(p < .001\), \(\hat{\eta}^2_G = .076\)), and intercation between Matchness and Valence (\(F(3.58, 132.31) = 3.12\), \(\mathit{MSE} = 2,663.50\), \(p = .021\), \(\hat{\eta}^2_G = .013\)). We then analyze the effect of valence for each task type. We found that for all four task, the valence effect was significant: morality (\(F(1.89, 69.98) = 8.66\), \(\mathit{MSE} = 1,875.36\), \(p = .001\), \(\hat{\eta}^2_G = .036\)), emotion (\(F(1.48, 54.68) = 9.74\), \(\mathit{MSE} = 2,664.24\), \(p = .001\), \(\hat{\eta}^2_G = .072\)), person (\(F(1.78, 65.78) = 39.39\), \(\mathit{MSE} = 1,310.51\), \(p < .001\), \(\hat{\eta}^2_G = .162\)), scene (\(F(1.75, 64.69) = 17.71\), \(\mathit{MSE} = 1,820.73\), \(p < .001\), \(\hat{\eta}^2_G = .102\)). Post-hoc analyses revealed that for emotion task, the good condition is reacted faster than bad condition (\emph{t}(37) = 3.475, \emph{p} = 0.0037) but not neutral conditions are not different ((\emph{t}(37) = -0.77, \emph{p} = 0.7236)). The bad condiiton is longer than the neutral (\emph{t}(37) = 5.09, \emph{p} \textless{} 0.0001). The pattern is different for morality, person, and scence tasks, which showed that good is faster than both neutral ((\emph{ts}(37) = {[}-7.289 -2.4{]}, \emph{ps} \textless{} 0.0548)) and bad (\emph{ts}(37) = {[}-7.232 -3.817{]}, \emph{ps} \textless{} 0.0014).

\hypertarget{discussion-5}{%
\subsection{Discussion}\label{discussion-5}}

Morality is not specific but reflected an general positive effect. However, this positive effect might not same as the emotion effect.

\hypertarget{experiment-6a-eeg-study-1}{%
\section{Experiment 6a: EEG study 1}\label{experiment-6a-eeg-study-1}}

Experiment 6a was conducted to study the neural correlates of the positive prioritization effect. The behavioral paradigm is same as experiment 2.

\hypertarget{method-7}{%
\subsection{Method}\label{method-7}}

\hypertarget{participants-9}{%
\subsubsection{Participants}\label{participants-9}}

25 college students (8 female, age = 22.72 \(\pm\) 2.84) participated the current study, all of them were from Tsinghua Universtiy in 2014. Informed consent was obtained from all participants prior to the experiment according to procedures approved by a local ethics committee. No participant was excluded from behavioral analysis.

\hypertarget{experimental-design-3}{%
\subsubsection{Experimental design}\label{experimental-design-3}}

The experimental design of this experiment is same as experiment 2: a 3 × 2 within-subject design with moral valence (good, neutral and bad associations) and matchness between shape and label (match vs.~mismatch for the personal association) as within-subject variables.

\hypertarget{stimuli-4}{%
\subsubsection{Stimuli}\label{stimuli-4}}

Three geometric shapes (triangle, square and circle, each 4.6º × 4.6º of visual angle) were presented at the center of screen for 50 ms after 500ms of fixation (0.8º × 0.8º of visual angle). The association of the three shapes to bad person (\enquote{坏人, HuaiRen}), good person (\enquote{好人, HaoRen}) or ordinary person (\enquote{常人, ChangRen}) was counterbalanced across participants. The words bad person, good person or ordinary person (3.6º × 1.6º) was also displayed at the center fo the screen. Participants had to judge whether the pairings of label and shape matched (e.g., Does the circle represent a bad person?). The experiment was run on a PC using E-prime software (version 2.0). These stimuli were displayed on a 22-in CRT monitor (1024×768 at 100Hz).
We used backward masking to avoid over-processing of the moral words, in which a scrabmled picture were presented for 900 ms after the label. Also, to avoid the celling effect on accruacy, shapes were presented on a noisy background based on our pilot studies. The noisy images were made by scrambling a picutre of 3/4gray and ¼ white at resoluation of 2 × 2 pixel.

\hypertarget{procedure-9}{%
\subsubsection{Procedure}\label{procedure-9}}

The procedure was similar to Experiment 2. Participants finished 9 blocks of trial, each with 120 trials. In total, participants finished 180 trials for each combination of condition.

As in experiment 2 (Sui, He, \& Humphreys, 2012), subjects first learned the associations between labels and shapes and then completed a shape-label matching task (e.g., good person-triangle). In each trial of the matching task, a fixation were first presented for 500 ms, followed by a 50 ms label; then, a scramled picture presented 900 ms. After the backward mask, the shape were presented on a noisy background for 50ms. Participant have to response in 1000ms after the presentation of the shape, and finnally, a feedback screen was presented for 500 ms (see figure 1). The inter-trial interval (ITI) were randomly varied at the range of 1000 \textasciitilde{} 1400 ms.

All the stimuli were presented on a gray background (RGB: 127, 127, 127). E-primed 2.0 was used to present stimuli and collect behavioral results. Data were collected and analyzed when accuracy performance in total reached 60\%.

\hypertarget{results-9}{%
\subsection{Results}\label{results-9}}

\begin{figure}
\centering
\includegraphics{Notebook_Pos_Self_Salience_APA_files/figure-latex/ex6a-dprime-rt-1.pdf}
\caption{\label{fig:ex6a-dprime-rt}RT and \emph{d} prime of Experiment 6a.}
\end{figure}

Only the behavioral results were reported here. Figure \ref{fig:ex6a-dprime-rt} shows \emph{d} prime and reaction times of experiment 6a.

\hypertarget{d-prime-6}{%
\subsubsection{\texorpdfstring{\emph{d} prime}{d prime}}\label{d-prime-6}}

We conducted repeated measures ANOVA, with moral valence as independent variable. The results revealed the main effect of valence (\(F(1.73, 41.45) = 4.63\), \(\mathit{MSE} = 0.10\), \(p = .019\), \(\hat{\eta}^2_G = .025\)). Post-hoc anlaysis revealed that shapes link with Good person (mean = 3.13, SE = 0.109) is greater than Neutral condition (mean = 2.88, SE = 0.14),\emph{t} = 2.916, \emph{df} = 24, \emph{p} = 0.02, p-value adjusted by Tukey method, but the \emph{d} prime between Good and bad (mean = 3.03, SE = 0.142) (\emph{t} = 1.512, \emph{df} = 24, \emph{p} = 0.3034, p-value adjusted by Tukey method), bad and neutral (\emph{t} = 1.599, \emph{df} = 24, \emph{p} = 0.2655, p-value adjusted by Tukey method) were not siginificant.

\hypertarget{analaysis-of-reaction-time.}{%
\subsubsection{Analaysis of reaction time.}\label{analaysis-of-reaction-time.}}

The results of reaction times of matchness trials showed similiar pattern as the \emph{d} prime data.

We found intercation between Matchness and Valence (\(F(1.96, 47.14) = 21.88\), \(\mathit{MSE} = 434.43\), \(p < .001\), \(\hat{\eta}^2_G = .021\)) and then analyzed the matched trials and mismatched trials separately, as in experiment 2. For matched trials, we found the effect of valence \(F(1.98, 47.44) = 34.68\), \(\mathit{MSE} = 511.80\), \(p < .001\), \(\hat{\eta}^2_G = .080\). For non-matched trials, there was no significant effect of Valence (\(F(1.78, 42.78) = 0.42\), \(\mathit{MSE} = 234.71\), \(p = .638\), \(\hat{\eta}^2_G = .000\)). Post-hoc \emph{t}-tests revealed that shapes associated with Good Person (mean = 550, SE = 13.8) were responded faster than Neutral-Person (501, SE = 14.7), (\emph{t}(24) = -5.171, \emph{p} = 0.0001) and Bad Person (523, SE = 16.3), \emph{t}(24) = -8.137, \emph{p} \textless{} 0.0001)., and Neutral is faster than Bad-Person condition (\emph{t}(32) = -3.282, \emph{p} = 0.0085).

\hypertarget{experiment-6b-eeg-study-2}{%
\section{Experiment 6b: EEG study 2}\label{experiment-6b-eeg-study-2}}

Experiment 6b was conducted to study the neural correlates of the prioritization effect of positive self, i.e., the neural underlying of the behavioral effect found int experiment 3a. However, as in experiment 5a, the procedure of this experiment was modified to adopted to ERP experiment.

\hypertarget{method-8}{%
\subsection{Method}\label{method-8}}

\hypertarget{participants-10}{%
\subsubsection{Participants}\label{participants-10}}

23 college students (8 female, age = 22.86 \(\pm\) 2.47) participated the current study, all of them were recruited from Tsinghua Universtiy in 2016. Informed consent was obtained from all participants prior to the experiment according to procedures approved by a local ethics committee. For day 1's data, 1 participant was excluded from the current analysis because of lower than 60\% overall accuracy, remaining 22 participants (8 female, age = 22.76 \(\pm\) 2.49). For day 2's data, one participant dropped out, leaving 22 participants (9 female, age = 23.05 \(\pm\) 2.46), all of them has overall accuracy higher than 60\%.

\hypertarget{experimental-design-4}{%
\subsubsection{Experimental design}\label{experimental-design-4}}

The experimental design of this experiment is same as experiment 3: a 2 × 3 × 2 within-subject design with self-relevance (self-relevant vs.~other-relevant), moral valence (good, neutral, and bad) and matchness between shape and label (match vs.~mismatch) as within-subject variables.

\hypertarget{stimuli-5}{%
\subsubsection{Stimuli}\label{stimuli-5}}

As in experiment 3a, 6 shapes were included (triangle, square, circle, trapezoid, diamond, regular pentagon), as well as 6 labels (good self, neutral self, bad self, good person, bad person, neutral person). To match the concreteness of the label, we asked participant to chosen an unfamiliar name of their own gender to be the stranger.

\hypertarget{procedure-10}{%
\subsubsection{Procedure}\label{procedure-10}}

The procedure was similar to Experiment 2 and 6a. Subjects first learned the associations between labels and shapes and then completed a shape-label matching task. In each trial of the matching task, a fixation were first presented for 500 ms, followed by a 50 ms label; then, a scramled picture presented 900 ms. After the backward mask, the shape were presented on a noisy background for 50ms. Participant have to response in 1000ms after the presentation of the shape, and finnally, a feedback screen was presented for 500 ms (see figure 1). The inter-trial interval (ITI) were randomly varied at the range of 1000 \textasciitilde{} 1400 ms.

All the stimuli were presented on a gray background (RGB: 127, 127, 127). E-primed 2.0 was used to present stimuli and collect behavioral results. Data were collected and analyzed when accuracy performance in total reached 60\%.

Because learning 6 associations was more difficult than 3 associations and participant might have low accuracy (see experiment 3a), the current study had extended to a two-day paradigm to maximizing the accurate trials that can be used in EEG data. At the first day, participants learnt the associations and finished 9 blocks of the matching task, each had 120 trials, without EEG recording. That is, each condition has 90 trials.

Participants came back to lab at the second day and finish the same task again, with EEG recorded. Before the EEG experiment, each participant finished a practice session again, if their accuracy is equal or higher than 85\%, they start the experiment (one participant used lower threshold 75\%). Each participant finished 18 blocks, each has 90 trials. One participant finished additional 6 blocks because of high error rate at the beginning, another two participant finished addition 3 blocks because of the technique failure in recording the EEG data. To increase the number of trials that can be used for EEG data analysis, matched trials has twice number as mismatched trials, therefore, for matched trials each participants finished 180 trials for each condition, for mismatched trials, each conditions has 90 trials.

\hypertarget{results-10}{%
\subsection{Results}\label{results-10}}

\begin{figure}
\centering
\includegraphics{Notebook_Pos_Self_Salience_APA_files/figure-latex/ex6b-d1-dprime-rt-1.pdf}
\caption{\label{fig:ex6b-d1-dprime-rt}RT and \emph{d} prime of Experiment 6a.}
\end{figure}

Only the behavioral results were reported here.

\hypertarget{day-one}{%
\subsubsection{Day one}\label{day-one}}

Figure \ref{fig:ex6b-d1-dprime-rt} shows \emph{d} prime and reaction times from day 1 of the experiment 6b.

\hypertarget{d-prime-7}{%
\paragraph{\texorpdfstring{\emph{d} prime}{d prime}}\label{d-prime-7}}

There was evidence for the interaction between identity and valence, \(F(2, 42) = 12.88\), \(\mathit{MSE} = 0.13\), \(p < .001\), \(\hat{\eta}^2_G = .041\). We further split the self- and other-relevant trials. For the self trials, there was significant effect of valence, \(F(2, 42) = 29.31\), \(\mathit{MSE} = 0.12\), \(p < .001\), \(\hat{\eta}^2_G = .147\). Post-hoc comparison showed that the good-self condition (2.71, SE = 0.199) is better than both neutral-self (1.98, SE = 0.151), \emph{t}(21) = 5.984, \emph{p} \textless{} 0.001, and bad-self condition (2.07, SE = 0.154), \emph{t}(21) = 6.555, \emph{p} \textless{} 0.001. But there was no significant difference between bad-self and neutral-self, \emph{t}(21) = -1.059, \emph{p} = 0.549. For other trials, there was no significant effect of valuence, \(F(2, 42) = 0.00\), \(\mathit{MSE} = 0.16\), \(p = .999\), \(\hat{\eta}^2_G = .000\).

\hypertarget{rt}{%
\paragraph{RT}\label{rt}}

For the matched trials, there was interaction between identity and valence, \(F(1.72, 36.16) = 4.55\), \(\mathit{MSE} = 1,560.90\), \(p = .022\), \(\hat{\eta}^2_G = .015\). We split the self-relevant and other relevant trials separately. For the self condition, the valence effect is significant, \(F(1.92, 40.38) = 14.48\), \(\mathit{MSE} = 1,647.20\), \(p < .001\), \(\hat{\eta}^2_G = .112\). The Self-good (484, SE = 13.2) is faster than self-neutral (543, SE = 16.7) , \emph{t} = -4.521, \emph{p} = 0.0005, \emph{df} = 21 and self-bad condition (535, SE = 18.4), \emph{t} = -4.489, \emph{p} = 0.0006, \emph{df} = 21. but not significant different between neutral and bad condition, \emph{t} = 0.689, \emph{p} = 0.772, \emph{df} = 21. For other condition, there was no effect of valence, \(F(1.79, 37.5) = 1.04\), \(\mathit{MSE} = 1,842.07\), \(p = .356\), \(\hat{\eta}^2_G = .008\).

\begin{figure}
\centering
\includegraphics{Notebook_Pos_Self_Salience_APA_files/figure-latex/ex6b-d2-dprime-rt-1.pdf}
\caption{\label{fig:ex6b-d2-dprime-rt}RT and \emph{d} prime of Experiment 6b.}
\end{figure}

\hypertarget{day-two}{%
\subsubsection{Day two}\label{day-two}}

Figure \ref{fig:ex6b-d2-dprime-rt} shows \emph{d} prime and reaction times from day 2 of the experiment 6b.

\hypertarget{d-prime-8}{%
\paragraph{\texorpdfstring{\emph{d} prime}{d prime}}\label{d-prime-8}}

There was evidence for the interaction between identity and valence, \(F(2, 42) = 3.86\), \(\mathit{MSE} = 0.08\), \(p = .029\), \(\hat{\eta}^2_G = .005\). We further split the self- and other-relevant trials. For the self trials, there was significant effect of valence, \(F(2, 42) = 7.35\), \(\mathit{MSE} = 0.08\), \(p = .002\), \(\hat{\eta}^2_G = .021\). Post-hoc comparison showed that the good-self condition (2.71, SE = 0.214) is better than both neutral-self (2.43, SE = 0.175), \emph{t}(21) = 2.98, \emph{p} = 0.0189, and bad-self condition (2.43, SE = 0.199), \emph{t}(21) = 3.93, \emph{p} = 0.0021. But there was no significant difference between bad-self and neutral-self, \emph{t}(21) = -0.097, \emph{p} = 0.995. For other trials, there was no significant effect of valuence, \(F(2, 42) = 1.46\), \(\mathit{MSE} = 0.09\), \(p = .245\), \(\hat{\eta}^2_G = .004\).

\hypertarget{rt-1}{%
\paragraph{RT}\label{rt-1}}

For the matched trials, the interaction between identity and valence, \(F(1.62, 34.1) = 3.04\), \(\mathit{MSE} = 978.35\), \(p = .071\), \(\hat{\eta}^2_G = .005\). As in previous studies, we splited the self- and other-relevant trials. For the self condition, the valence effect is significant, \(F(1.46, 30.76) = 6.57\), \(\mathit{MSE} = 1,007.62\), \(p = .008\), \(\hat{\eta}^2_G = .023\). The Self-good (480, SE = 16.9) is faster than self-neutral (504, SE = 17.3) , \emph{t} = -2.289, \emph{p} = 0.0795, \emph{df} = 21 and self-bad condition (508, SE = 17.9), \emph{t} = -4.342, \emph{p} = 0.0008, \emph{df} = 21. but not significant different between neutral and bad condition, \emph{t} = -0.503, \emph{p} = 0.871, \emph{df} = 21. For other condition, there was no effect of valence, \(F(1.91, 40.04) = 1.75\), \(\mathit{MSE} = 1,070.90\), \(p = .188\), \(\hat{\eta}^2_G = .007\).

\newpage

\hypertarget{meta-analysis-of-the-effect-size}{%
\section{Meta-analysis of the effect size}\label{meta-analysis-of-the-effect-size}}

To get a better estimation of the effect in the current study, we combined the data of the 11 experiments described above and 2 experiments from another study (Hu, Lan, Macrae, \& Sui, 2019) by conducting a mini-meta-analysis (Goh, Hall, \& Rosenthal, 2016). More specifically, we conducted random effect model meta-analyses of the effect size of \emph{d}' and RTs across our 13 experiments.

\hypertarget{methods-2}{%
\subsection{Methods}\label{methods-2}}

\hypertarget{software}{%
\subsubsection{Software}\label{software}}

Mini Meta-analysis was carried out in R 3.6. As for the meta-analysis of the effect size of \emph{d}' and RTs, we used \enquote{metafor} package (Viechtbauer, 2010).

\hypertarget{data-analysis-8}{%
\subsubsection{Data analysis}\label{data-analysis-8}}

We first calculated the mean of \emph{d}' and RT of each condition for each participant, then calculate the effect size (Cohen's d) and variance of the effect size for all contrast we interested: Good v. Bad, Good v. Neutral, and Neutral v. Bad for the effect of valence, and self vs.~other for the effect of self-relevance. Cohen'd and its variance were estimated using the following formula (Cooper, Hedges, \& Valentine, 2009):

\[d = \frac {(M_{1} - M_{2})}{\sqrt {(sd_{1}^2 + sd_{2}^2) - 2*r*sd_{1}*sd_{2}}} * \sqrt {2*(1-r)}\]

\[var.d = 2*(1-r) * (\frac{1}{n} + \frac{d^2}{2*n})\]

\(M_1\) is the mean of the first condition, \(sd_1\) is the standard deviation of the first condition, while \(M_2\) is the mean of the second condition, \(sd_2\) is the standard deviation of the second condition. \(r\) is the correlation coefficient between data from first and second condition. \(n\) is the number of data point (in our case the number of participants).

To avoid the cases that some participants participated more than one experiments, we inspected the all available information about participants. For those participants, only the results from their first participation were included. As mentioned above, 24 participants were intentionally recruited to participate both exp 1a and exp 2, we only included their results from exp 1a in the current meta-analysis.

In total, we conducted 13 meta-analyses for both reaction times and \emph{d} prime for both valence effect and self-relevance effect. For the valence effect, we compared the differences between valences for over all effect as well as for self-referential and other-referential separately. The Good-Bad contrast included 13 experiments (1a - 7b, N = 475) while the Good-Neutral and Neutral-Bad contrasts included 11 experiments (1a \textasciitilde{} 6b, N = 405). Then we combined the experiments with the variable of self-referential, and calculated the effect of valence for self-referential and other-referential separately. For the Good-Bad contrast, both self- and other-referential condition included 7 experiments (3a, 3b, 4a, 4b, 6b, 7a, 7b, N = 282), while for the Good-Neutral and Neutroal contrast, both conditions included 5 experiments (3a, 3b, 4a, 4b, 6b, N = 212).

The self-referential effect was also calculated overall as well as under three valence conditions. The overall self-referential effect and the self-referential effect under good and bad conditions was estimated from 7 experiments (3a, 3b, 4a, 4b, 6b, 7a, 7b, N = 282), while the self-referential effect under the neutral condition were estimated from 5 experiments (3a, 3b, 4a, 4b, 6b, N = 212)

\hypertarget{results-11}{%
\subsection{Results}\label{results-11}}

\begin{figure}

{\centering \includegraphics{Notebook_Pos_Self_Salience_APA_files/figure-latex/meta-all-val-1} 

}

\caption{Meta-analysis of RT and *d* prime for valence effect.}\label{fig:meta-all-val}
\end{figure}

Figure \ref{fig:meta-all-val} shows meta-analytic results for the effect of \emph{d} prime and reaction times from Good-Bad, Good-Neutral, and Neutral-Bad contrast.

Across all experiments, we found that the good-association condition has advantage over bad conditions for both RT (Cohen's d = -0.51, 95\%CI{[}-0.65 -0.37{]}) and \emph{d} prime (Cohen's \emph{d} = 0.23, 95\%CI{[}0.15 0.32{]}). Also the good-association has advantages over the neutral condition for both RT (Cohen's \emph{d} = -0.39, 95\%CI{[}-0.54 -0.23{]}) and \emph{d} prime (Cohen's \emph{d} = 0.27, 95\%CI{[}0.14 0.40{]}). But the neutral condition did not differ from the bad conditions for d prime (Cohen's \emph{d} = -0.03, 95\%CI{[}-0.13 0.08{]}) but slightly faster on RT, RT Cohen's (Cohen's \emph{d} = -0.11, 95\%CI{[}-0.21 -0.01{]}).

When we distinguish between self-referential and other-referential conditions, it is clear that the over all effect was mainly stem from the self-referential conditions: The good-association condition has advantage over bad conditions for both RT (Cohen's d = , 95\%CI{[} {]}) and \emph{d} prime (Cohen's \emph{d} = , 95\%CI{[} {]}), and over neutral condition for both both RT (Cohen's \emph{d} = , 95\%CI{[} {]}) and \emph{d} prime (Cohen's \emph{d} = , 95\%CI{[} {]}), but not for the \emph{d} prime between neutral and bad on RT (Cohen's \emph{d} = , 95\%CI{[} {]}) or \emph{d} prime (Cohen's \emph{d} = , 95\%CI{[} {]}).

For the other condition, no differences were observed for \emph{d} prime: Good vs.~Bad (Cohen's \emph{d} = , 95\%CI{[} {]}); good vs.~neutral (Cohen's \emph{d} = , 95\%CI{[} {]}); neutral vs.~bad (Cohen's \emph{d} = , 95\%CI{[} {]}). But the effect on RT has the similar pattern as the overall effect, with much small effect size on Good vs.~Bad, (Cohen's \emph{d} = , 95\%CI{[} {]}) and Good vs.~Neutral, (Cohen's \emph{d} = , 95\%CI{[} {]}), and similar effect size on neutral vs.~bad condition, (Cohen's \emph{d} = , 95\%CI{[} {]}).

\begin{figure}

{\centering \includegraphics{Notebook_Pos_Self_Salience_APA_files/figure-latex/meta-all-self-ref-1} 

}

\caption{Meta-analysis of RT and *d* prime for self-referential effect.}\label{fig:meta-all-self-ref}
\end{figure}

Figure \ref{fig:meta-all-self-ref} shows meta-analytic results for the effect of \emph{d} prime and reaction times from Good-Bad, Good-Neutral, and Neutral-Bad contrast.

As for the self-relevance effect, we found that there was no overall self-relevance effect on both \emph{d} prime (Cohen's \emph{d} = 0.08, 95\%CI{[}-0.24 0.40{]}) and RT (Cohen's \emph{d} = -0.10, 95\%CI{[}-0.54 0.33{]}). When looking at different valence conditions, we found that self condition was performed better than the other condition for the good condition for \emph{d} (Cohen's \emph{d} = 0.40, 95\%CI{[}0.09 0.71{]}), and also marginal for RT (Cohen's \emph{d} = -0.36, 95\%CI{[}-0.79 0.07{]}). but not for neutral or bad conditions. see Figure \ref{fig:meta-all-self-ref}.

\hypertarget{behavioral-traits-correlation-analysis}{%
\section{Behavioral-Traits correlation analysis}\label{behavioral-traits-correlation-analysis}}

We also analyzed the relationship between behavioral response and self-reported psychological traits. First, we conducted repeated measure ANOVAs for the psychological distance across all experiment, to check the validity of psychological distance. We predicted that the distance between self and good person should be the shortest, while self and bad-person would show longest distance. Second, we conducted a correlation analysis for behavioral data and score data, i.e., between psychological distance and the bias in perceptual matching task.

\hypertarget{references}{%
\section{References}\label{references}}

\begingroup
\setlength{\parindent}{-0.5in}
\setlength{\leftskip}{0.5in}

\hypertarget{refs}{}
\leavevmode\hypertarget{ref-R-papaja}{}%
Aust, F., \& Barth, M. (2018). \emph{papaja: Create APA manuscripts with R Markdown}. Retrieved from \url{https://github.com/crsh/papaja}

\leavevmode\hypertarget{ref-R-Matrix}{}%
Bates, D., \& Maechler, M. (2019). \emph{Matrix: Sparse and dense matrix classes and methods}. Retrieved from \url{https://CRAN.R-project.org/package=Matrix}

\leavevmode\hypertarget{ref-R-lme4}{}%
Bates, D., Mächler, M., Bolker, B., \& Walker, S. (2015). Fitting linear mixed-effects models using lme4. \emph{Journal of Statistical Software}, \emph{67}(1), 1--48. \url{https://doi.org/10.18637/jss.v067.i01}

\leavevmode\hypertarget{ref-R-boot}{}%
Davison, A. C., \& Hinkley, D. V. (1997). \emph{Bootstrap methods and their applications}. Cambridge: Cambridge University Press. Retrieved from \url{http://statwww.epfl.ch/davison/BMA/}

\leavevmode\hypertarget{ref-R-mvtnorm}{}%
Genz, A., \& Bretz, F. (2009). \emph{Computation of multivariate normal and t probabilities}. Heidelberg: Springer-Verlag.

\leavevmode\hypertarget{ref-R-bootES}{}%
Gerlanc, D., \& Kirby, K. (2015). \emph{BootES: Bootstrap effect sizes}. Retrieved from \url{https://CRAN.R-project.org/package=bootES}

\leavevmode\hypertarget{ref-R-Hmisc}{}%
Harrell Jr, F. E., Charles Dupont, \& others. (2019). \emph{Hmisc: Harrell miscellaneous}. Retrieved from \url{https://CRAN.R-project.org/package=Hmisc}

\leavevmode\hypertarget{ref-R-purrr}{}%
Henry, L., \& Wickham, H. (2019). \emph{Purrr: Functional programming tools}. Retrieved from \url{https://CRAN.R-project.org/package=purrr}

\leavevmode\hypertarget{ref-R-ggstance}{}%
Henry, L., Wickham, H., \& Chang, W. (2018). \emph{Ggstance: Horizontal 'ggplot2' components}. Retrieved from \url{https://CRAN.R-project.org/package=ggstance}

\leavevmode\hypertarget{ref-R-TH.data}{}%
Hothorn, T. (2019). \emph{TH.data: TH's data archive}. Retrieved from \url{https://CRAN.R-project.org/package=TH.data}

\leavevmode\hypertarget{ref-R-multcomp}{}%
Hothorn, T., Bretz, F., \& Westfall, P. (2008). Simultaneous inference in general parametric models. \emph{Biometrical Journal}, \emph{50}(3), 346--363.

\leavevmode\hypertarget{ref-R-ggformula}{}%
Kaplan, D., \& Pruim, R. (2019). \emph{Ggformula: Formula interface to the grammar of graphics}. Retrieved from \url{https://CRAN.R-project.org/package=ggformula}

\leavevmode\hypertarget{ref-R-MBESS}{}%
Kelley, K. (2018). \emph{MBESS: The mbess r package}. Retrieved from \url{https://CRAN.R-project.org/package=MBESS}

\leavevmode\hypertarget{ref-R-emmeans}{}%
Lenth, R. (2019). \emph{Emmeans: Estimated marginal means, aka least-squares means}. Retrieved from \url{https://CRAN.R-project.org/package=emmeans}

\leavevmode\hypertarget{ref-R-BayesFactor}{}%
Morey, R. D., \& Rouder, J. N. (2018). \emph{BayesFactor: Computation of bayes factors for common designs}. Retrieved from \url{https://CRAN.R-project.org/package=BayesFactor}

\leavevmode\hypertarget{ref-R-here}{}%
Müller, K. (2017). \emph{Here: A simpler way to find your files}. Retrieved from \url{https://CRAN.R-project.org/package=here}

\leavevmode\hypertarget{ref-R-tibble}{}%
Müller, K., \& Wickham, H. (2019). \emph{Tibble: Simple data frames}. Retrieved from \url{https://CRAN.R-project.org/package=tibble}

\leavevmode\hypertarget{ref-R-RColorBrewer}{}%
Neuwirth, E. (2014). \emph{RColorBrewer: ColorBrewer palettes}. Retrieved from \url{https://CRAN.R-project.org/package=RColorBrewer}

\leavevmode\hypertarget{ref-R-ggstatsplot}{}%
Patil, I., \& Powell, C. (2018). \emph{Ggstatsplot: 'Ggplot2' based plots with statistical details}. \url{https://doi.org/10.5281/zenodo.2074621}

\leavevmode\hypertarget{ref-R-coda}{}%
Plummer, M., Best, N., Cowles, K., \& Vines, K. (2006). CODA: Convergence diagnosis and output analysis for mcmc. \emph{R News}, \emph{6}(1), 7--11. Retrieved from \url{https://journal.r-project.org/archive/}

\leavevmode\hypertarget{ref-R-mosaicData}{}%
Pruim, R., Kaplan, D., \& Horton, N. (2018). \emph{MosaicData: Project mosaic data sets}. Retrieved from \url{https://CRAN.R-project.org/package=mosaicData}

\leavevmode\hypertarget{ref-R-mosaic}{}%
Pruim, R., Kaplan, D. T., \& Horton, N. J. (2017). The mosaic package: Helping students to 'think with data' using r. \emph{The R Journal}, \emph{9}(1), 77--102. Retrieved from \url{https://journal.r-project.org/archive/2017/RJ-2017-024/index.html}

\leavevmode\hypertarget{ref-R-base}{}%
R Core Team. (2018). \emph{R: A language and environment for statistical computing}. Vienna, Austria: R Foundation for Statistical Computing. Retrieved from \url{https://www.R-project.org/}

\leavevmode\hypertarget{ref-R-psych}{}%
Revelle, W. (2018). \emph{Psych: Procedures for psychological, psychometric, and personality research}. Evanston, Illinois: Northwestern University. Retrieved from \url{https://CRAN.R-project.org/package=psych}

\leavevmode\hypertarget{ref-R-lattice}{}%
Sarkar, D. (2008). \emph{Lattice: Multivariate data visualization with r}. New York: Springer. Retrieved from \url{http://lmdvr.r-forge.r-project.org}

\leavevmode\hypertarget{ref-R-afex}{}%
Singmann, H., Bolker, B., Westfall, J., \& Aust, F. (2019). \emph{Afex: Analysis of factorial experiments}. Retrieved from \url{https://CRAN.R-project.org/package=afex}

\leavevmode\hypertarget{ref-R-survival-book}{}%
Terry M. Therneau, \& Patricia M. Grambsch. (2000). \emph{Modeling survival data: Extending the Cox model}. New York: Springer.

\leavevmode\hypertarget{ref-R-MASS}{}%
Venables, W. N., \& Ripley, B. D. (2002). \emph{Modern applied statistics with s} (Fourth). New York: Springer. Retrieved from \url{http://www.stats.ox.ac.uk/pub/MASS4}

\leavevmode\hypertarget{ref-R-corrplot2017}{}%
Wei, T., \& Simko, V. (2017). \emph{R package "corrplot": Visualization of a correlation matrix}. Retrieved from \url{https://github.com/taiyun/corrplot}

\leavevmode\hypertarget{ref-R-reshape2}{}%
Wickham, H. (2007). Reshaping data with the reshape package. \emph{Journal of Statistical Software}, \emph{21}(12), 1--20. Retrieved from \url{http://www.jstatsoft.org/v21/i12/}

\leavevmode\hypertarget{ref-R-plyr}{}%
Wickham, H. (2011). The split-apply-combine strategy for data analysis. \emph{Journal of Statistical Software}, \emph{40}(1), 1--29. Retrieved from \url{http://www.jstatsoft.org/v40/i01/}

\leavevmode\hypertarget{ref-R-ggplot2}{}%
Wickham, H. (2016). \emph{Ggplot2: Elegant graphics for data analysis}. Springer-Verlag New York. Retrieved from \url{https://ggplot2.tidyverse.org}

\leavevmode\hypertarget{ref-R-tidyverse}{}%
Wickham, H. (2017). \emph{Tidyverse: Easily install and load the 'tidyverse'}. Retrieved from \url{https://CRAN.R-project.org/package=tidyverse}

\leavevmode\hypertarget{ref-R-forcats}{}%
Wickham, H. (2019a). \emph{Forcats: Tools for working with categorical variables (factors)}. Retrieved from \url{https://CRAN.R-project.org/package=forcats}

\leavevmode\hypertarget{ref-R-stringr}{}%
Wickham, H. (2019b). \emph{Stringr: Simple, consistent wrappers for common string operations}. Retrieved from \url{https://CRAN.R-project.org/package=stringr}

\leavevmode\hypertarget{ref-R-dplyr}{}%
Wickham, H., François, R., Henry, L., \& Müller, K. (2019). \emph{Dplyr: A grammar of data manipulation}. Retrieved from \url{https://CRAN.R-project.org/package=dplyr}

\leavevmode\hypertarget{ref-R-tidyr}{}%
Wickham, H., \& Henry, L. (2019). \emph{Tidyr: Easily tidy data with 'spread()' and 'gather()' functions}. Retrieved from \url{https://CRAN.R-project.org/package=tidyr}

\leavevmode\hypertarget{ref-R-readr}{}%
Wickham, H., Hester, J., \& Francois, R. (2018). \emph{Readr: Read rectangular text data}. Retrieved from \url{https://CRAN.R-project.org/package=readr}

\leavevmode\hypertarget{ref-R-Formula}{}%
Zeileis, A., \& Croissant, Y. (2010). Extended model formulas in R: Multiple parts and multiple responses. \emph{Journal of Statistical Software}, \emph{34}(1), 1--13. \url{https://doi.org/10.18637/jss.v034.i01}

\endgroup


\end{document}
